% Options for packages loaded elsewhere
\PassOptionsToPackage{unicode}{hyperref}
\PassOptionsToPackage{hyphens}{url}
%
\documentclass[
]{article}
\usepackage{amsmath,amssymb}
\usepackage{iftex}
\ifPDFTeX
  \usepackage[T1]{fontenc}
  \usepackage[utf8]{inputenc}
  \usepackage{textcomp} % provide euro and other symbols
\else % if luatex or xetex
  \usepackage{unicode-math} % this also loads fontspec
  \defaultfontfeatures{Scale=MatchLowercase}
  \defaultfontfeatures[\rmfamily]{Ligatures=TeX,Scale=1}
\fi
\usepackage{lmodern}
\ifPDFTeX\else
  % xetex/luatex font selection
\fi
% Use upquote if available, for straight quotes in verbatim environments
\IfFileExists{upquote.sty}{\usepackage{upquote}}{}
\IfFileExists{microtype.sty}{% use microtype if available
  \usepackage[]{microtype}
  \UseMicrotypeSet[protrusion]{basicmath} % disable protrusion for tt fonts
}{}
\makeatletter
\@ifundefined{KOMAClassName}{% if non-KOMA class
  \IfFileExists{parskip.sty}{%
    \usepackage{parskip}
  }{% else
    \setlength{\parindent}{0pt}
    \setlength{\parskip}{6pt plus 2pt minus 1pt}}
}{% if KOMA class
  \KOMAoptions{parskip=half}}
\makeatother
\usepackage{xcolor}
\usepackage[margin=1in]{geometry}
\usepackage{color}
\usepackage{fancyvrb}
\newcommand{\VerbBar}{|}
\newcommand{\VERB}{\Verb[commandchars=\\\{\}]}
\DefineVerbatimEnvironment{Highlighting}{Verbatim}{commandchars=\\\{\}}
% Add ',fontsize=\small' for more characters per line
\usepackage{framed}
\definecolor{shadecolor}{RGB}{248,248,248}
\newenvironment{Shaded}{\begin{snugshade}}{\end{snugshade}}
\newcommand{\AlertTok}[1]{\textcolor[rgb]{0.94,0.16,0.16}{#1}}
\newcommand{\AnnotationTok}[1]{\textcolor[rgb]{0.56,0.35,0.01}{\textbf{\textit{#1}}}}
\newcommand{\AttributeTok}[1]{\textcolor[rgb]{0.13,0.29,0.53}{#1}}
\newcommand{\BaseNTok}[1]{\textcolor[rgb]{0.00,0.00,0.81}{#1}}
\newcommand{\BuiltInTok}[1]{#1}
\newcommand{\CharTok}[1]{\textcolor[rgb]{0.31,0.60,0.02}{#1}}
\newcommand{\CommentTok}[1]{\textcolor[rgb]{0.56,0.35,0.01}{\textit{#1}}}
\newcommand{\CommentVarTok}[1]{\textcolor[rgb]{0.56,0.35,0.01}{\textbf{\textit{#1}}}}
\newcommand{\ConstantTok}[1]{\textcolor[rgb]{0.56,0.35,0.01}{#1}}
\newcommand{\ControlFlowTok}[1]{\textcolor[rgb]{0.13,0.29,0.53}{\textbf{#1}}}
\newcommand{\DataTypeTok}[1]{\textcolor[rgb]{0.13,0.29,0.53}{#1}}
\newcommand{\DecValTok}[1]{\textcolor[rgb]{0.00,0.00,0.81}{#1}}
\newcommand{\DocumentationTok}[1]{\textcolor[rgb]{0.56,0.35,0.01}{\textbf{\textit{#1}}}}
\newcommand{\ErrorTok}[1]{\textcolor[rgb]{0.64,0.00,0.00}{\textbf{#1}}}
\newcommand{\ExtensionTok}[1]{#1}
\newcommand{\FloatTok}[1]{\textcolor[rgb]{0.00,0.00,0.81}{#1}}
\newcommand{\FunctionTok}[1]{\textcolor[rgb]{0.13,0.29,0.53}{\textbf{#1}}}
\newcommand{\ImportTok}[1]{#1}
\newcommand{\InformationTok}[1]{\textcolor[rgb]{0.56,0.35,0.01}{\textbf{\textit{#1}}}}
\newcommand{\KeywordTok}[1]{\textcolor[rgb]{0.13,0.29,0.53}{\textbf{#1}}}
\newcommand{\NormalTok}[1]{#1}
\newcommand{\OperatorTok}[1]{\textcolor[rgb]{0.81,0.36,0.00}{\textbf{#1}}}
\newcommand{\OtherTok}[1]{\textcolor[rgb]{0.56,0.35,0.01}{#1}}
\newcommand{\PreprocessorTok}[1]{\textcolor[rgb]{0.56,0.35,0.01}{\textit{#1}}}
\newcommand{\RegionMarkerTok}[1]{#1}
\newcommand{\SpecialCharTok}[1]{\textcolor[rgb]{0.81,0.36,0.00}{\textbf{#1}}}
\newcommand{\SpecialStringTok}[1]{\textcolor[rgb]{0.31,0.60,0.02}{#1}}
\newcommand{\StringTok}[1]{\textcolor[rgb]{0.31,0.60,0.02}{#1}}
\newcommand{\VariableTok}[1]{\textcolor[rgb]{0.00,0.00,0.00}{#1}}
\newcommand{\VerbatimStringTok}[1]{\textcolor[rgb]{0.31,0.60,0.02}{#1}}
\newcommand{\WarningTok}[1]{\textcolor[rgb]{0.56,0.35,0.01}{\textbf{\textit{#1}}}}
\usepackage{longtable,booktabs,array}
\usepackage{calc} % for calculating minipage widths
% Correct order of tables after \paragraph or \subparagraph
\usepackage{etoolbox}
\makeatletter
\patchcmd\longtable{\par}{\if@noskipsec\mbox{}\fi\par}{}{}
\makeatother
% Allow footnotes in longtable head/foot
\IfFileExists{footnotehyper.sty}{\usepackage{footnotehyper}}{\usepackage{footnote}}
\makesavenoteenv{longtable}
\usepackage{graphicx}
\makeatletter
\def\maxwidth{\ifdim\Gin@nat@width>\linewidth\linewidth\else\Gin@nat@width\fi}
\def\maxheight{\ifdim\Gin@nat@height>\textheight\textheight\else\Gin@nat@height\fi}
\makeatother
% Scale images if necessary, so that they will not overflow the page
% margins by default, and it is still possible to overwrite the defaults
% using explicit options in \includegraphics[width, height, ...]{}
\setkeys{Gin}{width=\maxwidth,height=\maxheight,keepaspectratio}
% Set default figure placement to htbp
\makeatletter
\def\fps@figure{htbp}
\makeatother
\setlength{\emergencystretch}{3em} % prevent overfull lines
\providecommand{\tightlist}{%
  \setlength{\itemsep}{0pt}\setlength{\parskip}{0pt}}
\setcounter{secnumdepth}{-\maxdimen} % remove section numbering
\ifLuaTeX
  \usepackage{selnolig}  % disable illegal ligatures
\fi
\IfFileExists{bookmark.sty}{\usepackage{bookmark}}{\usepackage{hyperref}}
\IfFileExists{xurl.sty}{\usepackage{xurl}}{} % add URL line breaks if available
\urlstyle{same}
\hypersetup{
  pdftitle={Bellabeat Case Study Report},
  pdfauthor={Nathan},
  hidelinks,
  pdfcreator={LaTeX via pandoc}}

\title{Bellabeat Case Study Report}
\author{Nathan}
\date{2023-07-27}

\begin{document}
\maketitle

\hypertarget{step-1-ask}{%
\section{Step 1: Ask}\label{step-1-ask}}

\hypertarget{business-task}{%
\subsection{Business Task}\label{business-task}}

This analysis was initiated by Urška Sršen, co-founder of Bellabeat.
Sršen knows that an analysis of available data on consumers' smart
device usage would reveal opportunities for the company to grow, and has
provided the following business task:

\begin{quote}
Analyse smart device usage data to gain insight into how people are
already using smart devices, then generate high-level recommendations
for how these insights can inform the marketing strategy for one
Bellabeat product.
\end{quote}

\hypertarget{key-stakeholders}{%
\subsection{Key Stakeholders}\label{key-stakeholders}}

\begin{itemize}
\tightlist
\item
  Urška Sršen

  \begin{itemize}
  \tightlist
  \item
    Co-founder of Bellabeat
  \item
    Initiator of this analysis
  \end{itemize}
\item
  Bellabeat marketing team

  \begin{itemize}
  \tightlist
  \item
    Intended audience for my presentation
  \item
    Will use my insights to guide marketing strategies
  \end{itemize}
\end{itemize}

\hypertarget{step-2-prepare}{%
\section{Step 2: Prepare}\label{step-2-prepare}}

\hypertarget{setting-up-my-tools}{%
\subsection{Setting up my tools}\label{setting-up-my-tools}}

\hypertarget{selecting-my-tools}{%
\subsubsection{Selecting my tools}\label{selecting-my-tools}}

For this analysis I wanted a tool or set of tools with the following
features:

\begin{itemize}
\tightlist
\item
  Sufficient power enough to handle large data sets, e.g.~FitBit data
  tables with \textgreater1M observations
\item
  Functions for data manipulation, e.g.~loading, cleaning and combining
  data sets
\item
  Functions for data analysis, e.g.~regression analysis and statistical
  analysis
\item
  Functions for data visualisation, e.g.~plotting
\item
  Methods for storing the details of my analysis methodology separate
  from the data itself, e.g.~separate source code files, as opposed to
  macros stored within spreadsheet files
\item
  Methods for generating reports from my analysis with as little
  repetition of work as possible, e.g.~inline markdown languages, or
  Page Layout views in spreadsheet applications
\item
  A straightforward learning process and user interface,
  e.g.~spreadsheet tools typically have a single ``Add Chart'' tool
  under which all of their powerful charting options can be found;
  contrast this with learning to download, enable, and finally use the
  ggplot2 R library
\end{itemize}

With these requirements in mind, I considered three tools: R,
spreadsheets, and databases:

\begin{longtable}[]{@{}lccc@{}}
\toprule\noalign{}
Feature & R & Spreadsheets & Databases \\
\midrule\noalign{}
\endhead
\bottomrule\noalign{}
\endlastfoot
Power for large data sets & Yes & No & Yes \\
Data manipulation tools & Yes & Yes & Yes \\
Data analysis tools & Yes & Yes & No \\
Data visualisation tools & Yes & Yes & No \\
Separate analysis files & Yes & No & Yes \\
Streamlined report generation & Yes & No & No \\
Straightforward to learn & No & Yes & No \\
\end{longtable}

Clearly, R is the best single tool for this analysis. R lacks the
straightfoward operation of spreadsheet tools, and I will need to learn
libraries and programming techniques as I do my analysis, but this is
acceptable given my prior experience with other programming languages
like Python.

Note: Python itself was not considered for this analysis: while it
shares most of the features, advantages, and disadvantages of R, I'm
already familiar with Python and wanted to use this case study to
familiarise myself with R instead

\hypertarget{setting-up-my-rstudio-environment}{%
\subsubsection{Setting up my RStudio
environment}\label{setting-up-my-rstudio-environment}}

The first preparation stage involves setting up my RStudio environment
for my analysis.

The R chunk below automatically loads all packages included in the
``rqd\_pkgs'' list, installing them first if required: this ensures all
packages can be loaded by other analysts replicating my work, and
minimises the effort required to modify the package list.

\hypertarget{getting-the-data}{%
\subsection{Getting the data}\label{getting-the-data}}

For this data analysis, I'll be making use of one public data set
specified by Sršen, plus additional data sets as required to address any
limitations found in that dataset.

The data set specified by Sršen is the
\href{https://www.kaggle.com/datasets/arashnic/fitbit}{FitBit Fitness
Tracker Data Set}. This is a public data set available under the
\href{https://creativecommons.org/share-your-work/public-domain/cc/}{CC0
License} via \href{https://www.kaggle.com/arashnic}{Kaggle user Mobius}.

For the initial analysis, the data set was downloaded in its entirety
from Kaggle and stored locally on my computer. This provided a baseline
for analysis, with any modifications to file names, folder
re-structuring, or removal of unnecessary data tables to be conducted
after I'm familiar with the raw data.

\hypertarget{loading-the-data}{%
\subsubsection{Loading the data}\label{loading-the-data}}

To get a top-level view of the data, I first load all of the required
data sets directly into the my R environment. Since all of the required
data tables are in CSV format, I use the readr package to iterate over
all .csv files in my source data directory, loading them into the
environment and naming each new data frame after its source file.

\begin{Shaded}
\begin{Highlighting}[]
\NormalTok{csv\_dir }\OtherTok{\textless{}{-}} \StringTok{"Fitabase\_Data\_Cleaned"}
\NormalTok{paths\_dfs }\OtherTok{\textless{}{-}} \FunctionTok{list.files}\NormalTok{(csv\_dir, }\AttributeTok{pattern =} \StringTok{"*.csv"}\NormalTok{, }\AttributeTok{full.names =} \ConstantTok{TRUE}\NormalTok{)}

\NormalTok{df\_names }\OtherTok{\textless{}{-}}\NormalTok{ paths\_dfs }\SpecialCharTok{\%\textgreater{}\%}
  \FunctionTok{basename}\NormalTok{() }\SpecialCharTok{\%\textgreater{}\%}
\NormalTok{  tools}\SpecialCharTok{::}\FunctionTok{file\_path\_sans\_ext}\NormalTok{() }

\ControlFlowTok{for}\NormalTok{ (i }\ControlFlowTok{in} \DecValTok{1}\SpecialCharTok{:}\FunctionTok{length}\NormalTok{(df\_names)) \{}
  \FunctionTok{assign}\NormalTok{(df\_names[i], }\FunctionTok{read\_csv}\NormalTok{(paths\_dfs[i]))}
\NormalTok{\}}
\FunctionTok{rm}\NormalTok{(i, paths\_dfs)}
\end{Highlighting}
\end{Shaded}

\hypertarget{understanding-the-data}{%
\subsection{Understanding the data}\label{understanding-the-data}}

\hypertarget{overview}{%
\subsubsection{Overview}\label{overview}}

The database contains data on the following features of the FitBit
devices:

\begin{longtable}[]{@{}
  >{\raggedright\arraybackslash}p{(\columnwidth - 4\tabcolsep) * \real{0.3333}}
  >{\raggedright\arraybackslash}p{(\columnwidth - 4\tabcolsep) * \real{0.3200}}
  >{\raggedright\arraybackslash}p{(\columnwidth - 4\tabcolsep) * \real{0.3467}}@{}}
\toprule\noalign{}
\begin{minipage}[b]{\linewidth}\raggedright
Feature
\end{minipage} & \begin{minipage}[b]{\linewidth}\raggedright
Units
\end{minipage} & \begin{minipage}[b]{\linewidth}\raggedright
Sampling Rate
\end{minipage} \\
\midrule\noalign{}
\endhead
\bottomrule\noalign{}
\endlastfoot
BMI & BMI & Manual/automatic logging \\
Body Weight & kg & Manual/automatic logging \\
Body Fat & \% & Manual/automatic logging \\
Calorie burn & Calories & 1 minute \\
Distance & Unknown & 1 minute \\
Heart Rate & Beats per Minute (BPM) & 5 seconds \\
Intensity (of exercise) & Factor, 0 to 3 & 1 minute \\
METs (during exercise) & METs & 1 minute \\
Sleep & Factor, 1 to 3 & 1 minute \\
Steps & Steps & 1 minute \\
\end{longtable}

Some details of the data gathering are not yet clear to me:

\begin{itemize}
\tightlist
\item
  The real-world meanings of the ``Intensity'' factor variable levels;
  presumably 3 is maximum intensity and 0 is inactivity.
\item
  The real-world meanings of the ``Sleep'' factor variable levels are;
  presumably it indicates ``quality of sleep'', with 3 being best.
\item
  What triggers the non-manual reports; presumably it's logged from
  another device, like a body composition analyser.
\end{itemize}

\hypertarget{inconsistent-and-misleading-file-names}{%
\subsubsection{Inconsistent and misleading file
names}\label{inconsistent-and-misleading-file-names}}

The file names in the data set present the following issues:

\begin{itemize}
\tightlist
\item
  Unclear if data is source or summarized,
  e.g.~``minuteCaloriesNarrow\_merged.csv'' is source data, while
  ``dailyCalories\_merged.csv'' is summarized data
\item
  Summarized data not always tied to specific features,
  e.g.~``dailyActivities\_merged.csv'' contains data summarized from
  multiple features: there is no ``Activity'' feature with an associated
  source data set.
\item
  Data shape unclear, e.g.~``minuteCaloriesNarrow\_merged.csv'' and
  ``dailyCalories\_merged.csv'' are both tall data, while
  ``minuteIntensitiesWide\_merged.csv'' and
  ``dailyIntensities\_merged.csv'' are both wide data
\item
  Data sampling rate unclear, e.g.~``minuteSleep\_merged.csv'' compared
  to ``daySleep\_merged.csv''
\item
  Feature is not first in the file name, affecting file sort operations,
  e.g.~``minuteSleep\_merged.csv'' is sorted closer to
  ``minuteStepsNarrow\_merged.csv'' than to the related
  ``daySleep\_merged.csv''
\item
  All files are suffixed with ``\_merged'', so no distinction is made by
  this information and it could be dropped
\end{itemize}

To correct each of these issues, I'll rename the files using this naming
convention:

\begin{quote}
{[}feature{]}\_{[}src/sum{]}\_{[}interval{]}\_{[}shape{]}.{[}filetype{]}
\end{quote}

For example, ``sleepDay\_merged.csv'' will be renamed to
``sleep\_sum\_days\_wide.csv''.

\hypertarget{inconsistent-variable-names}{%
\subsubsection{Inconsistent variable
names}\label{inconsistent-variable-names}}

All variables in the data set are named in CapitalisedCase, whereas I
would typically use snake\_case by convention. This is a relatively
minor issue that I could go without correcting, however a small number
of variable names, like logId, are also inconsistently capitalised
across different tables. Since I'll be adjusting some names anyway, and
there's presumably a library function to change this with minimal
effort, I'll put it on the data-cleaning to-do list.

\hypertarget{inappropriate-variable-types}{%
\subsubsection{Inappropriate variable
types}\label{inappropriate-variable-types}}

Some of the tables in the data set use variables of a type unsuitable
for analysis. These variables and their required modifications are
tabulated below:

\begin{longtable}[]{@{}
  >{\raggedright\arraybackslash}p{(\columnwidth - 6\tabcolsep) * \real{0.3019}}
  >{\raggedright\arraybackslash}p{(\columnwidth - 6\tabcolsep) * \real{0.2830}}
  >{\raggedright\arraybackslash}p{(\columnwidth - 6\tabcolsep) * \real{0.2642}}
  >{\raggedright\arraybackslash}p{(\columnwidth - 6\tabcolsep) * \real{0.1509}}@{}}
\toprule\noalign{}
\begin{minipage}[b]{\linewidth}\raggedright
Variable
\end{minipage} & \begin{minipage}[b]{\linewidth}\raggedright
Original Type
\end{minipage} & \begin{minipage}[b]{\linewidth}\raggedright
Updated Type
\end{minipage} & \begin{minipage}[b]{\linewidth}\raggedright
Reason
\end{minipage} \\
\midrule\noalign{}
\endhead
\bottomrule\noalign{}
\endlastfoot
ActivityDay & chr & datetime & Cannot perform datetime operations on chr
variables \\
ActivityHour & chr & datetime & Cannot perform datetime operations on
chr variables \\
ActivityMinute & chr & datetime & Cannot perform datetime operations on
chr variables \\
Date & chr & datetime & Cannot perform datetime operations on chr
variables \\
Id & num & chr & Disable scientific notation and numerical operations
(IDs are not a numeric value) \\
LogId & num & chr & Disable scientific notation and numerical operations
(IDs are not a numeric value) \\
SleepDay & chr & datetime & Cannot perform datetime operations on chr
variables \\
Time & chr & datetime & Cannot perform datetime operations on chr
variables \\
\end{longtable}

\hypertarget{missing-context-for-numeric-variables}{%
\subsubsection{Missing context for numeric
variables}\label{missing-context-for-numeric-variables}}

All of the numeric variables in the data set have clearly defined units
except for ``Distance''. Distance does not appear to have a source data
table: it's only included in the ``activity\_sum\_wide\_days'' and
``intensity\_sum\_wide\_days'' tables, and only as summary data grouped
via Intensity level. Floating-point values between 0 and 1 are present,
so it seems reasonable to assume this is either kilometers or miles, as
opposed to meters or feet. No geographical information is given in the
data set, so I can't assume the participants are from the U.S., where
miles would be appropriate. Given miles and kilometers represent the
same information on slightly different scales, any insights about
different use cases between users should still be apparent, therefore I
think it's reasonable to assume the distances are given in kilometers
for this analysis.

\hypertarget{missing-context-for-factor-variables}{%
\subsubsection{Missing context for factor
variables}\label{missing-context-for-factor-variables}}

Two variables in the data set, exercise ``Intensity'' and sleep
``Value'', are numerical factors with no defined range. For these
factors, I can't tell from the data alone whether all possible values
that a FitBit can record are present. The values for exercise intensity,
for example, range from 0 to 3; it could be the case that the FitBits
used only generate four levels of intensity, but it could just as easily
be the case that the values go up to 100 (i.e.~a percentage). This has
clear implications for our analysis: if 3 is the max, records of 3
indicate users wear their FitBits while exercising as hard as they can,
whereas if 100 is the max, records of 3 indicate users wear their
FitBits while sitting on the couch as hard as they can.

With this in mind, I'm going to invest a bit of time to confirm the
meaning of these variables before attempting to analyse them (read:
about ten hours to investigate everything \emph{and} learn how to do it
all in R \emph{and} learn how to make it look suitably pretty and
coherent in RMarkdown).

\hypertarget{validating-exercise-intensity-by-use-of-r}{%
\paragraph{Validating ``Exercise Intensity'' by use of
R}\label{validating-exercise-intensity-by-use-of-r}}

I'll start by determining the range of values present in the Intensity
data:

\begin{Shaded}
\begin{Highlighting}[]
\FunctionTok{cat}\NormalTok{(}\StringTok{"Min Intensity:"}\NormalTok{, }\FunctionTok{min}\NormalTok{(intensity\_src\_mins\_tall}\SpecialCharTok{$}\NormalTok{Intensity),}
    \StringTok{"}\SpecialCharTok{\textbackslash{}n}\StringTok{Max Intensity:"}\NormalTok{, }\FunctionTok{max}\NormalTok{(intensity\_src\_mins\_tall}\SpecialCharTok{$}\NormalTok{Intensity), }\StringTok{"}\SpecialCharTok{\textbackslash{}n}\StringTok{"}\NormalTok{, }\AttributeTok{sep =} \StringTok{""}\NormalTok{)}
\end{Highlighting}
\end{Shaded}

The ``activity\_sum\_days\_wide'' table provides some context clues as
to what the levels might mean: the table summarises Intensity data into
four new variables which, based on their names, appear to be associated
with intensity levels like so:

\begin{enumerate}
\def\labelenumi{\arabic{enumi}.}
\tightlist
\item
  Sedentary Minutes
\item
  Lightly Active Minutes
\item
  Fairly Active Minutes
\item
  Very Active Minutes
\end{enumerate}

Let's see if I can confirm this by recreating the data with that naming
convention:

\begin{Shaded}
\begin{Highlighting}[]
\CommentTok{\# Generate my version of sleepDay\_merged for comparison with the original}
\NormalTok{intensity\_daily\_sum\_wide }\OtherTok{\textless{}{-}}\NormalTok{ intensity\_src\_mins\_tall }\SpecialCharTok{\%\textgreater{}\%}
    \FunctionTok{mutate}\NormalTok{(}\AttributeTok{activity\_date\_floored =} \FunctionTok{floor\_date}\NormalTok{(}\FunctionTok{mdy\_hms}\NormalTok{(ActivityMinute), }\AttributeTok{unit =} \StringTok{"days"}\NormalTok{)) }\SpecialCharTok{\%\textgreater{}\%}
    \FunctionTok{group\_by}\NormalTok{(Id, activity\_date\_floored) }\SpecialCharTok{\%\textgreater{}\%}
    \FunctionTok{summarize}\NormalTok{(}
        \AttributeTok{minutes\_sedentary      =} \FunctionTok{sum}\NormalTok{(}\FunctionTok{case\_when}\NormalTok{(Intensity }\SpecialCharTok{==} \DecValTok{0} \SpecialCharTok{\textasciitilde{}} \DecValTok{1}\NormalTok{, }\ConstantTok{TRUE} \SpecialCharTok{\textasciitilde{}} \DecValTok{0}\NormalTok{)),}
        \AttributeTok{minutes\_lightly\_active =} \FunctionTok{sum}\NormalTok{(}\FunctionTok{case\_when}\NormalTok{(Intensity }\SpecialCharTok{==} \DecValTok{1} \SpecialCharTok{\textasciitilde{}} \DecValTok{1}\NormalTok{, }\ConstantTok{TRUE} \SpecialCharTok{\textasciitilde{}} \DecValTok{0}\NormalTok{)),}
        \AttributeTok{minutes\_fairly\_active  =} \FunctionTok{sum}\NormalTok{(}\FunctionTok{case\_when}\NormalTok{(Intensity }\SpecialCharTok{==} \DecValTok{2} \SpecialCharTok{\textasciitilde{}} \DecValTok{1}\NormalTok{, }\ConstantTok{TRUE} \SpecialCharTok{\textasciitilde{}} \DecValTok{0}\NormalTok{)),}
        \AttributeTok{minutes\_very\_active    =} \FunctionTok{sum}\NormalTok{(}\FunctionTok{case\_when}\NormalTok{(Intensity }\SpecialCharTok{==} \DecValTok{3} \SpecialCharTok{\textasciitilde{}} \DecValTok{1}\NormalTok{, }\ConstantTok{TRUE} \SpecialCharTok{\textasciitilde{}} \DecValTok{0}\NormalTok{))}
\NormalTok{    ) }\SpecialCharTok{\%\textgreater{}\%}
    \FunctionTok{mutate}\NormalTok{(}\AttributeTok{Id\_ActivityDate\_UID =} \FunctionTok{paste}\NormalTok{(Id, activity\_date\_floored, }\AttributeTok{sep =} \StringTok{"\_"}\NormalTok{)) }\SpecialCharTok{\%\textgreater{}\%}
    \FunctionTok{arrange}\NormalTok{(Id, activity\_date\_floored, Id\_ActivityDate\_UID)}

\CommentTok{\# Compare both versions of the data and return any dates with different values}
\NormalTok{intensity\_daily\_comp }\OtherTok{\textless{}{-}}\NormalTok{ activity\_sum\_days\_wide }\SpecialCharTok{\%\textgreater{}\%}
    \FunctionTok{mutate}\NormalTok{(}\AttributeTok{ActivityDate\_floored =} \FunctionTok{floor\_date}\NormalTok{(}\FunctionTok{mdy}\NormalTok{(ActivityDate), }\AttributeTok{unit =} \StringTok{"days"}\NormalTok{)) }\SpecialCharTok{\%\textgreater{}\%}
    \FunctionTok{mutate}\NormalTok{(}\AttributeTok{Id\_ActivityDate\_UID =} \FunctionTok{paste}\NormalTok{(Id, ActivityDate\_floored, }\AttributeTok{sep =} \StringTok{"\_"}\NormalTok{)) }\SpecialCharTok{\%\textgreater{}\%}
    \FunctionTok{with}\NormalTok{(}\FunctionTok{merge}\NormalTok{(}
\NormalTok{        .,}
\NormalTok{        intensity\_daily\_sum\_wide,}
        \AttributeTok{by =} \FunctionTok{c}\NormalTok{(}\StringTok{"Id\_ActivityDate\_UID"}\NormalTok{),}
        \AttributeTok{all =} \ConstantTok{TRUE}
\NormalTok{    )}
\NormalTok{    ) }\SpecialCharTok{\%\textgreater{}\%}
    \FunctionTok{mutate}\NormalTok{(}\AttributeTok{diff\_minutes\_sedentary      =}\NormalTok{ minutes\_sedentary      }\SpecialCharTok{{-}}\NormalTok{ SedentaryMinutes     ) }\SpecialCharTok{\%\textgreater{}\%}
    \FunctionTok{mutate}\NormalTok{(}\AttributeTok{diff\_minutes\_lightly\_active =}\NormalTok{ minutes\_lightly\_active }\SpecialCharTok{{-}}\NormalTok{ LightlyActiveMinutes ) }\SpecialCharTok{\%\textgreater{}\%}
    \FunctionTok{mutate}\NormalTok{(}\AttributeTok{diff\_minutes\_fairly\_active  =}\NormalTok{ minutes\_fairly\_active  }\SpecialCharTok{{-}}\NormalTok{ FairlyActiveMinutes  ) }\SpecialCharTok{\%\textgreater{}\%}
    \FunctionTok{mutate}\NormalTok{(}\AttributeTok{diff\_minutes\_very\_active    =}\NormalTok{ minutes\_very\_active    }\SpecialCharTok{{-}}\NormalTok{ VeryActiveMinutes    ) }\SpecialCharTok{\%\textgreater{}\%}
    \FunctionTok{select}\NormalTok{(}
\NormalTok{        Id\_ActivityDate\_UID,}
\NormalTok{        diff\_minutes\_sedentary,}
\NormalTok{        diff\_minutes\_lightly\_active,}
\NormalTok{        diff\_minutes\_fairly\_active,}
\NormalTok{        diff\_minutes\_very\_active}
\NormalTok{    ) }\SpecialCharTok{\%\textgreater{}\%}
    \FunctionTok{filter}\NormalTok{(}\SpecialCharTok{!}\NormalTok{(diff\_minutes\_sedentary }\SpecialCharTok{==} \DecValTok{0} \SpecialCharTok{\&} 
\NormalTok{                 diff\_minutes\_lightly\_active }\SpecialCharTok{==} \DecValTok{0} \SpecialCharTok{\&} 
\NormalTok{                 diff\_minutes\_fairly\_active }\SpecialCharTok{==} \DecValTok{0} \SpecialCharTok{\&} 
\NormalTok{                 diff\_minutes\_very\_active }\SpecialCharTok{==} \DecValTok{0}\NormalTok{)}
\NormalTok{    ) }\SpecialCharTok{\%\textgreater{}\%}
    \FunctionTok{arrange}\NormalTok{(Id\_ActivityDate\_UID)}

\CommentTok{\# For this table, glimpse() shows enough to demonstrate the validity of the method}
\FunctionTok{glimpse}\NormalTok{(intensity\_daily\_comp)}
\FunctionTok{rm}\NormalTok{(intensity\_daily\_sum\_wide, intensity\_daily\_comp)}
\end{Highlighting}
\end{Shaded}

The approach above appears to work perfectly, with the exception of the
``sedentary minutes'' calculation, which is consistently higher in my
version.

I think its safe to conclude that I got the mapping correct, given that:
a) It's consistently higher by at least 6 hours, which I suspect is
caused by my method counting time asleep as ``sedentary minutes'' -
which is not \emph{technically} wrong, you know - and b) Reversing the
order of the mapping produces entirely wrong results.

That being said, I still don't know if these are the categories FitBits
work in, not another naming convention that the data authors came up
with, and I don't know if all FitBit models work this way, so let's do
something I should have done from the start: read the manuals.

\hypertarget{validating-exercise-intensity-by-reading-of-manuals}{%
\paragraph{Validating ``Exercise Intensity'' by reading of
manuals}\label{validating-exercise-intensity-by-reading-of-manuals}}

Activity trackers like FitBits detect activity intensity partly by
measuring the user's heart rate while exercising: a higher heart rate
corresponds with a higher degree of exertion. As of April 2016, the
three latest FitBit models with heart-rate tracking were:

\begin{itemize}
\tightlist
\item
  FitBit Blaze
  \href{https://www.youtube.com/watch?v=3k3DNT54NkA}{(released January
  2016)}
\item
  FitBit Charge HR
  \href{https://blog.fitbit.com/charge-hr-and-surge-available-now-plus-new-charge-colors/}{(released
  January 2015)}
\item
  FitBit Surge
  \href{https://blog.fitbit.com/charge-hr-and-surge-available-now-plus-new-charge-colors/}{(released
  January 2015)}
\end{itemize}

A quick look through the product manuals for each model confirms they
all break down user activity into four default heart-rate zones:

\begin{longtable}[]{@{}
  >{\raggedright\arraybackslash}p{(\columnwidth - 8\tabcolsep) * \real{0.2258}}
  >{\centering\arraybackslash}p{(\columnwidth - 8\tabcolsep) * \real{0.1935}}
  >{\centering\arraybackslash}p{(\columnwidth - 8\tabcolsep) * \real{0.1935}}
  >{\centering\arraybackslash}p{(\columnwidth - 8\tabcolsep) * \real{0.1935}}
  >{\centering\arraybackslash}p{(\columnwidth - 8\tabcolsep) * \real{0.1935}}@{}}
\toprule\noalign{}
\begin{minipage}[b]{\linewidth}\raggedright
Product
\end{minipage} & \begin{minipage}[b]{\linewidth}\centering
HR Zone 1
\end{minipage} & \begin{minipage}[b]{\linewidth}\centering
HR Zone 2
\end{minipage} & \begin{minipage}[b]{\linewidth}\centering
HR Zone 3
\end{minipage} & \begin{minipage}[b]{\linewidth}\centering
HR Zone 4
\end{minipage} \\
\midrule\noalign{}
\endhead
\bottomrule\noalign{}
\endlastfoot
\href{https://staticcs.fitbit.com/content/assets/help/manuals/manual_blaze_en_US.pdf}{FitBit
Blaze} & ``Out of Zone'' & ``Fat burn'' & ``Cardio'' & ``Peak'' \\
\href{https://staticcs.fitbit.com/content/assets/help/manuals/manual_charge_hr_en_US.pdf}{FitBit
Charge HR} & ``Out of Zone'' & ``Fat burn'' & ``Cardio'' & ``Peak'' \\
\href{https://myhelp.fitbit.com/resource/manual_surge_en_US}{FitBit
Surge} & ``Out of Zone'' & ``Fat burn'' & ``Cardio'' & ``Peak'' \\
\end{longtable}

While these aren't exactly the same terms as used in the data set,
they're clearly related - ``Out of Zone'' equates to ``Sedentary'', for
example.

All three FitBit manuals also make the same claim that the default zones
are ``based on American Heart Association recommendations''. Even
without
\href{https://www.heart.org/en/healthy-living/fitness/fitness-basics/aha-recs-for-physical-activity-in-adults}{validating
that claim}, it indicates to me that the reasoning behind each zone is
not arbitrary, and is consistent across devices, so I think I can assume
any other FitBit models circa 2016 would follow the same classification
scheme.

At this point, I'm satisfied that the below are the only four intensity
levels I need to consider when analysing the data set, regardless of
what models of FitBits were being used:

\begin{enumerate}
\def\labelenumi{\arabic{enumi}.}
\setcounter{enumi}{-1}
\tightlist
\item
  Sedentary Minutes
\item
  Lightly Active Minutes
\item
  Fairly Active Minutes
\item
  Very Active Minutes
\end{enumerate}

\hypertarget{validating-sleep-quality-by-use-of-r-and-reading-of-manuals}{%
\paragraph{\texorpdfstring{Validating ``Sleep Quality'' by use of R
\emph{and} reading of
manuals}{Validating ``Sleep Quality'' by use of R and reading of manuals}}\label{validating-sleep-quality-by-use-of-r-and-reading-of-manuals}}

As with exercise intensity, I start by determining the range of values
present in the data:

\begin{Shaded}
\begin{Highlighting}[]
\FunctionTok{cat}\NormalTok{(}\StringTok{"Min Sleep:"}\NormalTok{, }\FunctionTok{min}\NormalTok{(sleep\_src\_mins\_tall}\SpecialCharTok{$}\NormalTok{value),}
    \StringTok{"}\SpecialCharTok{\textbackslash{}n}\StringTok{Max Sleep:"}\NormalTok{, }\FunctionTok{max}\NormalTok{(sleep\_src\_mins\_tall}\SpecialCharTok{$}\NormalTok{value), }\StringTok{"}\SpecialCharTok{\textbackslash{}n}\StringTok{"}\NormalTok{, }\AttributeTok{sep =} \StringTok{""}\NormalTok{)}
\end{Highlighting}
\end{Shaded}

\begin{verbatim}
## Min Sleep:1
## Max Sleep:3
\end{verbatim}

The values for sleep quality range from 1 to 3. The summary data tables
for sleep quality introduce only two new variable names: ``Total Time In
Bed'', and ``Total Minutes Asleep''. There's not a valid name for each
level of factor like there was for exercise, so in this case we go
straight back to the manuals:

\begin{itemize}
\tightlist
\item
  Blaze: tracks ``both your time spent asleep and your sleep quality''
\item
  Charge HR: tracks ``the hours you sleep and your movement during the
  night''
\item
  Surge: tracks ``the hours you sleep and your movement during the
  night''
\end{itemize}

Further poking around the FitBit help pages on
\href{https://help.fitbit.com/articles/en_US/Help_article/1314.htm}{how
to track sleep stats} and
\href{https://help.fitbit.com/articles/en_US/Help_article/2163.htm}{what
they all mean} reveals that different devices track slightly different
data if they have heart rate tracking:

\begin{itemize}
\tightlist
\item
  No HR tracking: Generic sleep quality tracking with ``Time spent
  awake, restless, and asleep'' categories
\item
  HR tracking: Sleep stage tracking with ``Light Sleep, Deep Sleep, and
  REM Sleep'' stages
\end{itemize}

The help pages also single out the Charge HR and the Surge as the only
HR-tracking FitBits to \emph{not} have full sleep stage tracking,
leaving the Blaze as the only device from this time period with that
feature. Blaze aside, motion-based sleep quality tracking appears to go
all the way back to the
\href{https://myhelp.fitbit.com/s/products?language=en_US\&p=one}{FitBit
One}. Given this information, it seems fair to assume the following
mapping for the sleep data:

\begin{itemize}
\tightlist
\item
  1: Awake
\item
  2: Restless
\item
  3: Asleep
\end{itemize}

I can confirm this mapping by attempting to recreare duplicate the
summary data in sleepDay\_merged:

As with the intensity data, I can confirm this by recreating the data
with that naming convention:

\begin{Shaded}
\begin{Highlighting}[]
\CommentTok{\# Generate my version of sleepDay\_merged for comparison with the original}
\NormalTok{sleep\_src\_mins\_tall\_NW }\OtherTok{\textless{}{-}}\NormalTok{ sleep\_src\_mins\_tall }\SpecialCharTok{\%\textgreater{}\%}
    \CommentTok{\# mutate(date\_typed = mdy\_hms(date)) \%\textgreater{}\%}
    \FunctionTok{mutate}\NormalTok{(}\AttributeTok{date\_floored =} \FunctionTok{floor\_date}\NormalTok{(}\FunctionTok{mdy\_hms}\NormalTok{(date), }\AttributeTok{unit =} \StringTok{"days"}\NormalTok{)) }\SpecialCharTok{\%\textgreater{}\%}
    \CommentTok{\# Sum time asleep for each Log ID}
    \FunctionTok{group\_by}\NormalTok{(logId) }\SpecialCharTok{\%\textgreater{}\%}
    \FunctionTok{summarize}\NormalTok{(}
        \StringTok{"Id"} \OtherTok{=} \FunctionTok{min}\NormalTok{(Id),}
        \CommentTok{\# Associate each Log ID with the latest date recorded under it}
        \StringTok{"SleepDay"} \OtherTok{=} \FunctionTok{max}\NormalTok{(date\_floored),}
        \StringTok{"minutes\_in\_bed"}   \OtherTok{=} \FunctionTok{n}\NormalTok{(),}
        \StringTok{"minutes\_awake"}    \OtherTok{=} \FunctionTok{sum}\NormalTok{(}\FunctionTok{case\_when}\NormalTok{(value }\SpecialCharTok{==} \DecValTok{3} \SpecialCharTok{\textasciitilde{}} \DecValTok{1}\NormalTok{, }\ConstantTok{TRUE} \SpecialCharTok{\textasciitilde{}} \DecValTok{0}\NormalTok{)),}
        \StringTok{"minutes\_restless"} \OtherTok{=} \FunctionTok{sum}\NormalTok{(}\FunctionTok{case\_when}\NormalTok{(value }\SpecialCharTok{==} \DecValTok{2} \SpecialCharTok{\textasciitilde{}} \DecValTok{1}\NormalTok{, }\ConstantTok{TRUE} \SpecialCharTok{\textasciitilde{}} \DecValTok{0}\NormalTok{)),}
        \StringTok{"minutes\_asleep"}   \OtherTok{=} \FunctionTok{sum}\NormalTok{(}\FunctionTok{case\_when}\NormalTok{(value }\SpecialCharTok{==} \DecValTok{1} \SpecialCharTok{\textasciitilde{}} \DecValTok{1}\NormalTok{, }\ConstantTok{TRUE} \SpecialCharTok{\textasciitilde{}} \DecValTok{0}\NormalTok{))}
\NormalTok{    ) }\SpecialCharTok{\%\textgreater{}\%}
    \CommentTok{\# Sum time asleep for each date based on SleepDay}
    \FunctionTok{group\_by}\NormalTok{(Id, SleepDay) }\SpecialCharTok{\%\textgreater{}\%}
    \FunctionTok{summarize}\NormalTok{(}
        \StringTok{"TotalSleepRecords\_2"} \OtherTok{=} \FunctionTok{n}\NormalTok{(),}
        \StringTok{"TotalMinutesAsleep\_2"} \OtherTok{=} \FunctionTok{sum}\NormalTok{(minutes\_asleep),}
        \StringTok{"TotalTimeInBed\_2"} \OtherTok{=} \FunctionTok{sum}\NormalTok{(minutes\_in\_bed),}
        \StringTok{"TotalMinutesAwake"} \OtherTok{=} \FunctionTok{sum}\NormalTok{(minutes\_awake),}
        \StringTok{"TotalMinutesRestless"} \OtherTok{=} \FunctionTok{sum}\NormalTok{(minutes\_restless),}
\NormalTok{    ) }\SpecialCharTok{\%\textgreater{}\%}
    \FunctionTok{mutate}\NormalTok{(}\StringTok{"Id\_SleepDay\_UID"} \OtherTok{=} \FunctionTok{paste}\NormalTok{(Id, SleepDay, }\AttributeTok{sep =} \StringTok{"\_"}\NormalTok{)) }\SpecialCharTok{\%\textgreater{}\%}
    \FunctionTok{arrange}\NormalTok{(Id\_SleepDay\_UID)}

\CommentTok{\# Compare both versions of the data and return any dates with different values}
\NormalTok{sleepDay\_comp }\OtherTok{\textless{}{-}}\NormalTok{ sleep\_sum\_days\_wide }\SpecialCharTok{\%\textgreater{}\%}
    \CommentTok{\# mutate("SleepDay\_typed" = mdy\_hms(SleepDay)) \%\textgreater{}\%}
    \FunctionTok{mutate}\NormalTok{(}\StringTok{"SleepDay\_floored"} \OtherTok{=} \FunctionTok{floor\_date}\NormalTok{(}\FunctionTok{mdy\_hms}\NormalTok{(SleepDay), }\AttributeTok{unit =} \StringTok{"days"}\NormalTok{)) }\SpecialCharTok{\%\textgreater{}\%}
    \FunctionTok{mutate}\NormalTok{(}\StringTok{"Id\_SleepDay\_UID"} \OtherTok{=} \FunctionTok{paste}\NormalTok{(Id, SleepDay\_floored, }\AttributeTok{sep =} \StringTok{"\_"}\NormalTok{)) }\SpecialCharTok{\%\textgreater{}\%}
    \FunctionTok{arrange}\NormalTok{(Id\_SleepDay\_UID) }\SpecialCharTok{\%\textgreater{}\%}
    \FunctionTok{with}\NormalTok{(}\FunctionTok{merge}\NormalTok{(}
\NormalTok{        .,}
\NormalTok{        sleep\_src\_mins\_tall\_NW,}
        \AttributeTok{by =} \FunctionTok{c}\NormalTok{(}\StringTok{"Id\_SleepDay\_UID"}\NormalTok{),}
        \AttributeTok{all =} \ConstantTok{TRUE}
\NormalTok{    )}
\NormalTok{    ) }\SpecialCharTok{\%\textgreater{}\%}
    \FunctionTok{mutate}\NormalTok{(}\AttributeTok{recordDiff =}\NormalTok{ TotalSleepRecords\_2 }\SpecialCharTok{{-}}\NormalTok{ TotalSleepRecords) }\SpecialCharTok{\%\textgreater{}\%}
    \FunctionTok{mutate}\NormalTok{(}\AttributeTok{sleepDiff =}\NormalTok{ TotalMinutesAsleep\_2 }\SpecialCharTok{{-}}\NormalTok{ TotalMinutesAsleep) }\SpecialCharTok{\%\textgreater{}\%}
    \FunctionTok{mutate}\NormalTok{(}\AttributeTok{bedDiff =}\NormalTok{ TotalTimeInBed\_2 }\SpecialCharTok{{-}}\NormalTok{ TotalTimeInBed) }\SpecialCharTok{\%\textgreater{}\%}
    \FunctionTok{select}\NormalTok{(}
\NormalTok{        Id\_SleepDay\_UID,}
\NormalTok{        recordDiff,}
\NormalTok{        sleepDiff,}
\NormalTok{        bedDiff}
\NormalTok{    ) }\SpecialCharTok{\%\textgreater{}\%}
    \FunctionTok{filter}\NormalTok{(}\SpecialCharTok{!}\NormalTok{(recordDiff }\SpecialCharTok{==} \DecValTok{0} \SpecialCharTok{\&}\NormalTok{ sleepDiff }\SpecialCharTok{==} \DecValTok{0} \SpecialCharTok{\&}\NormalTok{ bedDiff }\SpecialCharTok{==} \DecValTok{0}\NormalTok{)) }\SpecialCharTok{\%\textgreater{}\%}
    \FunctionTok{arrange}\NormalTok{(Id\_SleepDay\_UID)}

\CommentTok{\# For this table, glimpse() shows enough to demonstrate the validity of the method}
\FunctionTok{glimpse}\NormalTok{(sleepDay\_comp)}
\FunctionTok{rm}\NormalTok{(sleep\_src\_mins\_tall\_NW, sleepDay\_comp)}
\end{Highlighting}
\end{Shaded}

I was able to recreate the existing sleep\_src\_mins\_tall table almost
perfectly by summing sleep times per Log ID, with the latest date
associated with each Log ID being used as the ``date'' for that sleep.
In practice it turns out I had the mapping inverted, and actually the
following is used:

\begin{itemize}
\tightlist
\item
  1: Asleep
\item
  2: Restless
\item
  3: Awake
\end{itemize}

So I guess read that as ``1 is highest-quality sleep, 3 is
worst-quality''.

My version of the data contains a few rows that vary slightly from the
original, by 1 to 22 minutes. I haven't been able to determine the
source of this error, but they're more than close enough to confirm I
don't have the factor level mapping backwards, so I'm ready to proceed.

\hypertarget{duplicate-data-between-tables}{%
\subsubsection{Duplicate data between
tables}\label{duplicate-data-between-tables}}

The data set includes some tables that contain source data for a given
feature, e.g.~heart-rate tracking, and others that contain summary data,
e.g.~everything in the ``activity\_days\_sum\_wide'' table.

Some of these are useful, for instance:

\begin{itemize}
\tightlist
\item
  dailyActivities\_merged.csv calculates Sedentary Minutes by excluding
  time spent asleep. This either requires clever/time-consuming
  cross-referencing with other tables, or is raw data from a calculation
  performed on the FitBit itself: either way, I don't want to have to do
  it again
\item
  sleepDay\_merged.csv summarises sleep on a per-night basis, which is
  more useful for comparing users than the raw, minute-by-minute sleep
  data
\end{itemize}

Others are not useful, for instance:

\begin{itemize}
\tightlist
\item
  calories\_sum\_mins\_wide.csv, intensity\_sum\_mins\_wide.csv, and
  steps\_sum\_mins\_wide.csv all just pivot minute-level data into one
  60-column row per hour containing the same data
\item
  minuteIntensitiesWide\_merged.csv just sums and averages intensity per
  hour
\end{itemize}

Those tables that do not provide useful summaries can be excluded from
the data analysis: if a specific need is found for their data, they can
be reloaded or recreated manually as required.

\hypertarget{duplicate-data-within-tables}{%
\subsubsection{Duplicate data within
tables}\label{duplicate-data-within-tables}}

The ``bodycomp\_logs\_src\_wide.csv'' file contains both kilogram and
pound variables: these describe the same information, and all of the
observations contain values for both variables, so one variable can be
dropped with no loss of data. The choice between the two formats seems
arbitrary for my analysis, so I'm choosing to keep the kilos data as its
expressed in an SI unit.

intensity\_sum\_hours\_wide.csv contains Total Intensity and Average
Intensity. Total intensity values exceed 4, the maximum for intensity,
and so are not actually useful. Average Intensity is just the Total
Intensity for each hour divided by 60 minutes per hour.

\hypertarget{step-3-clean}{%
\section{Step 3: Clean}\label{step-3-clean}}

\hypertarget{todo-data-cleaning-checklist}{%
\subsection{TODO: Data Cleaning
Checklist}\label{todo-data-cleaning-checklist}}

\begin{itemize}
\tightlist
\item
  Convert list into changelog format
\end{itemize}

\hypertarget{update-file-names}{%
\subsection{Update file names}\label{update-file-names}}

For this analysis, I'll rename the files to use this naming convention:

\begin{quote}
{[}feature{]}\_{[}src/sum{]}\_{[}interval{]}\_{[}shape{]}.{[}filetype{]}
\end{quote}

Applying the naming convention to the data set yields the following file
names:

\begin{longtable}[]{@{}ll@{}}
\toprule\noalign{}
Original & Updated \\
\midrule\noalign{}
\endhead
\bottomrule\noalign{}
\endlastfoot
dailyActivity\_merged.csv & activity\_sum\_days\_wide.csv \\
dailyCalories\_merged.csv & calories\_sum\_days\_tall.csv \\
dailyIntensities\_merged.csv & intensity\_sum\_days\_wide.csv \\
dailySteps\_merged.csv & steps\_sum\_days\_tall.csv \\
heartrate\_seconds\_merged.csv & heartrate\_src\_seconds\_tall.csv \\
hourlyCalories\_merged.csv & calories\_sum\_hours\_tall.csv \\
hourlyIntensities\_merged.csv & intensity\_sum\_hours\_wide.csv \\
hourlySteps\_merged.csv & steps\_sum\_hours\_tall.csv \\
minuteCaloriesNarrow\_merged.csv & calories\_src\_mins\_tall.csv \\
minuteCaloriesWide\_merged.csv & calories\_sum\_mins\_wide.csv \\
minuteIntensitiesNarrow\_merged.csv & intensity\_src\_mins\_tall.csv \\
minuteIntensitiesWide\_merged.csv & intensity\_sum\_mins\_wide.csv \\
minuteMETsNarrow\_merged.csv & mets\_src\_mins\_tall.csv \\
minuteSleep\_merged.csv & sleep\_src\_mins\_tall.csv \\
minuteStepsNarrow\_merged.csv & steps\_src\_mins\_tall.csv \\
minuteStepsWide\_merged.csv & steps\_sum\_mins\_wide.csv \\
sleepDay\_merged.csv & sleep\_sum\_days\_wide.csv \\
weightLogInfo\_merged.csv & bodycomp\_src\_logs\_wide.csv \\
\end{longtable}

The conversion was performed manually on my local device.

\hypertarget{check-for-missing-data}{%
\subsection{Check for missing data}\label{check-for-missing-data}}

Each data frame was checked for NULL values and non-null empty strings.

The only table with missing data was ``bodycomp\_src\_logs\_wide'',
which was missing all but two of the data points for Body Fat
Percentage. Given this lack of data, Body Fat Percentage will be
excluded from the data analysis.

\begin{Shaded}
\begin{Highlighting}[]
\CommentTok{\# Check for NULLs}
\FunctionTok{cat}\NormalTok{(}\StringTok{"Checking for NULL/empty values...}\SpecialCharTok{\textbackslash{}n}\StringTok{"}\NormalTok{, }\AttributeTok{sep=}\StringTok{""}\NormalTok{)}
\end{Highlighting}
\end{Shaded}

\begin{verbatim}
## Checking for NULL/empty values...
\end{verbatim}

\begin{Shaded}
\begin{Highlighting}[]
\ControlFlowTok{for}\NormalTok{ (df\_name }\ControlFlowTok{in}\NormalTok{ df\_names) \{}
\NormalTok{  df }\OtherTok{\textless{}{-}} \FunctionTok{get}\NormalTok{(df\_name)}
\NormalTok{  num\_nulls }\OtherTok{\textless{}{-}} \FunctionTok{sum}\NormalTok{(}\FunctionTok{is.na}\NormalTok{(df))}
  \ControlFlowTok{if}\NormalTok{(num\_nulls) \{}
    \FunctionTok{cat}\NormalTok{(df\_name,}\StringTok{": "}\NormalTok{,num\_nulls,}\StringTok{" NULLs"}\NormalTok{,}\StringTok{"}\SpecialCharTok{\textbackslash{}n}\StringTok{"}\NormalTok{,}\AttributeTok{sep=}\StringTok{""}\NormalTok{)}
\NormalTok{  \}}
  \CommentTok{\# Check for empty strings (which do not show up as NULLs)}
\NormalTok{  empty\_strings }\OtherTok{\textless{}{-}}\NormalTok{ df }\SpecialCharTok{\%\textgreater{}\%}
    \FunctionTok{filter}\NormalTok{(}\FunctionTok{if\_any}\NormalTok{(}\FunctionTok{where}\NormalTok{(is.character), }\SpecialCharTok{\textasciitilde{}} \FunctionTok{nchar}\NormalTok{(.) }\SpecialCharTok{==} \DecValTok{0}\NormalTok{))}
\NormalTok{  num\_empty\_strings }\OtherTok{=} \FunctionTok{nrow}\NormalTok{(empty\_strings)}
  \ControlFlowTok{if}\NormalTok{(num\_empty\_strings) \{}
    \FunctionTok{cat}\NormalTok{(df\_name,}\StringTok{": "}\NormalTok{,num\_empty\_strings,}\StringTok{" empty strings"}\NormalTok{,}\StringTok{"}\SpecialCharTok{\textbackslash{}n}\StringTok{"}\NormalTok{,}\AttributeTok{sep=}\StringTok{""}\NormalTok{)}
\NormalTok{  \}}
\NormalTok{\}}
\end{Highlighting}
\end{Shaded}

\begin{verbatim}
## bodycomp_src_logs_wide: 65 NULLs
\end{verbatim}

\begin{Shaded}
\begin{Highlighting}[]
\FunctionTok{cat}\NormalTok{(}\StringTok{"Checking for NULL/empty values done.}\SpecialCharTok{\textbackslash{}n}\StringTok{"}\NormalTok{, }\AttributeTok{sep=}\StringTok{""}\NormalTok{)}
\end{Highlighting}
\end{Shaded}

\begin{verbatim}
## Checking for NULL/empty values done.
\end{verbatim}

\begin{Shaded}
\begin{Highlighting}[]
\FunctionTok{rm}\NormalTok{(df, df\_name, num\_nulls, empty\_strings, num\_empty\_strings, df\_name)}
\end{Highlighting}
\end{Shaded}

\begin{itemize}
\tightlist
\item
  Looked for NULLs and empty strings. The ``bodycomp\_src\_logs\_wide''
  table was found to be missing the majority of the entries for the
  body-fat percentage variable: this variable has been excluded from the
  analysis.
\end{itemize}

\hypertarget{clean-and-update-variable-names}{%
\subsection{Clean and update variable
names}\label{clean-and-update-variable-names}}

\begin{Shaded}
\begin{Highlighting}[]
\CommentTok{\# Function Declarations {-}{-}{-}{-}}

\NormalTok{rename\_df\_variables }\OtherTok{\textless{}{-}} \ControlFlowTok{function}\NormalTok{(df\_name, var\_mods) \{}
  \FunctionTok{cat}\NormalTok{(}\StringTok{"Renaming variables in }\SpecialCharTok{\textbackslash{}"}\StringTok{"}\NormalTok{,df\_name,}\StringTok{"}\SpecialCharTok{\textbackslash{}"}\StringTok{...}\SpecialCharTok{\textbackslash{}n}\StringTok{"}\NormalTok{, }\AttributeTok{sep =} \StringTok{""}\NormalTok{)}
\NormalTok{  df }\OtherTok{\textless{}{-}} \FunctionTok{get}\NormalTok{(df\_name)}
  \CommentTok{\# Check each var name requiring correction against the var names in the df}
  \ControlFlowTok{for}\NormalTok{ (i }\ControlFlowTok{in} \DecValTok{1}\SpecialCharTok{:}\FunctionTok{nrow}\NormalTok{(var\_mods)) \{}
\NormalTok{    var\_old }\OtherTok{=}\NormalTok{ var\_mods}\SpecialCharTok{$}\NormalTok{var\_old[i]}
    \ControlFlowTok{if}\NormalTok{ (}\SpecialCharTok{!}\NormalTok{(var\_old }\SpecialCharTok{\%in\%} \FunctionTok{colnames}\NormalTok{(df))) \{}
      \ControlFlowTok{next}
\NormalTok{    \}}
    \CommentTok{\# If found, make sure the conversion is applicable to this or all dfs}
\NormalTok{    tbl }\OtherTok{\textless{}{-}}\NormalTok{ var\_mods}\SpecialCharTok{$}\NormalTok{tbl[i]}
    \ControlFlowTok{if}\NormalTok{ (tbl }\SpecialCharTok{!=} \StringTok{""} \SpecialCharTok{\&\&}\NormalTok{ tbl }\SpecialCharTok{!=}\NormalTok{ df\_name) \{}
      \ControlFlowTok{next}
\NormalTok{    \}}
    \CommentTok{\# Perform the conversion if all checks passed}
\NormalTok{    var\_new }\OtherTok{=}\NormalTok{ var\_mods}\SpecialCharTok{$}\NormalTok{var\_new[i]}
    \FunctionTok{cat}\NormalTok{(}\StringTok{"df: "}\NormalTok{,df\_name, }\StringTok{"}\SpecialCharTok{\textbackslash{}t}\StringTok{var\_old: "}\NormalTok{,var\_old,}\StringTok{"}\SpecialCharTok{\textbackslash{}t}\StringTok{"}\NormalTok{,}\AttributeTok{sep=}\StringTok{""}\NormalTok{)}
    \FunctionTok{cat}\NormalTok{(}\StringTok{"var\_new: "}\NormalTok{,var\_new,}\StringTok{"}\SpecialCharTok{\textbackslash{}t}\StringTok{"}\NormalTok{, }\AttributeTok{sep=}\StringTok{""}\NormalTok{)}
    \FunctionTok{cat}\NormalTok{(}\StringTok{"tbl: "}\NormalTok{,tbl,}\StringTok{"    "}\NormalTok{, }\AttributeTok{sep=}\StringTok{""}\NormalTok{)}
    \FunctionTok{cat}\NormalTok{(}\StringTok{"Replacing... "}\NormalTok{, }\AttributeTok{sep =} \StringTok{""}\NormalTok{)}
\NormalTok{    df }\OtherTok{\textless{}{-}}\NormalTok{ df }\SpecialCharTok{\%\textgreater{}\%} \FunctionTok{rename}\NormalTok{(}\SpecialCharTok{!!}\AttributeTok{var\_new :=} \SpecialCharTok{!!}\NormalTok{var\_old)}
    \FunctionTok{cat}\NormalTok{(}\StringTok{"Done.}\SpecialCharTok{\textbackslash{}n}\StringTok{"}\NormalTok{, }\AttributeTok{sep =} \StringTok{""}\NormalTok{)}
\NormalTok{  \}}
  \FunctionTok{rm}\NormalTok{(i)}
  \FunctionTok{cat}\NormalTok{(}\StringTok{"Renaming variables in }\SpecialCharTok{\textbackslash{}"}\StringTok{"}\NormalTok{,df\_name,}\StringTok{"}\SpecialCharTok{\textbackslash{}"}\StringTok{ complete.}\SpecialCharTok{\textbackslash{}n}\StringTok{"}\NormalTok{, }\AttributeTok{sep =} \StringTok{""}\NormalTok{)}
  \FunctionTok{return}\NormalTok{(df)}
\NormalTok{\}}

\CommentTok{\# Global Variable Declarations {-}{-}{-}{-}}

\NormalTok{var\_mods }\OtherTok{\textless{}{-}} \FunctionTok{data.frame}\NormalTok{(}
  \AttributeTok{var\_old =} \FunctionTok{character}\NormalTok{(}\DecValTok{0}\NormalTok{),}
  \AttributeTok{var\_new  =} \FunctionTok{character}\NormalTok{(}\DecValTok{0}\NormalTok{),}
  \AttributeTok{type\_new =} \FunctionTok{character}\NormalTok{(}\DecValTok{0}\NormalTok{),}
  \AttributeTok{tbl =} \FunctionTok{character}\NormalTok{(}\DecValTok{0}\NormalTok{)}
\NormalTok{)}

\CommentTok{\# }\AlertTok{WARNING}\CommentTok{: Ensure table{-}specific modifications (tbl != "") are positioned above non{-}specific modifications with matching var\_old/var\_new values: only the first modification in the list will be applied to matching variables.}
\CommentTok{\# }\AlertTok{TODO}\CommentTok{: Eliminate this issue by modifying code to warn/handle conflicting rows}

\NormalTok{var\_mods }\OtherTok{\textless{}{-}}\NormalTok{ var\_mods }\SpecialCharTok{\%\textgreater{}\%}
  \FunctionTok{rbind}\NormalTok{(.,}\FunctionTok{data.frame}\NormalTok{(}\AttributeTok{var\_old=}\StringTok{"date"}\NormalTok{,                       }\AttributeTok{var\_new=}\StringTok{"bodycomp\_datetime"}\NormalTok{,          }\AttributeTok{type\_new=}\StringTok{"POSIXct"}\NormalTok{, }\AttributeTok{tbl=}\StringTok{"bodycomp\_src\_logs\_wide"}\NormalTok{)) }\SpecialCharTok{\%\textgreater{}\%}
  \FunctionTok{rbind}\NormalTok{(.,}\FunctionTok{data.frame}\NormalTok{(}\AttributeTok{var\_old=}\StringTok{"time"}\NormalTok{,                       }\AttributeTok{var\_new=}\StringTok{"heart\_rate\_second"}\NormalTok{,          }\AttributeTok{type\_new=}\StringTok{"POSIXct"}\NormalTok{, }\AttributeTok{tbl=}\StringTok{"heartrate\_src\_seconds\_tall"}\NormalTok{)) }\SpecialCharTok{\%\textgreater{}\%}
  \FunctionTok{rbind}\NormalTok{(.,}\FunctionTok{data.frame}\NormalTok{(}\AttributeTok{var\_old=}\StringTok{"value"}\NormalTok{,                      }\AttributeTok{var\_new=}\StringTok{"heart\_rate"}\NormalTok{,                 }\AttributeTok{type\_new=}\StringTok{""}\NormalTok{,        }\AttributeTok{tbl=}\StringTok{"heartrate\_src\_seconds\_tall"}\NormalTok{)) }\SpecialCharTok{\%\textgreater{}\%}
  \FunctionTok{rbind}\NormalTok{(.,}\FunctionTok{data.frame}\NormalTok{(}\AttributeTok{var\_old=}\StringTok{"date"}\NormalTok{,                       }\AttributeTok{var\_new=}\StringTok{"sleep\_minute"}\NormalTok{,               }\AttributeTok{type\_new=}\StringTok{"POSIXct"}\NormalTok{, }\AttributeTok{tbl=}\StringTok{"sleep\_src\_mins\_tall"}\NormalTok{)) }\SpecialCharTok{\%\textgreater{}\%}
  \FunctionTok{rbind}\NormalTok{(.,}\FunctionTok{data.frame}\NormalTok{(}\AttributeTok{var\_old=}\StringTok{"value"}\NormalTok{,                      }\AttributeTok{var\_new=}\StringTok{"sleep\_rank"}\NormalTok{,                 }\AttributeTok{type\_new=}\StringTok{""}\NormalTok{,        }\AttributeTok{tbl=}\StringTok{"sleep\_src\_mins\_tall"}\NormalTok{)) }\SpecialCharTok{\%\textgreater{}\%}
  \FunctionTok{rbind}\NormalTok{(.,}\FunctionTok{data.frame}\NormalTok{(}\AttributeTok{var\_old=}\StringTok{""}\NormalTok{,                           }\AttributeTok{var\_new=}\StringTok{"activity\_hour"}\NormalTok{,              }\AttributeTok{type\_new=}\StringTok{"POSIXct"}\NormalTok{, }\AttributeTok{tbl=}\StringTok{""}\NormalTok{)) }\SpecialCharTok{\%\textgreater{}\%}
  \FunctionTok{rbind}\NormalTok{(.,}\FunctionTok{data.frame}\NormalTok{(}\AttributeTok{var\_old=}\StringTok{""}\NormalTok{,                           }\AttributeTok{var\_new=}\StringTok{"activity\_minute"}\NormalTok{,            }\AttributeTok{type\_new=}\StringTok{"POSIXct"}\NormalTok{, }\AttributeTok{tbl=}\StringTok{""}\NormalTok{)) }\SpecialCharTok{\%\textgreater{}\%}
  \FunctionTok{rbind}\NormalTok{(.,}\FunctionTok{data.frame}\NormalTok{(}\AttributeTok{var\_old=}\StringTok{""}\NormalTok{,                           }\AttributeTok{var\_new=}\StringTok{"sleep\_day"}\NormalTok{,                  }\AttributeTok{type\_new=}\StringTok{"POSIXct"}\NormalTok{, }\AttributeTok{tbl=}\StringTok{""}\NormalTok{)) }\SpecialCharTok{\%\textgreater{}\%}
  \FunctionTok{rbind}\NormalTok{(.,}\FunctionTok{data.frame}\NormalTok{(}\AttributeTok{var\_old=}\StringTok{""}\NormalTok{,                           }\AttributeTok{var\_new=}\StringTok{"id"}\NormalTok{,                         }\AttributeTok{type\_new=}\StringTok{"character"}\NormalTok{, }\AttributeTok{tbl=}\StringTok{""}\NormalTok{)) }\SpecialCharTok{\%\textgreater{}\%}
  \FunctionTok{rbind}\NormalTok{(.,}\FunctionTok{data.frame}\NormalTok{(}\AttributeTok{var\_old=}\StringTok{"log\_id"}\NormalTok{,                     }\AttributeTok{var\_new=}\StringTok{"sleep\_log\_id"}\NormalTok{,               }\AttributeTok{type\_new=}\StringTok{"character"}\NormalTok{, }\AttributeTok{tbl=}\StringTok{"sleep\_src\_mins\_tall"}\NormalTok{)) }\SpecialCharTok{\%\textgreater{}\%}
  \FunctionTok{rbind}\NormalTok{(.,}\FunctionTok{data.frame}\NormalTok{(}\AttributeTok{var\_old=}\StringTok{"log\_id"}\NormalTok{,                     }\AttributeTok{var\_new=}\StringTok{"bodycomp\_log\_id"}\NormalTok{,            }\AttributeTok{type\_new=}\StringTok{"character"}\NormalTok{, }\AttributeTok{tbl=}\StringTok{"bodycomp\_src\_logs\_wide"}\NormalTok{)) }\SpecialCharTok{\%\textgreater{}\%}
  \FunctionTok{rbind}\NormalTok{(.,}\FunctionTok{data.frame}\NormalTok{(}\AttributeTok{var\_old=}\StringTok{"activity\_date"}\NormalTok{,              }\AttributeTok{var\_new=}\StringTok{"activity\_day"}\NormalTok{,               }\AttributeTok{type\_new=}\StringTok{"Date"}\NormalTok{,   }\AttributeTok{tbl=}\StringTok{""}\NormalTok{)) }\SpecialCharTok{\%\textgreater{}\%}
  \FunctionTok{rbind}\NormalTok{(.,}\FunctionTok{data.frame}\NormalTok{(}\AttributeTok{var\_old=}\StringTok{"fairly\_active\_distance"}\NormalTok{,     }\AttributeTok{var\_new=}\StringTok{"distance\_fairly\_active"}\NormalTok{,     }\AttributeTok{type\_new=}\StringTok{""}\NormalTok{,        }\AttributeTok{tbl=}\StringTok{""}\NormalTok{)) }\SpecialCharTok{\%\textgreater{}\%}
  \FunctionTok{rbind}\NormalTok{(.,}\FunctionTok{data.frame}\NormalTok{(}\AttributeTok{var\_old=}\StringTok{"fairly\_active\_minutes"}\NormalTok{,      }\AttributeTok{var\_new=}\StringTok{"minutes\_fairly\_active"}\NormalTok{,      }\AttributeTok{type\_new=}\StringTok{""}\NormalTok{,        }\AttributeTok{tbl=}\StringTok{""}\NormalTok{)) }\SpecialCharTok{\%\textgreater{}\%}
  \FunctionTok{rbind}\NormalTok{(.,}\FunctionTok{data.frame}\NormalTok{(}\AttributeTok{var\_old=}\StringTok{"light\_active\_distance"}\NormalTok{,      }\AttributeTok{var\_new=}\StringTok{"distance\_lightly\_active"}\NormalTok{,    }\AttributeTok{type\_new=}\StringTok{""}\NormalTok{,        }\AttributeTok{tbl=}\StringTok{""}\NormalTok{)) }\SpecialCharTok{\%\textgreater{}\%}
  \FunctionTok{rbind}\NormalTok{(.,}\FunctionTok{data.frame}\NormalTok{(}\AttributeTok{var\_old=}\StringTok{"light\_active\_minutes"}\NormalTok{,       }\AttributeTok{var\_new=}\StringTok{"minutes\_lightly\_active"}\NormalTok{,     }\AttributeTok{type\_new=}\StringTok{""}\NormalTok{,        }\AttributeTok{tbl=}\StringTok{""}\NormalTok{)) }\SpecialCharTok{\%\textgreater{}\%}
  \FunctionTok{rbind}\NormalTok{(.,}\FunctionTok{data.frame}\NormalTok{(}\AttributeTok{var\_old=}\StringTok{"lightly\_active\_distance"}\NormalTok{,    }\AttributeTok{var\_new=}\StringTok{"distance\_lightly\_active"}\NormalTok{,    }\AttributeTok{type\_new=}\StringTok{""}\NormalTok{,        }\AttributeTok{tbl=}\StringTok{""}\NormalTok{)) }\SpecialCharTok{\%\textgreater{}\%}
  \FunctionTok{rbind}\NormalTok{(.,}\FunctionTok{data.frame}\NormalTok{(}\AttributeTok{var\_old=}\StringTok{"lightly\_active\_minutes"}\NormalTok{,     }\AttributeTok{var\_new=}\StringTok{"minutes\_lightly\_active"}\NormalTok{,     }\AttributeTok{type\_new=}\StringTok{""}\NormalTok{,        }\AttributeTok{tbl=}\StringTok{""}\NormalTok{)) }\SpecialCharTok{\%\textgreater{}\%}
  \FunctionTok{rbind}\NormalTok{(.,}\FunctionTok{data.frame}\NormalTok{(}\AttributeTok{var\_old=}\StringTok{"logged\_activities\_distance"}\NormalTok{, }\AttributeTok{var\_new=}\StringTok{"distance\_logged\_activities"}\NormalTok{, }\AttributeTok{type\_new=}\StringTok{""}\NormalTok{,        }\AttributeTok{tbl=}\StringTok{""}\NormalTok{)) }\SpecialCharTok{\%\textgreater{}\%}
  \FunctionTok{rbind}\NormalTok{(.,}\FunctionTok{data.frame}\NormalTok{(}\AttributeTok{var\_old=}\StringTok{"me\_ts"}\NormalTok{,                      }\AttributeTok{var\_new=}\StringTok{"mets"}\NormalTok{,                       }\AttributeTok{type\_new=}\StringTok{""}\NormalTok{,        }\AttributeTok{tbl=}\StringTok{""}\NormalTok{)) }\SpecialCharTok{\%\textgreater{}\%}
  \FunctionTok{rbind}\NormalTok{(.,}\FunctionTok{data.frame}\NormalTok{(}\AttributeTok{var\_old=}\StringTok{"moderately\_active\_distance"}\NormalTok{, }\AttributeTok{var\_new=}\StringTok{"distance\_moderately\_active"}\NormalTok{, }\AttributeTok{type\_new=}\StringTok{""}\NormalTok{,        }\AttributeTok{tbl=}\StringTok{""}\NormalTok{)) }\SpecialCharTok{\%\textgreater{}\%}
  \FunctionTok{rbind}\NormalTok{(.,}\FunctionTok{data.frame}\NormalTok{(}\AttributeTok{var\_old=}\StringTok{"sedentary\_active\_distance"}\NormalTok{,  }\AttributeTok{var\_new=}\StringTok{"distance\_sedentary"}\NormalTok{,         }\AttributeTok{type\_new=}\StringTok{""}\NormalTok{,        }\AttributeTok{tbl=}\StringTok{""}\NormalTok{)) }\SpecialCharTok{\%\textgreater{}\%}
  \FunctionTok{rbind}\NormalTok{(.,}\FunctionTok{data.frame}\NormalTok{(}\AttributeTok{var\_old=}\StringTok{"sedentary\_distance"}\NormalTok{,         }\AttributeTok{var\_new=}\StringTok{"distance\_sedentary"}\NormalTok{,         }\AttributeTok{type\_new=}\StringTok{""}\NormalTok{,        }\AttributeTok{tbl=}\StringTok{""}\NormalTok{)) }\SpecialCharTok{\%\textgreater{}\%}
  \FunctionTok{rbind}\NormalTok{(.,}\FunctionTok{data.frame}\NormalTok{(}\AttributeTok{var\_old=}\StringTok{"sedentary\_active\_minutes"}\NormalTok{,   }\AttributeTok{var\_new=}\StringTok{"minutes\_sedentary"}\NormalTok{,          }\AttributeTok{type\_new=}\StringTok{""}\NormalTok{,        }\AttributeTok{tbl=}\StringTok{""}\NormalTok{)) }\SpecialCharTok{\%\textgreater{}\%}
  \FunctionTok{rbind}\NormalTok{(.,}\FunctionTok{data.frame}\NormalTok{(}\AttributeTok{var\_old=}\StringTok{"sedentary\_minutes"}\NormalTok{,          }\AttributeTok{var\_new=}\StringTok{"minutes\_sedentary"}\NormalTok{,          }\AttributeTok{type\_new=}\StringTok{""}\NormalTok{,        }\AttributeTok{tbl=}\StringTok{""}\NormalTok{)) }\SpecialCharTok{\%\textgreater{}\%}
  \FunctionTok{rbind}\NormalTok{(.,}\FunctionTok{data.frame}\NormalTok{(}\AttributeTok{var\_old=}\StringTok{"step\_total"}\NormalTok{,                 }\AttributeTok{var\_new=}\StringTok{"steps\_total"}\NormalTok{,                }\AttributeTok{type\_new=}\StringTok{""}\NormalTok{,        }\AttributeTok{tbl=}\StringTok{""}\NormalTok{)) }\SpecialCharTok{\%\textgreater{}\%}
  \FunctionTok{rbind}\NormalTok{(.,}\FunctionTok{data.frame}\NormalTok{(}\AttributeTok{var\_old=}\StringTok{"total\_distance"}\NormalTok{,             }\AttributeTok{var\_new=}\StringTok{"distance\_total"}\NormalTok{,             }\AttributeTok{type\_new=}\StringTok{""}\NormalTok{,        }\AttributeTok{tbl=}\StringTok{""}\NormalTok{)) }\SpecialCharTok{\%\textgreater{}\%}
  \FunctionTok{rbind}\NormalTok{(.,}\FunctionTok{data.frame}\NormalTok{(}\AttributeTok{var\_old=}\StringTok{"total\_intensity"}\NormalTok{,            }\AttributeTok{var\_new=}\StringTok{"intensity\_total"}\NormalTok{,            }\AttributeTok{type\_new=}\StringTok{""}\NormalTok{,        }\AttributeTok{tbl=}\StringTok{""}\NormalTok{)) }\SpecialCharTok{\%\textgreater{}\%}
  \FunctionTok{rbind}\NormalTok{(.,}\FunctionTok{data.frame}\NormalTok{(}\AttributeTok{var\_old=}\StringTok{"total\_minutes\_asleep"}\NormalTok{,       }\AttributeTok{var\_new=}\StringTok{"minutes\_asleep\_total"}\NormalTok{,       }\AttributeTok{type\_new=}\StringTok{""}\NormalTok{,        }\AttributeTok{tbl=}\StringTok{""}\NormalTok{)) }\SpecialCharTok{\%\textgreater{}\%}
  \FunctionTok{rbind}\NormalTok{(.,}\FunctionTok{data.frame}\NormalTok{(}\AttributeTok{var\_old=}\StringTok{"total\_sleep\_records"}\NormalTok{,        }\AttributeTok{var\_new=}\StringTok{"sleep\_records\_total"}\NormalTok{,        }\AttributeTok{type\_new=}\StringTok{""}\NormalTok{,        }\AttributeTok{tbl=}\StringTok{""}\NormalTok{)) }\SpecialCharTok{\%\textgreater{}\%}
  \FunctionTok{rbind}\NormalTok{(.,}\FunctionTok{data.frame}\NormalTok{(}\AttributeTok{var\_old=}\StringTok{"total\_steps"}\NormalTok{,                }\AttributeTok{var\_new=}\StringTok{"steps\_total"}\NormalTok{,                }\AttributeTok{type\_new=}\StringTok{""}\NormalTok{,        }\AttributeTok{tbl=}\StringTok{""}\NormalTok{)) }\SpecialCharTok{\%\textgreater{}\%}
  \FunctionTok{rbind}\NormalTok{(.,}\FunctionTok{data.frame}\NormalTok{(}\AttributeTok{var\_old=}\StringTok{"total\_time\_in\_bed"}\NormalTok{,          }\AttributeTok{var\_new=}\StringTok{"minutes\_in\_bed\_total"}\NormalTok{,       }\AttributeTok{type\_new=}\StringTok{""}\NormalTok{,        }\AttributeTok{tbl=}\StringTok{""}\NormalTok{)) }\SpecialCharTok{\%\textgreater{}\%}
  \FunctionTok{rbind}\NormalTok{(.,}\FunctionTok{data.frame}\NormalTok{(}\AttributeTok{var\_old=}\StringTok{"tracker\_distance"}\NormalTok{,           }\AttributeTok{var\_new=}\StringTok{"distance\_tracker"}\NormalTok{,           }\AttributeTok{type\_new=}\StringTok{""}\NormalTok{,        }\AttributeTok{tbl=}\StringTok{""}\NormalTok{)) }\SpecialCharTok{\%\textgreater{}\%}
  \FunctionTok{rbind}\NormalTok{(.,}\FunctionTok{data.frame}\NormalTok{(}\AttributeTok{var\_old=}\StringTok{"very\_active\_distance"}\NormalTok{,       }\AttributeTok{var\_new=}\StringTok{"distance\_very\_active"}\NormalTok{,       }\AttributeTok{type\_new=}\StringTok{""}\NormalTok{,        }\AttributeTok{tbl=}\StringTok{""}\NormalTok{)) }\SpecialCharTok{\%\textgreater{}\%}
  \FunctionTok{rbind}\NormalTok{(.,}\FunctionTok{data.frame}\NormalTok{(}\AttributeTok{var\_old=}\StringTok{"very\_active\_minutes"}\NormalTok{,        }\AttributeTok{var\_new=}\StringTok{"minutes\_very\_active"}\NormalTok{,        }\AttributeTok{type\_new=}\StringTok{""}\NormalTok{,        }\AttributeTok{tbl=}\StringTok{""}\NormalTok{))}

\DocumentationTok{\#\# Rename Variables {-}{-}{-}{-}}

\FunctionTok{cat}\NormalTok{(}\StringTok{"Cleaning variable names...}\SpecialCharTok{\textbackslash{}n}\StringTok{"}\NormalTok{, }\AttributeTok{sep =} \StringTok{""}\NormalTok{)}
\end{Highlighting}
\end{Shaded}

\begin{verbatim}
## Cleaning variable names...
\end{verbatim}

\begin{Shaded}
\begin{Highlighting}[]
\ControlFlowTok{for}\NormalTok{(df\_name }\ControlFlowTok{in}\NormalTok{ df\_names) \{}
  \FunctionTok{cat}\NormalTok{(}\StringTok{"Cleaning "}\NormalTok{,df\_name,}\StringTok{"...}\SpecialCharTok{\textbackslash{}n}\StringTok{"}\NormalTok{, }\AttributeTok{sep =} \StringTok{""}\NormalTok{)}
  \FunctionTok{assign}\NormalTok{(df\_name, }\FunctionTok{get}\NormalTok{(df\_name) }\SpecialCharTok{\%\textgreater{}\%} \FunctionTok{clean\_names}\NormalTok{())}
\NormalTok{\}}
\end{Highlighting}
\end{Shaded}

\begin{verbatim}
## Cleaning activity_sum_days_wide...
## Cleaning bodycomp_src_logs_wide...
## Cleaning calories_src_mins_tall...
## Cleaning calories_sum_days_tall...
## Cleaning calories_sum_hours_tall...
## Cleaning calories_sum_mins_wide...
## Cleaning heartrate_src_seconds_tall...
## Cleaning intensity_src_mins_tall...
## Cleaning intensity_sum_days_wide...
## Cleaning intensity_sum_hours_wide...
## Cleaning intensity_sum_mins_wide...
## Cleaning mets_src_mins_tall...
## Cleaning sleep_src_mins_tall...
## Cleaning sleep_sum_days_wide...
## Cleaning steps_src_mins_tall...
## Cleaning steps_sum_days_tall...
## Cleaning steps_sum_hours_tall...
## Cleaning steps_sum_mins_wide...
\end{verbatim}

\begin{Shaded}
\begin{Highlighting}[]
\FunctionTok{cat}\NormalTok{(}\StringTok{"Cleaning variable names complete.}\SpecialCharTok{\textbackslash{}n}\StringTok{"}\NormalTok{, }\AttributeTok{sep =} \StringTok{""}\NormalTok{)}
\end{Highlighting}
\end{Shaded}

\begin{verbatim}
## Cleaning variable names complete.
\end{verbatim}

\begin{Shaded}
\begin{Highlighting}[]
\FunctionTok{cat}\NormalTok{(}\StringTok{"Renaming variables...}\SpecialCharTok{\textbackslash{}n}\StringTok{"}\NormalTok{, }\AttributeTok{sep =} \StringTok{""}\NormalTok{)}
\end{Highlighting}
\end{Shaded}

\begin{verbatim}
## Renaming variables...
\end{verbatim}

\begin{Shaded}
\begin{Highlighting}[]
\NormalTok{var\_mods\_rename }\OtherTok{\textless{}{-}}\NormalTok{ var\_mods }\SpecialCharTok{\%\textgreater{}\%}
  \FunctionTok{filter}\NormalTok{(var\_old }\SpecialCharTok{!=} \StringTok{""} \SpecialCharTok{\&}\NormalTok{ var\_new }\SpecialCharTok{!=} \StringTok{""}\NormalTok{)}
\ControlFlowTok{for}\NormalTok{(df\_name }\ControlFlowTok{in}\NormalTok{ df\_names) \{}
  \FunctionTok{assign}\NormalTok{(df\_name, }\FunctionTok{rename\_df\_variables}\NormalTok{(df\_name, var\_mods\_rename))}
\NormalTok{\}}
\end{Highlighting}
\end{Shaded}

\begin{verbatim}
## Renaming variables in "activity_sum_days_wide"...
## df: activity_sum_days_wide   var_old: activity_date  var_new: activity_day   tbl:     Replacing... Done.
## df: activity_sum_days_wide   var_old: fairly_active_minutes  var_new: minutes_fairly_active  tbl:     Replacing... Done.
## df: activity_sum_days_wide   var_old: light_active_distance  var_new: distance_lightly_active    tbl:     Replacing... Done.
## df: activity_sum_days_wide   var_old: lightly_active_minutes var_new: minutes_lightly_active tbl:     Replacing... Done.
## df: activity_sum_days_wide   var_old: logged_activities_distance var_new: distance_logged_activities tbl:     Replacing... Done.
## df: activity_sum_days_wide   var_old: moderately_active_distance var_new: distance_moderately_active tbl:     Replacing... Done.
## df: activity_sum_days_wide   var_old: sedentary_active_distance  var_new: distance_sedentary tbl:     Replacing... Done.
## df: activity_sum_days_wide   var_old: sedentary_minutes  var_new: minutes_sedentary  tbl:     Replacing... Done.
## df: activity_sum_days_wide   var_old: total_distance var_new: distance_total tbl:     Replacing... Done.
## df: activity_sum_days_wide   var_old: total_steps    var_new: steps_total    tbl:     Replacing... Done.
## df: activity_sum_days_wide   var_old: tracker_distance   var_new: distance_tracker   tbl:     Replacing... Done.
## df: activity_sum_days_wide   var_old: very_active_distance   var_new: distance_very_active   tbl:     Replacing... Done.
## df: activity_sum_days_wide   var_old: very_active_minutes    var_new: minutes_very_active    tbl:     Replacing... Done.
## Renaming variables in "activity_sum_days_wide" complete.
## Renaming variables in "bodycomp_src_logs_wide"...
## df: bodycomp_src_logs_wide   var_old: date   var_new: bodycomp_datetime  tbl: bodycomp_src_logs_wide    Replacing... Done.
## df: bodycomp_src_logs_wide   var_old: log_id var_new: bodycomp_log_id    tbl: bodycomp_src_logs_wide    Replacing... Done.
## Renaming variables in "bodycomp_src_logs_wide" complete.
## Renaming variables in "calories_src_mins_tall"...
## Renaming variables in "calories_src_mins_tall" complete.
## Renaming variables in "calories_sum_days_tall"...
## Renaming variables in "calories_sum_days_tall" complete.
## Renaming variables in "calories_sum_hours_tall"...
## Renaming variables in "calories_sum_hours_tall" complete.
## Renaming variables in "calories_sum_mins_wide"...
## Renaming variables in "calories_sum_mins_wide" complete.
## Renaming variables in "heartrate_src_seconds_tall"...
## df: heartrate_src_seconds_tall   var_old: time   var_new: heart_rate_second  tbl: heartrate_src_seconds_tall    Replacing... Done.
## df: heartrate_src_seconds_tall   var_old: value  var_new: heart_rate tbl: heartrate_src_seconds_tall    Replacing... Done.
## Renaming variables in "heartrate_src_seconds_tall" complete.
## Renaming variables in "intensity_src_mins_tall"...
## Renaming variables in "intensity_src_mins_tall" complete.
## Renaming variables in "intensity_sum_days_wide"...
## df: intensity_sum_days_wide  var_old: fairly_active_minutes  var_new: minutes_fairly_active  tbl:     Replacing... Done.
## df: intensity_sum_days_wide  var_old: light_active_distance  var_new: distance_lightly_active    tbl:     Replacing... Done.
## df: intensity_sum_days_wide  var_old: lightly_active_minutes var_new: minutes_lightly_active tbl:     Replacing... Done.
## df: intensity_sum_days_wide  var_old: moderately_active_distance var_new: distance_moderately_active tbl:     Replacing... Done.
## df: intensity_sum_days_wide  var_old: sedentary_active_distance  var_new: distance_sedentary tbl:     Replacing... Done.
## df: intensity_sum_days_wide  var_old: sedentary_minutes  var_new: minutes_sedentary  tbl:     Replacing... Done.
## df: intensity_sum_days_wide  var_old: very_active_distance   var_new: distance_very_active   tbl:     Replacing... Done.
## df: intensity_sum_days_wide  var_old: very_active_minutes    var_new: minutes_very_active    tbl:     Replacing... Done.
## Renaming variables in "intensity_sum_days_wide" complete.
## Renaming variables in "intensity_sum_hours_wide"...
## df: intensity_sum_hours_wide var_old: total_intensity    var_new: intensity_total    tbl:     Replacing... Done.
## Renaming variables in "intensity_sum_hours_wide" complete.
## Renaming variables in "intensity_sum_mins_wide"...
## Renaming variables in "intensity_sum_mins_wide" complete.
## Renaming variables in "mets_src_mins_tall"...
## df: mets_src_mins_tall   var_old: me_ts  var_new: mets   tbl:     Replacing... Done.
## Renaming variables in "mets_src_mins_tall" complete.
## Renaming variables in "sleep_src_mins_tall"...
## df: sleep_src_mins_tall  var_old: date   var_new: sleep_minute   tbl: sleep_src_mins_tall    Replacing... Done.
## df: sleep_src_mins_tall  var_old: value  var_new: sleep_rank tbl: sleep_src_mins_tall    Replacing... Done.
## df: sleep_src_mins_tall  var_old: log_id var_new: sleep_log_id   tbl: sleep_src_mins_tall    Replacing... Done.
## Renaming variables in "sleep_src_mins_tall" complete.
## Renaming variables in "sleep_sum_days_wide"...
## df: sleep_sum_days_wide  var_old: total_minutes_asleep   var_new: minutes_asleep_total   tbl:     Replacing... Done.
## df: sleep_sum_days_wide  var_old: total_sleep_records    var_new: sleep_records_total    tbl:     Replacing... Done.
## df: sleep_sum_days_wide  var_old: total_time_in_bed  var_new: minutes_in_bed_total   tbl:     Replacing... Done.
## Renaming variables in "sleep_sum_days_wide" complete.
## Renaming variables in "steps_src_mins_tall"...
## Renaming variables in "steps_src_mins_tall" complete.
## Renaming variables in "steps_sum_days_tall"...
## df: steps_sum_days_tall  var_old: step_total var_new: steps_total    tbl:     Replacing... Done.
## Renaming variables in "steps_sum_days_tall" complete.
## Renaming variables in "steps_sum_hours_tall"...
## df: steps_sum_hours_tall var_old: step_total var_new: steps_total    tbl:     Replacing... Done.
## Renaming variables in "steps_sum_hours_tall" complete.
## Renaming variables in "steps_sum_mins_wide"...
## Renaming variables in "steps_sum_mins_wide" complete.
\end{verbatim}

\begin{Shaded}
\begin{Highlighting}[]
\FunctionTok{rm}\NormalTok{(df\_name)}
\FunctionTok{cat}\NormalTok{(}\StringTok{"Renaming variables complete.}\SpecialCharTok{\textbackslash{}n}\StringTok{"}\NormalTok{, }\AttributeTok{sep =} \StringTok{""}\NormalTok{)}
\end{Highlighting}
\end{Shaded}

\begin{verbatim}
## Renaming variables complete.
\end{verbatim}

\hypertarget{remove-duplicate-data}{%
\subsection{Remove duplicate data}\label{remove-duplicate-data}}

All dataframes were checked for duplicate rows using the anyDuplicated()
function, and duplicate rows were removed using the distinct() function.
The function was tested using a prototype version that generated
dataframes of all of the apparent duplicate rows: these were verified
manually before the function was allowed to modify the actual
dataframes. The logic was verified again by re-running it after the
initial removal: all dataframes reported zero duplicates on the second
run, confirming the success of the first pass.

\begin{Shaded}
\begin{Highlighting}[]
\FunctionTok{cat}\NormalTok{(}\StringTok{"Checking for duplicate values...}\SpecialCharTok{\textbackslash{}n}\StringTok{"}\NormalTok{,}\AttributeTok{sep=}\StringTok{""}\NormalTok{)}
\end{Highlighting}
\end{Shaded}

\begin{verbatim}
## Checking for duplicate values...
\end{verbatim}

\begin{Shaded}
\begin{Highlighting}[]
\ControlFlowTok{for}\NormalTok{ (df\_name }\ControlFlowTok{in}\NormalTok{ df\_names) \{}
  \FunctionTok{cat}\NormalTok{(}\StringTok{"Checking "}\NormalTok{,df\_name,}\StringTok{" for duplicates... "}\NormalTok{,}\AttributeTok{sep=}\StringTok{""}\NormalTok{)}
\NormalTok{  df }\OtherTok{\textless{}{-}} \FunctionTok{get}\NormalTok{(df\_name)}
  \ControlFlowTok{if}\NormalTok{ (}\SpecialCharTok{!}\FunctionTok{anyDuplicated}\NormalTok{(df)) \{}
    \FunctionTok{cat}\NormalTok{(}\StringTok{"0 duplicates removed. Done.}\SpecialCharTok{\textbackslash{}n}\StringTok{"}\NormalTok{,}\AttributeTok{sep=}\StringTok{""}\NormalTok{)}
\NormalTok{  \} }\ControlFlowTok{else}\NormalTok{ \{}
\NormalTok{    nrow\_before }\OtherTok{\textless{}{-}} \FunctionTok{nrow}\NormalTok{(df)}
\NormalTok{    df }\OtherTok{\textless{}{-}} \FunctionTok{distinct}\NormalTok{(df)}
\NormalTok{    nrow\_after }\OtherTok{\textless{}{-}} \FunctionTok{nrow}\NormalTok{(df)}
    \FunctionTok{cat}\NormalTok{(}\StringTok{"Removing "}\NormalTok{,(nrow\_before }\SpecialCharTok{{-}}\NormalTok{ nrow\_after),}\StringTok{" duplicates... "}\NormalTok{,}\AttributeTok{sep=}\StringTok{""}\NormalTok{)}
    \FunctionTok{assign}\NormalTok{(df\_name, df)}
    \FunctionTok{cat}\NormalTok{(}\StringTok{"Done.}\SpecialCharTok{\textbackslash{}n}\StringTok{"}\NormalTok{,}\AttributeTok{sep=}\StringTok{""}\NormalTok{)}
\NormalTok{  \}}
\NormalTok{\}}
\end{Highlighting}
\end{Shaded}

\begin{verbatim}
## Checking activity_sum_days_wide for duplicates... 0 duplicates removed. Done.
## Checking bodycomp_src_logs_wide for duplicates... 0 duplicates removed. Done.
## Checking calories_src_mins_tall for duplicates... 0 duplicates removed. Done.
## Checking calories_sum_days_tall for duplicates... 0 duplicates removed. Done.
## Checking calories_sum_hours_tall for duplicates... 0 duplicates removed. Done.
## Checking calories_sum_mins_wide for duplicates... 0 duplicates removed. Done.
## Checking heartrate_src_seconds_tall for duplicates... 0 duplicates removed. Done.
## Checking intensity_src_mins_tall for duplicates... 0 duplicates removed. Done.
## Checking intensity_sum_days_wide for duplicates... 0 duplicates removed. Done.
## Checking intensity_sum_hours_wide for duplicates... 0 duplicates removed. Done.
## Checking intensity_sum_mins_wide for duplicates... 0 duplicates removed. Done.
## Checking mets_src_mins_tall for duplicates... 0 duplicates removed. Done.
## Checking sleep_src_mins_tall for duplicates... Removing 543 duplicates... Done.
## Checking sleep_sum_days_wide for duplicates... Removing 3 duplicates... Done.
## Checking steps_src_mins_tall for duplicates... 0 duplicates removed. Done.
## Checking steps_sum_days_tall for duplicates... 0 duplicates removed. Done.
## Checking steps_sum_hours_tall for duplicates... 0 duplicates removed. Done.
## Checking steps_sum_mins_wide for duplicates... 0 duplicates removed. Done.
\end{verbatim}

\begin{Shaded}
\begin{Highlighting}[]
\FunctionTok{rm}\NormalTok{(df, df\_name, nrow\_before, nrow\_after)}
\FunctionTok{cat}\NormalTok{(}\StringTok{"Checking for duplicate values complete.}\SpecialCharTok{\textbackslash{}n}\StringTok{"}\NormalTok{,}\AttributeTok{sep=}\StringTok{""}\NormalTok{)}
\end{Highlighting}
\end{Shaded}

\begin{verbatim}
## Checking for duplicate values complete.
\end{verbatim}

\hypertarget{recast-mismatched-variable-data-types}{%
\subsection{Recast mismatched variable data
types}\label{recast-mismatched-variable-data-types}}

\begin{longtable}[]{@{}
  >{\raggedright\arraybackslash}p{(\columnwidth - 6\tabcolsep) * \real{0.1518}}
  >{\raggedright\arraybackslash}p{(\columnwidth - 6\tabcolsep) * \real{0.1339}}
  >{\raggedright\arraybackslash}p{(\columnwidth - 6\tabcolsep) * \real{0.1250}}
  >{\raggedright\arraybackslash}p{(\columnwidth - 6\tabcolsep) * \real{0.5893}}@{}}
\toprule\noalign{}
\begin{minipage}[b]{\linewidth}\raggedright
Variable
\end{minipage} & \begin{minipage}[b]{\linewidth}\raggedright
Original Type
\end{minipage} & \begin{minipage}[b]{\linewidth}\raggedright
Updated Type
\end{minipage} & \begin{minipage}[b]{\linewidth}\raggedright
Reason
\end{minipage} \\
\midrule\noalign{}
\endhead
\bottomrule\noalign{}
\endlastfoot
activity\_day & chr & datetime & Cannot perform datetime operations on
chr variables \\
activity\_hour & chr & datetime & Cannot perform datetime operations on
chr variables \\
activity\_minute & chr & datetime & Cannot perform datetime operations
on chr variables \\
date & chr & datetime & Cannot perform datetime operations on chr
variables \\
id & num & chr & Disable numerical operations (IDs are considered a UID
string) \\
log\_id & num & chr & Disable numerical operations (IDs are considered a
UID string) \\
time & chr & datetime & Cannot perform datetime operations on chr
variables \\
sleep\_rank & num & factor 1:3 & Disable numerical operations (value is
a ranking, not an amount) \\
intensity & num & factor 0:3 & Disable numerical operations (value is a
ranking, not an amount) \\
\end{longtable}

\begin{Shaded}
\begin{Highlighting}[]
\CommentTok{\# Global Variable Declarations {-}{-}{-}{-}}

\NormalTok{var\_mods\_recast }\OtherTok{\textless{}{-}}\NormalTok{ var\_mods }\SpecialCharTok{\%\textgreater{}\%}
  \FunctionTok{filter}\NormalTok{(type\_new }\SpecialCharTok{!=} \StringTok{""}\NormalTok{)}

\CommentTok{\# Function Declarations {-}{-}{-}{-}}

\NormalTok{are\_identical\_lists }\OtherTok{\textless{}{-}} \ControlFlowTok{function}\NormalTok{(list1, list2) \{}
  \ControlFlowTok{if}\NormalTok{ (}\FunctionTok{length}\NormalTok{(list1) }\SpecialCharTok{!=} \FunctionTok{length}\NormalTok{(list2)) \{}
    \FunctionTok{return}\NormalTok{(}\ConstantTok{FALSE}\NormalTok{)}
\NormalTok{  \}}
  \ControlFlowTok{for}\NormalTok{ (i }\ControlFlowTok{in} \FunctionTok{seq\_along}\NormalTok{(list1)) \{}
    \ControlFlowTok{if}\NormalTok{(}\SpecialCharTok{!}\FunctionTok{identical}\NormalTok{(list1[[i]], list2[[i]])) \{}
      \FunctionTok{cat}\NormalTok{(}\StringTok{"Non{-}identical lists at list1["}\NormalTok{,i,}\StringTok{"]. Exiting.}\SpecialCharTok{\textbackslash{}n}\StringTok{"}\NormalTok{, }\AttributeTok{sep =} \StringTok{""}\NormalTok{)}
      \FunctionTok{print}\NormalTok{(list1[[i]])}
      \FunctionTok{print}\NormalTok{(list2[[i]])}
      \FunctionTok{rm}\NormalTok{(i)}
      \FunctionTok{return}\NormalTok{(}\ConstantTok{FALSE}\NormalTok{)}
\NormalTok{    \}}
\NormalTok{  \}}
  \FunctionTok{rm}\NormalTok{(i)}
  \FunctionTok{return}\NormalTok{(}\ConstantTok{TRUE}\NormalTok{)}
\NormalTok{\}}

\NormalTok{get\_df\_var\_types }\OtherTok{\textless{}{-}} \ControlFlowTok{function}\NormalTok{(df\_name) \{}
  \FunctionTok{cat}\NormalTok{(}\StringTok{"Getting current variable types for "}\NormalTok{, df\_name, }\StringTok{"... "}\NormalTok{, }\AttributeTok{sep =} \StringTok{""}\NormalTok{)}
\NormalTok{  df }\OtherTok{\textless{}{-}} \FunctionTok{get}\NormalTok{(df\_name)}
\NormalTok{  var\_types }\OtherTok{\textless{}{-}} \FunctionTok{data.frame}\NormalTok{(}
    \AttributeTok{var =} \FunctionTok{names}\NormalTok{(df),}
    \AttributeTok{type =} \FunctionTok{sapply}\NormalTok{(df, }\ControlFlowTok{function}\NormalTok{(col) }\FunctionTok{class}\NormalTok{(col)[}\DecValTok{1}\NormalTok{])}
\NormalTok{  )}
  \FunctionTok{cat}\NormalTok{(}\StringTok{"Done.}\SpecialCharTok{\textbackslash{}n}\StringTok{"}\NormalTok{, }\AttributeTok{sep =} \StringTok{""}\NormalTok{)}
  \FunctionTok{return}\NormalTok{(var\_types)}
\NormalTok{\}}

\NormalTok{get\_df\_target\_var\_types }\OtherTok{\textless{}{-}} \ControlFlowTok{function}\NormalTok{(df\_name) \{}
  \FunctionTok{cat}\NormalTok{(}\StringTok{"Getting target variable types for "}\NormalTok{, df\_name, }\StringTok{"...}\SpecialCharTok{\textbackslash{}n}\StringTok{"}\NormalTok{, }\AttributeTok{sep =} \StringTok{""}\NormalTok{)}
\NormalTok{  var\_types }\OtherTok{\textless{}{-}} \FunctionTok{get\_df\_var\_types}\NormalTok{(df\_name)}
  \CommentTok{\# Iterate over rows in current var\_types}
  \ControlFlowTok{for}\NormalTok{ (var\_row }\ControlFlowTok{in} \DecValTok{1}\SpecialCharTok{:}\FunctionTok{nrow}\NormalTok{(var\_types)) \{}
    \CommentTok{\# Check if var name is in var\_mods\_recast}
\NormalTok{    var\_name }\OtherTok{\textless{}{-}}\NormalTok{ var\_types}\SpecialCharTok{$}\NormalTok{var[var\_row]}
    \ControlFlowTok{for}\NormalTok{(mods\_row }\ControlFlowTok{in} \DecValTok{1}\SpecialCharTok{:}\FunctionTok{nrow}\NormalTok{(var\_mods\_recast)) \{}
\NormalTok{      var\_name\_mods }\OtherTok{\textless{}{-}}\NormalTok{ var\_mods\_recast}\SpecialCharTok{$}\NormalTok{var\_new[mods\_row]}
      \CommentTok{\# If no, skip: if yes, replace type with target}
      \ControlFlowTok{if}\NormalTok{(var\_name }\SpecialCharTok{==}\NormalTok{ var\_name\_mods) \{}
\NormalTok{        type\_new }\OtherTok{\textless{}{-}}\NormalTok{ var\_mods\_recast}\SpecialCharTok{$}\NormalTok{type\_new[mods\_row]}
        \FunctionTok{cat}\NormalTok{(}\StringTok{"Updated }\SpecialCharTok{\textbackslash{}"}\StringTok{"}\NormalTok{, var\_name, }\StringTok{"}\SpecialCharTok{\textbackslash{}"}\StringTok{ from "}\NormalTok{, var\_types}\SpecialCharTok{$}\NormalTok{type[var\_row], }\StringTok{" to "}\NormalTok{, var\_mods\_recast}\SpecialCharTok{$}\NormalTok{type\_new[mods\_row], }\StringTok{".}\SpecialCharTok{\textbackslash{}n}\StringTok{"}\NormalTok{, }\AttributeTok{sep =} \StringTok{""}\NormalTok{)}
\NormalTok{        var\_types}\SpecialCharTok{$}\NormalTok{type[var\_row] }\OtherTok{\textless{}{-}}\NormalTok{ type\_new}
\NormalTok{      \}}
\NormalTok{    \}}
\NormalTok{  \}}
  \FunctionTok{cat}\NormalTok{(}\StringTok{"Getting target variable types for "}\NormalTok{, df\_name, }\StringTok{" done.}\SpecialCharTok{\textbackslash{}n}\StringTok{"}\NormalTok{, }\AttributeTok{sep =} \StringTok{""}\NormalTok{)}
  \FunctionTok{return}\NormalTok{(var\_types)}
\NormalTok{\}}

\NormalTok{recast\_variables }\OtherTok{\textless{}{-}} \ControlFlowTok{function}\NormalTok{(df\_name) \{}
\NormalTok{  df }\OtherTok{\textless{}{-}} \FunctionTok{get}\NormalTok{(df\_name)}
  \ControlFlowTok{for}\NormalTok{ (i }\ControlFlowTok{in} \DecValTok{1}\SpecialCharTok{:}\FunctionTok{nrow}\NormalTok{(var\_mods\_recast)) \{}
\NormalTok{    var\_new }\OtherTok{\textless{}{-}}\NormalTok{ var\_mods\_recast}\SpecialCharTok{$}\NormalTok{var\_new[i]}
\NormalTok{    type\_new }\OtherTok{\textless{}{-}}\NormalTok{ var\_mods\_recast}\SpecialCharTok{$}\NormalTok{type\_new[i]}
    \ControlFlowTok{if}\NormalTok{ (var\_new }\SpecialCharTok{\%in\%} \FunctionTok{colnames}\NormalTok{(df)) \{}
      \FunctionTok{cat}\NormalTok{(}\StringTok{"Converting "}\NormalTok{,df\_name,}\StringTok{"$"}\NormalTok{,var\_new,}\StringTok{" to "}\NormalTok{,type\_new, }\StringTok{"... "}\NormalTok{, }\AttributeTok{sep =} \StringTok{""}\NormalTok{)}
      \ControlFlowTok{if}\NormalTok{ (type\_new }\SpecialCharTok{==} \StringTok{"character"}\NormalTok{) \{}
\NormalTok{        df }\OtherTok{\textless{}{-}}\NormalTok{ df }\SpecialCharTok{\%\textgreater{}\%} \FunctionTok{mutate}\NormalTok{(}\StringTok{"\{var\_new\}"} \SpecialCharTok{:=} \FunctionTok{as.character}\NormalTok{(}\SpecialCharTok{!!}\FunctionTok{sym}\NormalTok{(var\_new)))}
\NormalTok{      \} }\ControlFlowTok{else} \ControlFlowTok{if}\NormalTok{ (type\_new }\SpecialCharTok{==} \StringTok{"Date"}\NormalTok{) \{}
\NormalTok{        df }\OtherTok{\textless{}{-}}\NormalTok{ df }\SpecialCharTok{\%\textgreater{}\%} \FunctionTok{mutate}\NormalTok{(}\StringTok{"\{var\_new\}"} \SpecialCharTok{:=} \FunctionTok{mdy}\NormalTok{(}\SpecialCharTok{!!}\FunctionTok{sym}\NormalTok{(var\_new)))}
\NormalTok{      \} }\ControlFlowTok{else} \ControlFlowTok{if}\NormalTok{ (type\_new }\SpecialCharTok{==} \StringTok{"POSIXct"}\NormalTok{) \{}
\NormalTok{        df }\OtherTok{\textless{}{-}}\NormalTok{ df }\SpecialCharTok{\%\textgreater{}\%} \FunctionTok{mutate}\NormalTok{(}\StringTok{"\{var\_new\}"} \SpecialCharTok{:=} \FunctionTok{mdy\_hms}\NormalTok{(}\SpecialCharTok{!!}\FunctionTok{sym}\NormalTok{(var\_new)))}
\NormalTok{      \} }\ControlFlowTok{else}\NormalTok{ \{}
        \FunctionTok{cat}\NormalTok{(}\StringTok{"type\_new not found: not converting."}\NormalTok{, }\AttributeTok{sep =} \StringTok{""}\NormalTok{)}
\NormalTok{      \}}
      \FunctionTok{cat}\NormalTok{(}\StringTok{"Done.}\SpecialCharTok{\textbackslash{}n}\StringTok{"}\NormalTok{, }\AttributeTok{sep =} \StringTok{""}\NormalTok{)}
\NormalTok{    \}}
\NormalTok{  \}}
  \FunctionTok{rm}\NormalTok{(i)}
  \FunctionTok{return}\NormalTok{(df)}
\NormalTok{\}}

\CommentTok{\# Recast Variables {-}{-}{-}{-}}

\FunctionTok{cat}\NormalTok{(}\StringTok{"Generating list of target column types for testing...}\SpecialCharTok{\textbackslash{}n}\StringTok{"}\NormalTok{, }\AttributeTok{sep =} \StringTok{""}\NormalTok{)}
\end{Highlighting}
\end{Shaded}

\begin{verbatim}
## Generating list of target column types for testing...
\end{verbatim}

\begin{Shaded}
\begin{Highlighting}[]
\NormalTok{df\_types\_target }\OtherTok{\textless{}{-}}\FunctionTok{lapply}\NormalTok{(df\_names, get\_df\_target\_var\_types)}
\end{Highlighting}
\end{Shaded}

\begin{verbatim}
## Getting target variable types for activity_sum_days_wide...
## Getting current variable types for activity_sum_days_wide... Done.
## Updated "id" from numeric to character.
## Updated "activity_day" from character to Date.
## Getting target variable types for activity_sum_days_wide done.
## Getting target variable types for bodycomp_src_logs_wide...
## Getting current variable types for bodycomp_src_logs_wide... Done.
## Updated "id" from numeric to character.
## Updated "bodycomp_datetime" from character to POSIXct.
## Updated "bodycomp_log_id" from numeric to character.
## Getting target variable types for bodycomp_src_logs_wide done.
## Getting target variable types for calories_src_mins_tall...
## Getting current variable types for calories_src_mins_tall... Done.
## Updated "id" from numeric to character.
## Updated "activity_minute" from character to POSIXct.
## Getting target variable types for calories_src_mins_tall done.
## Getting target variable types for calories_sum_days_tall...
## Getting current variable types for calories_sum_days_tall... Done.
## Updated "id" from numeric to character.
## Updated "activity_day" from character to Date.
## Getting target variable types for calories_sum_days_tall done.
## Getting target variable types for calories_sum_hours_tall...
## Getting current variable types for calories_sum_hours_tall... Done.
## Updated "id" from numeric to character.
## Updated "activity_hour" from character to POSIXct.
## Getting target variable types for calories_sum_hours_tall done.
## Getting target variable types for calories_sum_mins_wide...
## Getting current variable types for calories_sum_mins_wide... Done.
## Updated "id" from numeric to character.
## Updated "activity_hour" from character to POSIXct.
## Getting target variable types for calories_sum_mins_wide done.
## Getting target variable types for heartrate_src_seconds_tall...
## Getting current variable types for heartrate_src_seconds_tall... Done.
## Updated "id" from numeric to character.
## Updated "heart_rate_second" from character to POSIXct.
## Getting target variable types for heartrate_src_seconds_tall done.
## Getting target variable types for intensity_src_mins_tall...
## Getting current variable types for intensity_src_mins_tall... Done.
## Updated "id" from numeric to character.
## Updated "activity_minute" from character to POSIXct.
## Getting target variable types for intensity_src_mins_tall done.
## Getting target variable types for intensity_sum_days_wide...
## Getting current variable types for intensity_sum_days_wide... Done.
## Updated "id" from numeric to character.
## Updated "activity_day" from character to Date.
## Getting target variable types for intensity_sum_days_wide done.
## Getting target variable types for intensity_sum_hours_wide...
## Getting current variable types for intensity_sum_hours_wide... Done.
## Updated "id" from numeric to character.
## Updated "activity_hour" from character to POSIXct.
## Getting target variable types for intensity_sum_hours_wide done.
## Getting target variable types for intensity_sum_mins_wide...
## Getting current variable types for intensity_sum_mins_wide... Done.
## Updated "id" from numeric to character.
## Updated "activity_hour" from character to POSIXct.
## Getting target variable types for intensity_sum_mins_wide done.
## Getting target variable types for mets_src_mins_tall...
## Getting current variable types for mets_src_mins_tall... Done.
## Updated "id" from numeric to character.
## Updated "activity_minute" from character to POSIXct.
## Getting target variable types for mets_src_mins_tall done.
## Getting target variable types for sleep_src_mins_tall...
## Getting current variable types for sleep_src_mins_tall... Done.
## Updated "id" from numeric to character.
## Updated "sleep_minute" from character to POSIXct.
## Updated "sleep_log_id" from numeric to character.
## Getting target variable types for sleep_src_mins_tall done.
## Getting target variable types for sleep_sum_days_wide...
## Getting current variable types for sleep_sum_days_wide... Done.
## Updated "id" from numeric to character.
## Updated "sleep_day" from character to POSIXct.
## Getting target variable types for sleep_sum_days_wide done.
## Getting target variable types for steps_src_mins_tall...
## Getting current variable types for steps_src_mins_tall... Done.
## Updated "id" from numeric to character.
## Updated "activity_minute" from character to POSIXct.
## Getting target variable types for steps_src_mins_tall done.
## Getting target variable types for steps_sum_days_tall...
## Getting current variable types for steps_sum_days_tall... Done.
## Updated "id" from numeric to character.
## Updated "activity_day" from character to Date.
## Getting target variable types for steps_sum_days_tall done.
## Getting target variable types for steps_sum_hours_tall...
## Getting current variable types for steps_sum_hours_tall... Done.
## Updated "id" from numeric to character.
## Updated "activity_hour" from character to POSIXct.
## Getting target variable types for steps_sum_hours_tall done.
## Getting target variable types for steps_sum_mins_wide...
## Getting current variable types for steps_sum_mins_wide... Done.
## Updated "id" from numeric to character.
## Updated "activity_hour" from character to POSIXct.
## Getting target variable types for steps_sum_mins_wide done.
\end{verbatim}

\begin{Shaded}
\begin{Highlighting}[]
\FunctionTok{names}\NormalTok{(df\_types\_target) }\OtherTok{\textless{}{-}}\NormalTok{ df\_names}
\FunctionTok{cat}\NormalTok{(}\StringTok{"Generating list of target column types complete.}\SpecialCharTok{\textbackslash{}n}\StringTok{"}\NormalTok{, }\AttributeTok{sep =} \StringTok{""}\NormalTok{)}
\end{Highlighting}
\end{Shaded}

\begin{verbatim}
## Generating list of target column types complete.
\end{verbatim}

\begin{Shaded}
\begin{Highlighting}[]
\FunctionTok{cat}\NormalTok{(}\StringTok{"Recasting variables...}\SpecialCharTok{\textbackslash{}n}\StringTok{"}\NormalTok{, }\AttributeTok{sep =} \StringTok{""}\NormalTok{)}
\end{Highlighting}
\end{Shaded}

\begin{verbatim}
## Recasting variables...
\end{verbatim}

\begin{Shaded}
\begin{Highlighting}[]
\ControlFlowTok{for}\NormalTok{ (df\_name }\ControlFlowTok{in}\NormalTok{ df\_names) \{}
  \FunctionTok{cat}\NormalTok{(}\StringTok{"Recasting "}\NormalTok{,df\_name,}\StringTok{"...}\SpecialCharTok{\textbackslash{}n}\StringTok{"}\NormalTok{, }\AttributeTok{sep =} \StringTok{""}\NormalTok{)}
  \FunctionTok{assign}\NormalTok{(df\_name, }\FunctionTok{recast\_variables}\NormalTok{(df\_name))}
  \FunctionTok{cat}\NormalTok{(}\StringTok{"Converting "}\NormalTok{,df\_name,}\StringTok{" complete.}\SpecialCharTok{\textbackslash{}n}\StringTok{"}\NormalTok{, }\AttributeTok{sep =} \StringTok{""}\NormalTok{)}
\NormalTok{\}}
\end{Highlighting}
\end{Shaded}

\begin{verbatim}
## Recasting activity_sum_days_wide...
## Converting activity_sum_days_wide$id to character... Done.
## Converting activity_sum_days_wide$activity_day to Date... Done.
## Converting activity_sum_days_wide complete.
## Recasting bodycomp_src_logs_wide...
## Converting bodycomp_src_logs_wide$bodycomp_datetime to POSIXct... Done.
## Converting bodycomp_src_logs_wide$id to character... Done.
## Converting bodycomp_src_logs_wide$bodycomp_log_id to character... Done.
## Converting bodycomp_src_logs_wide complete.
## Recasting calories_src_mins_tall...
## Converting calories_src_mins_tall$activity_minute to POSIXct... Done.
## Converting calories_src_mins_tall$id to character... Done.
## Converting calories_src_mins_tall complete.
## Recasting calories_sum_days_tall...
## Converting calories_sum_days_tall$id to character... Done.
## Converting calories_sum_days_tall$activity_day to Date... Done.
## Converting calories_sum_days_tall complete.
## Recasting calories_sum_hours_tall...
## Converting calories_sum_hours_tall$activity_hour to POSIXct... Done.
## Converting calories_sum_hours_tall$id to character... Done.
## Converting calories_sum_hours_tall complete.
## Recasting calories_sum_mins_wide...
## Converting calories_sum_mins_wide$activity_hour to POSIXct... Done.
## Converting calories_sum_mins_wide$id to character... Done.
## Converting calories_sum_mins_wide complete.
## Recasting heartrate_src_seconds_tall...
## Converting heartrate_src_seconds_tall$heart_rate_second to POSIXct... Done.
## Converting heartrate_src_seconds_tall$id to character... Done.
## Converting heartrate_src_seconds_tall complete.
## Recasting intensity_src_mins_tall...
## Converting intensity_src_mins_tall$activity_minute to POSIXct... Done.
## Converting intensity_src_mins_tall$id to character... Done.
## Converting intensity_src_mins_tall complete.
## Recasting intensity_sum_days_wide...
## Converting intensity_sum_days_wide$id to character... Done.
## Converting intensity_sum_days_wide$activity_day to Date... Done.
## Converting intensity_sum_days_wide complete.
## Recasting intensity_sum_hours_wide...
## Converting intensity_sum_hours_wide$activity_hour to POSIXct... Done.
## Converting intensity_sum_hours_wide$id to character... Done.
## Converting intensity_sum_hours_wide complete.
## Recasting intensity_sum_mins_wide...
## Converting intensity_sum_mins_wide$activity_hour to POSIXct... Done.
## Converting intensity_sum_mins_wide$id to character... Done.
## Converting intensity_sum_mins_wide complete.
## Recasting mets_src_mins_tall...
## Converting mets_src_mins_tall$activity_minute to POSIXct... Done.
## Converting mets_src_mins_tall$id to character... Done.
## Converting mets_src_mins_tall complete.
## Recasting sleep_src_mins_tall...
## Converting sleep_src_mins_tall$sleep_minute to POSIXct... Done.
## Converting sleep_src_mins_tall$id to character... Done.
## Converting sleep_src_mins_tall$sleep_log_id to character... Done.
## Converting sleep_src_mins_tall complete.
## Recasting sleep_sum_days_wide...
## Converting sleep_sum_days_wide$sleep_day to POSIXct... Done.
## Converting sleep_sum_days_wide$id to character... Done.
## Converting sleep_sum_days_wide complete.
## Recasting steps_src_mins_tall...
## Converting steps_src_mins_tall$activity_minute to POSIXct... Done.
## Converting steps_src_mins_tall$id to character... Done.
## Converting steps_src_mins_tall complete.
## Recasting steps_sum_days_tall...
## Converting steps_sum_days_tall$id to character... Done.
## Converting steps_sum_days_tall$activity_day to Date... Done.
## Converting steps_sum_days_tall complete.
## Recasting steps_sum_hours_tall...
## Converting steps_sum_hours_tall$activity_hour to POSIXct... Done.
## Converting steps_sum_hours_tall$id to character... Done.
## Converting steps_sum_hours_tall complete.
## Recasting steps_sum_mins_wide...
## Converting steps_sum_mins_wide$activity_hour to POSIXct... Done.
## Converting steps_sum_mins_wide$id to character... Done.
## Converting steps_sum_mins_wide complete.
\end{verbatim}

\begin{Shaded}
\begin{Highlighting}[]
\FunctionTok{cat}\NormalTok{(}\StringTok{"Recasting variables complete.}\SpecialCharTok{\textbackslash{}n}\StringTok{"}\NormalTok{, }\AttributeTok{sep =} \StringTok{""}\NormalTok{)}
\end{Highlighting}
\end{Shaded}

\begin{verbatim}
## Recasting variables complete.
\end{verbatim}

\begin{Shaded}
\begin{Highlighting}[]
\CommentTok{\# Test Recasting of Variables {-}{-}{-}{-}}

\FunctionTok{cat}\NormalTok{(}\StringTok{"Generating list of updated column types...}\SpecialCharTok{\textbackslash{}n}\StringTok{"}\NormalTok{, }\AttributeTok{sep =} \StringTok{""}\NormalTok{)}
\end{Highlighting}
\end{Shaded}

\begin{verbatim}
## Generating list of updated column types...
\end{verbatim}

\begin{Shaded}
\begin{Highlighting}[]
\NormalTok{df\_types\_after }\OtherTok{\textless{}{-}} \FunctionTok{lapply}\NormalTok{(df\_names, get\_df\_var\_types)}
\end{Highlighting}
\end{Shaded}

\begin{verbatim}
## Getting current variable types for activity_sum_days_wide... Done.
## Getting current variable types for bodycomp_src_logs_wide... Done.
## Getting current variable types for calories_src_mins_tall... Done.
## Getting current variable types for calories_sum_days_tall... Done.
## Getting current variable types for calories_sum_hours_tall... Done.
## Getting current variable types for calories_sum_mins_wide... Done.
## Getting current variable types for heartrate_src_seconds_tall... Done.
## Getting current variable types for intensity_src_mins_tall... Done.
## Getting current variable types for intensity_sum_days_wide... Done.
## Getting current variable types for intensity_sum_hours_wide... Done.
## Getting current variable types for intensity_sum_mins_wide... Done.
## Getting current variable types for mets_src_mins_tall... Done.
## Getting current variable types for sleep_src_mins_tall... Done.
## Getting current variable types for sleep_sum_days_wide... Done.
## Getting current variable types for steps_src_mins_tall... Done.
## Getting current variable types for steps_sum_days_tall... Done.
## Getting current variable types for steps_sum_hours_tall... Done.
## Getting current variable types for steps_sum_mins_wide... Done.
\end{verbatim}

\begin{Shaded}
\begin{Highlighting}[]
\FunctionTok{names}\NormalTok{(df\_types\_after) }\OtherTok{\textless{}{-}}\NormalTok{ df\_names}
\FunctionTok{cat}\NormalTok{(}\StringTok{"Generating list of updated column types complete.}\SpecialCharTok{\textbackslash{}n}\StringTok{"}\NormalTok{, }\AttributeTok{sep =} \StringTok{""}\NormalTok{)}
\end{Highlighting}
\end{Shaded}

\begin{verbatim}
## Generating list of updated column types complete.
\end{verbatim}

\begin{Shaded}
\begin{Highlighting}[]
\FunctionTok{cat}\NormalTok{(}\StringTok{"Checking updated column types against target types...}\SpecialCharTok{\textbackslash{}n}\StringTok{"}\NormalTok{, }\AttributeTok{sep =} \StringTok{""}\NormalTok{)}
\end{Highlighting}
\end{Shaded}

\begin{verbatim}
## Checking updated column types against target types...
\end{verbatim}

\begin{Shaded}
\begin{Highlighting}[]
\NormalTok{test\_succeeded }\OtherTok{\textless{}{-}} \FunctionTok{are\_identical\_lists}\NormalTok{(df\_types\_after, df\_types\_target)}
\FunctionTok{cat}\NormalTok{(}\StringTok{"Checking updated column types against target types complete.}\SpecialCharTok{\textbackslash{}n}\StringTok{"}\NormalTok{, }\AttributeTok{sep =} \StringTok{""}\NormalTok{)}
\end{Highlighting}
\end{Shaded}

\begin{verbatim}
## Checking updated column types against target types complete.
\end{verbatim}

\begin{Shaded}
\begin{Highlighting}[]
\FunctionTok{cat}\NormalTok{(}\StringTok{"Data recasting "}\NormalTok{, }\FunctionTok{case\_when}\NormalTok{(test\_succeeded }\SpecialCharTok{\textasciitilde{}} \StringTok{"succeeded"}\NormalTok{, }\ConstantTok{TRUE} \SpecialCharTok{\textasciitilde{}}\StringTok{"failed"}\NormalTok{), }\StringTok{"."}\NormalTok{, }\AttributeTok{sep =} \StringTok{""}\NormalTok{)}
\end{Highlighting}
\end{Shaded}

\begin{verbatim}
## Data recasting succeeded.
\end{verbatim}

\begin{Shaded}
\begin{Highlighting}[]
\FunctionTok{rm}\NormalTok{(df\_name, df\_types\_target, df\_types\_after, test\_succeeded)}
\end{Highlighting}
\end{Shaded}

\begin{Shaded}
\begin{Highlighting}[]
\CommentTok{\# Individual recast of variables with only one occurrence}
\NormalTok{sleep\_src\_mins\_tall }\OtherTok{\textless{}{-}}\NormalTok{ sleep\_src\_mins\_tall }\SpecialCharTok{\%\textgreater{}\%}
    \FunctionTok{mutate}\NormalTok{(}\AttributeTok{sleep\_rank =} \FunctionTok{factor}\NormalTok{(sleep\_rank, }\AttributeTok{levels =} \DecValTok{1}\SpecialCharTok{:}\DecValTok{3}\NormalTok{, }\AttributeTok{labels =} \FunctionTok{c}\NormalTok{(}\StringTok{"Asleep"}\NormalTok{, }\StringTok{"Restless"}\NormalTok{, }\StringTok{"Awake"}\NormalTok{)))}

\NormalTok{intensity\_src\_mins\_tall }\OtherTok{\textless{}{-}}\NormalTok{ intensity\_src\_mins\_tall }\SpecialCharTok{\%\textgreater{}\%}
    \FunctionTok{mutate}\NormalTok{(}\AttributeTok{intensity =} \FunctionTok{factor}\NormalTok{(intensity, }\AttributeTok{levels =} \DecValTok{0}\SpecialCharTok{:}\DecValTok{3}\NormalTok{, }\AttributeTok{labels =} \FunctionTok{c}\NormalTok{(}\StringTok{"Sedentary"}\NormalTok{, }\StringTok{"Lightly Active"}\NormalTok{, }\StringTok{"Fairly Active"}\NormalTok{, }\StringTok{"Very Active"}\NormalTok{)))}
\end{Highlighting}
\end{Shaded}

Given the large number of variables in the data set, the recasting
procedure includes test code to confirm the updated variable types match
the types specified in the variable mods list. The test code works by
first generating a list of the desired final variable/type pairs, then,
once the conversion is completed, generating a second list of the actual
variable/type pairs in the data frames to compare it to. This has the
advantage of not just confirming the desired conversions took place, but
also checks for any unintended changes to variables that did not require
conversion.

Building the test code added a decent amount of work to the project, but
now I have it working, it can be reused and scaled to future projects.

\hypertarget{trim-leading-or-trailing-characters}{%
\subsection{Trim leading or trailing
characters}\label{trim-leading-or-trailing-characters}}

\begin{itemize}
\tightlist
\item
  Manual check for variable names with non-whitespace trailing
  characters
\item
  Automated trim of variable names
\item
  Automated trim of character-type data entries
\end{itemize}

\begin{Shaded}
\begin{Highlighting}[]
\FunctionTok{cat}\NormalTok{(}\StringTok{"Trimming whitespace in variable names...}\SpecialCharTok{\textbackslash{}n}\StringTok{"}\NormalTok{, }\AttributeTok{sep=}\StringTok{""}\NormalTok{)}
\end{Highlighting}
\end{Shaded}

\begin{verbatim}
## Trimming whitespace in variable names...
\end{verbatim}

\begin{Shaded}
\begin{Highlighting}[]
\ControlFlowTok{for}\NormalTok{ (df\_name }\ControlFlowTok{in}\NormalTok{ df\_names) \{}
\NormalTok{  df }\OtherTok{\textless{}{-}} \FunctionTok{get}\NormalTok{(df\_name)}
  \ControlFlowTok{for}\NormalTok{ (col\_name }\ControlFlowTok{in} \FunctionTok{colnames}\NormalTok{(df)) \{}
\NormalTok{    col\_name\_trimmed }\OtherTok{\textless{}{-}} \FunctionTok{str\_trim}\NormalTok{(col\_name)}
    \FunctionTok{cat}\NormalTok{(}\StringTok{"Trimming "}\NormalTok{,df\_name, }\StringTok{"["}\NormalTok{,col\_name,}\StringTok{"] to "}\NormalTok{,col\_name\_trimmed,}\StringTok{"... "}\NormalTok{,}\AttributeTok{sep=}\StringTok{""}\NormalTok{)}
\NormalTok{    df }\OtherTok{\textless{}{-}}\NormalTok{ df }\SpecialCharTok{\%\textgreater{}\%} \FunctionTok{rename}\NormalTok{(}\SpecialCharTok{!!}\AttributeTok{col\_name :=} \SpecialCharTok{!!}\NormalTok{col\_name\_trimmed)}
    \FunctionTok{cat}\NormalTok{(}\StringTok{"Done.}\SpecialCharTok{\textbackslash{}n}\StringTok{"}\NormalTok{, }\AttributeTok{sep =} \StringTok{""}\NormalTok{)}
\NormalTok{  \}}
\NormalTok{\}}
\end{Highlighting}
\end{Shaded}

\begin{verbatim}
## Trimming activity_sum_days_wide[id] to id... Done.
## Trimming activity_sum_days_wide[activity_day] to activity_day... Done.
## Trimming activity_sum_days_wide[steps_total] to steps_total... Done.
## Trimming activity_sum_days_wide[distance_total] to distance_total... Done.
## Trimming activity_sum_days_wide[distance_tracker] to distance_tracker... Done.
## Trimming activity_sum_days_wide[distance_logged_activities] to distance_logged_activities... Done.
## Trimming activity_sum_days_wide[distance_very_active] to distance_very_active... Done.
## Trimming activity_sum_days_wide[distance_moderately_active] to distance_moderately_active... Done.
## Trimming activity_sum_days_wide[distance_lightly_active] to distance_lightly_active... Done.
## Trimming activity_sum_days_wide[distance_sedentary] to distance_sedentary... Done.
## Trimming activity_sum_days_wide[minutes_very_active] to minutes_very_active... Done.
## Trimming activity_sum_days_wide[minutes_fairly_active] to minutes_fairly_active... Done.
## Trimming activity_sum_days_wide[minutes_lightly_active] to minutes_lightly_active... Done.
## Trimming activity_sum_days_wide[minutes_sedentary] to minutes_sedentary... Done.
## Trimming activity_sum_days_wide[calories] to calories... Done.
## Trimming bodycomp_src_logs_wide[id] to id... Done.
## Trimming bodycomp_src_logs_wide[bodycomp_datetime] to bodycomp_datetime... Done.
## Trimming bodycomp_src_logs_wide[weight_kg] to weight_kg... Done.
## Trimming bodycomp_src_logs_wide[weight_pounds] to weight_pounds... Done.
## Trimming bodycomp_src_logs_wide[fat] to fat... Done.
## Trimming bodycomp_src_logs_wide[bmi] to bmi... Done.
## Trimming bodycomp_src_logs_wide[is_manual_report] to is_manual_report... Done.
## Trimming bodycomp_src_logs_wide[bodycomp_log_id] to bodycomp_log_id... Done.
## Trimming calories_src_mins_tall[id] to id... Done.
## Trimming calories_src_mins_tall[activity_minute] to activity_minute... Done.
## Trimming calories_src_mins_tall[calories] to calories... Done.
## Trimming calories_sum_days_tall[id] to id... Done.
## Trimming calories_sum_days_tall[activity_day] to activity_day... Done.
## Trimming calories_sum_days_tall[calories] to calories... Done.
## Trimming calories_sum_hours_tall[id] to id... Done.
## Trimming calories_sum_hours_tall[activity_hour] to activity_hour... Done.
## Trimming calories_sum_hours_tall[calories] to calories... Done.
## Trimming calories_sum_mins_wide[id] to id... Done.
## Trimming calories_sum_mins_wide[activity_hour] to activity_hour... Done.
## Trimming calories_sum_mins_wide[calories00] to calories00... Done.
## Trimming calories_sum_mins_wide[calories01] to calories01... Done.
## Trimming calories_sum_mins_wide[calories02] to calories02... Done.
## Trimming calories_sum_mins_wide[calories03] to calories03... Done.
## Trimming calories_sum_mins_wide[calories04] to calories04... Done.
## Trimming calories_sum_mins_wide[calories05] to calories05... Done.
## Trimming calories_sum_mins_wide[calories06] to calories06... Done.
## Trimming calories_sum_mins_wide[calories07] to calories07... Done.
## Trimming calories_sum_mins_wide[calories08] to calories08... Done.
## Trimming calories_sum_mins_wide[calories09] to calories09... Done.
## Trimming calories_sum_mins_wide[calories10] to calories10... Done.
## Trimming calories_sum_mins_wide[calories11] to calories11... Done.
## Trimming calories_sum_mins_wide[calories12] to calories12... Done.
## Trimming calories_sum_mins_wide[calories13] to calories13... Done.
## Trimming calories_sum_mins_wide[calories14] to calories14... Done.
## Trimming calories_sum_mins_wide[calories15] to calories15... Done.
## Trimming calories_sum_mins_wide[calories16] to calories16... Done.
## Trimming calories_sum_mins_wide[calories17] to calories17... Done.
## Trimming calories_sum_mins_wide[calories18] to calories18... Done.
## Trimming calories_sum_mins_wide[calories19] to calories19... Done.
## Trimming calories_sum_mins_wide[calories20] to calories20... Done.
## Trimming calories_sum_mins_wide[calories21] to calories21... Done.
## Trimming calories_sum_mins_wide[calories22] to calories22... Done.
## Trimming calories_sum_mins_wide[calories23] to calories23... Done.
## Trimming calories_sum_mins_wide[calories24] to calories24... Done.
## Trimming calories_sum_mins_wide[calories25] to calories25... Done.
## Trimming calories_sum_mins_wide[calories26] to calories26... Done.
## Trimming calories_sum_mins_wide[calories27] to calories27... Done.
## Trimming calories_sum_mins_wide[calories28] to calories28... Done.
## Trimming calories_sum_mins_wide[calories29] to calories29... Done.
## Trimming calories_sum_mins_wide[calories30] to calories30... Done.
## Trimming calories_sum_mins_wide[calories31] to calories31... Done.
## Trimming calories_sum_mins_wide[calories32] to calories32... Done.
## Trimming calories_sum_mins_wide[calories33] to calories33... Done.
## Trimming calories_sum_mins_wide[calories34] to calories34... Done.
## Trimming calories_sum_mins_wide[calories35] to calories35... Done.
## Trimming calories_sum_mins_wide[calories36] to calories36... Done.
## Trimming calories_sum_mins_wide[calories37] to calories37... Done.
## Trimming calories_sum_mins_wide[calories38] to calories38... Done.
## Trimming calories_sum_mins_wide[calories39] to calories39... Done.
## Trimming calories_sum_mins_wide[calories40] to calories40... Done.
## Trimming calories_sum_mins_wide[calories41] to calories41... Done.
## Trimming calories_sum_mins_wide[calories42] to calories42... Done.
## Trimming calories_sum_mins_wide[calories43] to calories43... Done.
## Trimming calories_sum_mins_wide[calories44] to calories44... Done.
## Trimming calories_sum_mins_wide[calories45] to calories45... Done.
## Trimming calories_sum_mins_wide[calories46] to calories46... Done.
## Trimming calories_sum_mins_wide[calories47] to calories47... Done.
## Trimming calories_sum_mins_wide[calories48] to calories48... Done.
## Trimming calories_sum_mins_wide[calories49] to calories49... Done.
## Trimming calories_sum_mins_wide[calories50] to calories50... Done.
## Trimming calories_sum_mins_wide[calories51] to calories51... Done.
## Trimming calories_sum_mins_wide[calories52] to calories52... Done.
## Trimming calories_sum_mins_wide[calories53] to calories53... Done.
## Trimming calories_sum_mins_wide[calories54] to calories54... Done.
## Trimming calories_sum_mins_wide[calories55] to calories55... Done.
## Trimming calories_sum_mins_wide[calories56] to calories56... Done.
## Trimming calories_sum_mins_wide[calories57] to calories57... Done.
## Trimming calories_sum_mins_wide[calories58] to calories58... Done.
## Trimming calories_sum_mins_wide[calories59] to calories59... Done.
## Trimming heartrate_src_seconds_tall[id] to id... Done.
## Trimming heartrate_src_seconds_tall[heart_rate_second] to heart_rate_second... Done.
## Trimming heartrate_src_seconds_tall[heart_rate] to heart_rate... Done.
## Trimming intensity_src_mins_tall[id] to id... Done.
## Trimming intensity_src_mins_tall[activity_minute] to activity_minute... Done.
## Trimming intensity_src_mins_tall[intensity] to intensity... Done.
## Trimming intensity_sum_days_wide[id] to id... Done.
## Trimming intensity_sum_days_wide[activity_day] to activity_day... Done.
## Trimming intensity_sum_days_wide[minutes_sedentary] to minutes_sedentary... Done.
## Trimming intensity_sum_days_wide[minutes_lightly_active] to minutes_lightly_active... Done.
## Trimming intensity_sum_days_wide[minutes_fairly_active] to minutes_fairly_active... Done.
## Trimming intensity_sum_days_wide[minutes_very_active] to minutes_very_active... Done.
## Trimming intensity_sum_days_wide[distance_sedentary] to distance_sedentary... Done.
## Trimming intensity_sum_days_wide[distance_lightly_active] to distance_lightly_active... Done.
## Trimming intensity_sum_days_wide[distance_moderately_active] to distance_moderately_active... Done.
## Trimming intensity_sum_days_wide[distance_very_active] to distance_very_active... Done.
## Trimming intensity_sum_hours_wide[id] to id... Done.
## Trimming intensity_sum_hours_wide[activity_hour] to activity_hour... Done.
## Trimming intensity_sum_hours_wide[intensity_total] to intensity_total... Done.
## Trimming intensity_sum_hours_wide[average_intensity] to average_intensity... Done.
## Trimming intensity_sum_mins_wide[id] to id... Done.
## Trimming intensity_sum_mins_wide[activity_hour] to activity_hour... Done.
## Trimming intensity_sum_mins_wide[intensity00] to intensity00... Done.
## Trimming intensity_sum_mins_wide[intensity01] to intensity01... Done.
## Trimming intensity_sum_mins_wide[intensity02] to intensity02... Done.
## Trimming intensity_sum_mins_wide[intensity03] to intensity03... Done.
## Trimming intensity_sum_mins_wide[intensity04] to intensity04... Done.
## Trimming intensity_sum_mins_wide[intensity05] to intensity05... Done.
## Trimming intensity_sum_mins_wide[intensity06] to intensity06... Done.
## Trimming intensity_sum_mins_wide[intensity07] to intensity07... Done.
## Trimming intensity_sum_mins_wide[intensity08] to intensity08... Done.
## Trimming intensity_sum_mins_wide[intensity09] to intensity09... Done.
## Trimming intensity_sum_mins_wide[intensity10] to intensity10... Done.
## Trimming intensity_sum_mins_wide[intensity11] to intensity11... Done.
## Trimming intensity_sum_mins_wide[intensity12] to intensity12... Done.
## Trimming intensity_sum_mins_wide[intensity13] to intensity13... Done.
## Trimming intensity_sum_mins_wide[intensity14] to intensity14... Done.
## Trimming intensity_sum_mins_wide[intensity15] to intensity15... Done.
## Trimming intensity_sum_mins_wide[intensity16] to intensity16... Done.
## Trimming intensity_sum_mins_wide[intensity17] to intensity17... Done.
## Trimming intensity_sum_mins_wide[intensity18] to intensity18... Done.
## Trimming intensity_sum_mins_wide[intensity19] to intensity19... Done.
## Trimming intensity_sum_mins_wide[intensity20] to intensity20... Done.
## Trimming intensity_sum_mins_wide[intensity21] to intensity21... Done.
## Trimming intensity_sum_mins_wide[intensity22] to intensity22... Done.
## Trimming intensity_sum_mins_wide[intensity23] to intensity23... Done.
## Trimming intensity_sum_mins_wide[intensity24] to intensity24... Done.
## Trimming intensity_sum_mins_wide[intensity25] to intensity25... Done.
## Trimming intensity_sum_mins_wide[intensity26] to intensity26... Done.
## Trimming intensity_sum_mins_wide[intensity27] to intensity27... Done.
## Trimming intensity_sum_mins_wide[intensity28] to intensity28... Done.
## Trimming intensity_sum_mins_wide[intensity29] to intensity29... Done.
## Trimming intensity_sum_mins_wide[intensity30] to intensity30... Done.
## Trimming intensity_sum_mins_wide[intensity31] to intensity31... Done.
## Trimming intensity_sum_mins_wide[intensity32] to intensity32... Done.
## Trimming intensity_sum_mins_wide[intensity33] to intensity33... Done.
## Trimming intensity_sum_mins_wide[intensity34] to intensity34... Done.
## Trimming intensity_sum_mins_wide[intensity35] to intensity35... Done.
## Trimming intensity_sum_mins_wide[intensity36] to intensity36... Done.
## Trimming intensity_sum_mins_wide[intensity37] to intensity37... Done.
## Trimming intensity_sum_mins_wide[intensity38] to intensity38... Done.
## Trimming intensity_sum_mins_wide[intensity39] to intensity39... Done.
## Trimming intensity_sum_mins_wide[intensity40] to intensity40... Done.
## Trimming intensity_sum_mins_wide[intensity41] to intensity41... Done.
## Trimming intensity_sum_mins_wide[intensity42] to intensity42... Done.
## Trimming intensity_sum_mins_wide[intensity43] to intensity43... Done.
## Trimming intensity_sum_mins_wide[intensity44] to intensity44... Done.
## Trimming intensity_sum_mins_wide[intensity45] to intensity45... Done.
## Trimming intensity_sum_mins_wide[intensity46] to intensity46... Done.
## Trimming intensity_sum_mins_wide[intensity47] to intensity47... Done.
## Trimming intensity_sum_mins_wide[intensity48] to intensity48... Done.
## Trimming intensity_sum_mins_wide[intensity49] to intensity49... Done.
## Trimming intensity_sum_mins_wide[intensity50] to intensity50... Done.
## Trimming intensity_sum_mins_wide[intensity51] to intensity51... Done.
## Trimming intensity_sum_mins_wide[intensity52] to intensity52... Done.
## Trimming intensity_sum_mins_wide[intensity53] to intensity53... Done.
## Trimming intensity_sum_mins_wide[intensity54] to intensity54... Done.
## Trimming intensity_sum_mins_wide[intensity55] to intensity55... Done.
## Trimming intensity_sum_mins_wide[intensity56] to intensity56... Done.
## Trimming intensity_sum_mins_wide[intensity57] to intensity57... Done.
## Trimming intensity_sum_mins_wide[intensity58] to intensity58... Done.
## Trimming intensity_sum_mins_wide[intensity59] to intensity59... Done.
## Trimming mets_src_mins_tall[id] to id... Done.
## Trimming mets_src_mins_tall[activity_minute] to activity_minute... Done.
## Trimming mets_src_mins_tall[mets] to mets... Done.
## Trimming sleep_src_mins_tall[id] to id... Done.
## Trimming sleep_src_mins_tall[sleep_minute] to sleep_minute... Done.
## Trimming sleep_src_mins_tall[sleep_rank] to sleep_rank... Done.
## Trimming sleep_src_mins_tall[sleep_log_id] to sleep_log_id... Done.
## Trimming sleep_sum_days_wide[id] to id... Done.
## Trimming sleep_sum_days_wide[sleep_day] to sleep_day... Done.
## Trimming sleep_sum_days_wide[sleep_records_total] to sleep_records_total... Done.
## Trimming sleep_sum_days_wide[minutes_asleep_total] to minutes_asleep_total... Done.
## Trimming sleep_sum_days_wide[minutes_in_bed_total] to minutes_in_bed_total... Done.
## Trimming steps_src_mins_tall[id] to id... Done.
## Trimming steps_src_mins_tall[activity_minute] to activity_minute... Done.
## Trimming steps_src_mins_tall[steps] to steps... Done.
## Trimming steps_sum_days_tall[id] to id... Done.
## Trimming steps_sum_days_tall[activity_day] to activity_day... Done.
## Trimming steps_sum_days_tall[steps_total] to steps_total... Done.
## Trimming steps_sum_hours_tall[id] to id... Done.
## Trimming steps_sum_hours_tall[activity_hour] to activity_hour... Done.
## Trimming steps_sum_hours_tall[steps_total] to steps_total... Done.
## Trimming steps_sum_mins_wide[id] to id... Done.
## Trimming steps_sum_mins_wide[activity_hour] to activity_hour... Done.
## Trimming steps_sum_mins_wide[steps00] to steps00... Done.
## Trimming steps_sum_mins_wide[steps01] to steps01... Done.
## Trimming steps_sum_mins_wide[steps02] to steps02... Done.
## Trimming steps_sum_mins_wide[steps03] to steps03... Done.
## Trimming steps_sum_mins_wide[steps04] to steps04... Done.
## Trimming steps_sum_mins_wide[steps05] to steps05... Done.
## Trimming steps_sum_mins_wide[steps06] to steps06... Done.
## Trimming steps_sum_mins_wide[steps07] to steps07... Done.
## Trimming steps_sum_mins_wide[steps08] to steps08... Done.
## Trimming steps_sum_mins_wide[steps09] to steps09... Done.
## Trimming steps_sum_mins_wide[steps10] to steps10... Done.
## Trimming steps_sum_mins_wide[steps11] to steps11... Done.
## Trimming steps_sum_mins_wide[steps12] to steps12... Done.
## Trimming steps_sum_mins_wide[steps13] to steps13... Done.
## Trimming steps_sum_mins_wide[steps14] to steps14... Done.
## Trimming steps_sum_mins_wide[steps15] to steps15... Done.
## Trimming steps_sum_mins_wide[steps16] to steps16... Done.
## Trimming steps_sum_mins_wide[steps17] to steps17... Done.
## Trimming steps_sum_mins_wide[steps18] to steps18... Done.
## Trimming steps_sum_mins_wide[steps19] to steps19... Done.
## Trimming steps_sum_mins_wide[steps20] to steps20... Done.
## Trimming steps_sum_mins_wide[steps21] to steps21... Done.
## Trimming steps_sum_mins_wide[steps22] to steps22... Done.
## Trimming steps_sum_mins_wide[steps23] to steps23... Done.
## Trimming steps_sum_mins_wide[steps24] to steps24... Done.
## Trimming steps_sum_mins_wide[steps25] to steps25... Done.
## Trimming steps_sum_mins_wide[steps26] to steps26... Done.
## Trimming steps_sum_mins_wide[steps27] to steps27... Done.
## Trimming steps_sum_mins_wide[steps28] to steps28... Done.
## Trimming steps_sum_mins_wide[steps29] to steps29... Done.
## Trimming steps_sum_mins_wide[steps30] to steps30... Done.
## Trimming steps_sum_mins_wide[steps31] to steps31... Done.
## Trimming steps_sum_mins_wide[steps32] to steps32... Done.
## Trimming steps_sum_mins_wide[steps33] to steps33... Done.
## Trimming steps_sum_mins_wide[steps34] to steps34... Done.
## Trimming steps_sum_mins_wide[steps35] to steps35... Done.
## Trimming steps_sum_mins_wide[steps36] to steps36... Done.
## Trimming steps_sum_mins_wide[steps37] to steps37... Done.
## Trimming steps_sum_mins_wide[steps38] to steps38... Done.
## Trimming steps_sum_mins_wide[steps39] to steps39... Done.
## Trimming steps_sum_mins_wide[steps40] to steps40... Done.
## Trimming steps_sum_mins_wide[steps41] to steps41... Done.
## Trimming steps_sum_mins_wide[steps42] to steps42... Done.
## Trimming steps_sum_mins_wide[steps43] to steps43... Done.
## Trimming steps_sum_mins_wide[steps44] to steps44... Done.
## Trimming steps_sum_mins_wide[steps45] to steps45... Done.
## Trimming steps_sum_mins_wide[steps46] to steps46... Done.
## Trimming steps_sum_mins_wide[steps47] to steps47... Done.
## Trimming steps_sum_mins_wide[steps48] to steps48... Done.
## Trimming steps_sum_mins_wide[steps49] to steps49... Done.
## Trimming steps_sum_mins_wide[steps50] to steps50... Done.
## Trimming steps_sum_mins_wide[steps51] to steps51... Done.
## Trimming steps_sum_mins_wide[steps52] to steps52... Done.
## Trimming steps_sum_mins_wide[steps53] to steps53... Done.
## Trimming steps_sum_mins_wide[steps54] to steps54... Done.
## Trimming steps_sum_mins_wide[steps55] to steps55... Done.
## Trimming steps_sum_mins_wide[steps56] to steps56... Done.
## Trimming steps_sum_mins_wide[steps57] to steps57... Done.
## Trimming steps_sum_mins_wide[steps58] to steps58... Done.
## Trimming steps_sum_mins_wide[steps59] to steps59... Done.
\end{verbatim}

\begin{Shaded}
\begin{Highlighting}[]
\FunctionTok{rm}\NormalTok{(df, df\_name, col\_name, col\_name\_trimmed)}
\FunctionTok{cat}\NormalTok{(}\StringTok{"Trimming whitespace in variable names complete.}\SpecialCharTok{\textbackslash{}n}\StringTok{"}\NormalTok{, }\AttributeTok{sep=}\StringTok{""}\NormalTok{)}
\end{Highlighting}
\end{Shaded}

\begin{verbatim}
## Trimming whitespace in variable names complete.
\end{verbatim}

\begin{Shaded}
\begin{Highlighting}[]
\NormalTok{trim\_chr\_column }\OtherTok{\textless{}{-}} \ControlFlowTok{function}\NormalTok{(col) \{}
  \ControlFlowTok{if}\NormalTok{ (}\FunctionTok{is.character}\NormalTok{(col)) \{}
    \FunctionTok{return}\NormalTok{(}\FunctionTok{str\_trim}\NormalTok{(col))}
\NormalTok{  \} }\ControlFlowTok{else}\NormalTok{ \{}
    \FunctionTok{return}\NormalTok{(col)}
\NormalTok{  \}}
\NormalTok{\}}

\FunctionTok{cat}\NormalTok{(}\StringTok{"Trimming whitespace in character{-}type data values...}\SpecialCharTok{\textbackslash{}n}\StringTok{"}\NormalTok{, }\AttributeTok{sep=}\StringTok{""}\NormalTok{)}
\end{Highlighting}
\end{Shaded}

\begin{verbatim}
## Trimming whitespace in character-type data values...
\end{verbatim}

\begin{Shaded}
\begin{Highlighting}[]
\ControlFlowTok{for}\NormalTok{ (df\_name }\ControlFlowTok{in}\NormalTok{ df\_names) \{}
\NormalTok{  df }\OtherTok{\textless{}{-}} \FunctionTok{get}\NormalTok{(df\_name)}
  \ControlFlowTok{for}\NormalTok{ (col\_name }\ControlFlowTok{in} \FunctionTok{colnames}\NormalTok{(df)) \{}
    \ControlFlowTok{if}\NormalTok{ (}\FunctionTok{is.character}\NormalTok{(df[[col\_name]])) \{}
      \FunctionTok{cat}\NormalTok{(}\StringTok{"Trimming "}\NormalTok{,df\_name, }\StringTok{"["}\NormalTok{,col\_name,}\StringTok{"]... "}\NormalTok{,}\AttributeTok{sep=}\StringTok{""}\NormalTok{)}
      \CommentTok{\# df \textless{}{-} df \%\textgreater{}\% mutate(if\_any(where(is.character), trim\_chr\_column))}
\NormalTok{      df }\OtherTok{\textless{}{-}}\NormalTok{ df }\SpecialCharTok{\%\textgreater{}\%} \FunctionTok{mutate}\NormalTok{(}\FunctionTok{across}\NormalTok{(}\FunctionTok{all\_of}\NormalTok{(col\_name), str\_trim))}
      \FunctionTok{cat}\NormalTok{(}\StringTok{"Done.}\SpecialCharTok{\textbackslash{}n}\StringTok{"}\NormalTok{, }\AttributeTok{sep =} \StringTok{""}\NormalTok{)}
\NormalTok{    \}}
\NormalTok{  \}}
\NormalTok{\}}
\end{Highlighting}
\end{Shaded}

\begin{verbatim}
## Trimming activity_sum_days_wide[id]... Done.
## Trimming bodycomp_src_logs_wide[id]... Done.
## Trimming bodycomp_src_logs_wide[bodycomp_log_id]... Done.
## Trimming calories_src_mins_tall[id]... Done.
## Trimming calories_sum_days_tall[id]... Done.
## Trimming calories_sum_hours_tall[id]... Done.
## Trimming calories_sum_mins_wide[id]... Done.
## Trimming heartrate_src_seconds_tall[id]... Done.
## Trimming intensity_src_mins_tall[id]... Done.
## Trimming intensity_sum_days_wide[id]... Done.
## Trimming intensity_sum_hours_wide[id]... Done.
## Trimming intensity_sum_mins_wide[id]... Done.
## Trimming mets_src_mins_tall[id]... Done.
## Trimming sleep_src_mins_tall[id]... Done.
## Trimming sleep_src_mins_tall[sleep_log_id]... Done.
## Trimming sleep_sum_days_wide[id]... Done.
## Trimming steps_src_mins_tall[id]... Done.
## Trimming steps_sum_days_tall[id]... Done.
## Trimming steps_sum_hours_tall[id]... Done.
## Trimming steps_sum_mins_wide[id]... Done.
\end{verbatim}

\begin{Shaded}
\begin{Highlighting}[]
\FunctionTok{rm}\NormalTok{(df, df\_name, col\_name)}
\FunctionTok{cat}\NormalTok{(}\StringTok{"Trimming whitespace in character{-}type data values complete.}\SpecialCharTok{\textbackslash{}n}\StringTok{"}\NormalTok{, }\AttributeTok{sep=}\StringTok{""}\NormalTok{)}
\end{Highlighting}
\end{Shaded}

\begin{verbatim}
## Trimming whitespace in character-type data values complete.
\end{verbatim}

\hypertarget{validate-numeric-data}{%
\subsection{Validate numeric data}\label{validate-numeric-data}}

Given that the data was not entered manually by the user, there's no way
for me to manually check the correctness of all of the numeric values
included in the data set. The values were instead checked against
pre-determined limits to verify that they fall within realistic ranges,
as detailed below.

Validating the data in this way also helps confirm the data makes sense
in terms of the business logic, by confirming the data falls within
realistic ranges given the capabilities of the devices and the types of
data they claim to track.

\hypertarget{id-lengths-and-value-ranges-are-correct}{%
\subsubsection{ID lengths and value ranges are
correct}\label{id-lengths-and-value-ranges-are-correct}}

I wrote a short function to validate the length of all rows in a data
frame for a given column number and valid length. This was then used to
check ID values in the data set against their correct length, including:

\begin{itemize}
\tightlist
\item
  ``id'' values: 10-digit
\item
  ``sleep\_log\_id'' values: 11-digit
\item
  ``bodycomp\_log\_id'' values: 13-digit
\end{itemize}

The validation confirmed all values were the correct length.

\begin{Shaded}
\begin{Highlighting}[]
\NormalTok{validate\_string\_length }\OtherTok{\textless{}{-}} \ControlFlowTok{function}\NormalTok{(df\_name, col\_name, valid\_length) \{}
  \FunctionTok{cat}\NormalTok{(}\StringTok{"Checking "}\NormalTok{,df\_name,}\StringTok{" for invalid "}\NormalTok{,col\_name,}\StringTok{" values... "}\NormalTok{,}\AttributeTok{sep=}\StringTok{""}\NormalTok{)}
\NormalTok{  df }\OtherTok{\textless{}{-}} \FunctionTok{get}\NormalTok{(df\_name)}
  \ControlFlowTok{if}\NormalTok{ (col\_name }\SpecialCharTok{\%in\%} \FunctionTok{colnames}\NormalTok{(df)) \{}
\NormalTok{    invalid\_values }\OtherTok{\textless{}{-}}\NormalTok{ df }\SpecialCharTok{\%\textgreater{}\%} \FunctionTok{select}\NormalTok{(}\SpecialCharTok{!!}\FunctionTok{sym}\NormalTok{(col\_name)) }\SpecialCharTok{\%\textgreater{}\%} \FunctionTok{filter}\NormalTok{(}\FunctionTok{nchar}\NormalTok{(}\SpecialCharTok{!!}\FunctionTok{sym}\NormalTok{(col\_name)) }\SpecialCharTok{!=}\NormalTok{ valid\_length)}
\NormalTok{  \}}
  \ControlFlowTok{if}\NormalTok{ (}\FunctionTok{exists}\NormalTok{(}\StringTok{\textquotesingle{}invalid\_values\textquotesingle{}}\NormalTok{) }\SpecialCharTok{\&\&} \FunctionTok{nrow}\NormalTok{(invalid\_values) }\SpecialCharTok{\textgreater{}} \DecValTok{0}\NormalTok{) \{}
    \FunctionTok{cat}\NormalTok{(}\FunctionTok{nrow}\NormalTok{(invalid\_values),}\StringTok{" invalid values found:}\SpecialCharTok{\textbackslash{}n}\StringTok{"}\NormalTok{,}\AttributeTok{sep=}\StringTok{""}\NormalTok{)}
    \FunctionTok{glimpse}\NormalTok{(invalid\_values)}
    \FunctionTok{rm}\NormalTok{(invalid\_values)}
\NormalTok{  \} }\ControlFlowTok{else}\NormalTok{ \{}
    \FunctionTok{cat}\NormalTok{(}\StringTok{"complete.}\SpecialCharTok{\textbackslash{}n}\StringTok{"}\NormalTok{,}\AttributeTok{sep=}\StringTok{""}\NormalTok{)}
\NormalTok{  \}}
  \FunctionTok{rm}\NormalTok{(df)}
\NormalTok{\}}

\NormalTok{result }\OtherTok{\textless{}{-}} \FunctionTok{map}\NormalTok{(df\_names, }\SpecialCharTok{\textasciitilde{}}\FunctionTok{validate\_string\_length}\NormalTok{(.x, }\AttributeTok{col\_name =} \StringTok{"id"}\NormalTok{, }\AttributeTok{valid\_length =} \DecValTok{10}\NormalTok{))}
\end{Highlighting}
\end{Shaded}

\begin{verbatim}
## Checking activity_sum_days_wide for invalid id values... complete.
## Checking bodycomp_src_logs_wide for invalid id values... complete.
## Checking calories_src_mins_tall for invalid id values... complete.
## Checking calories_sum_days_tall for invalid id values... complete.
## Checking calories_sum_hours_tall for invalid id values... complete.
## Checking calories_sum_mins_wide for invalid id values... complete.
## Checking heartrate_src_seconds_tall for invalid id values... complete.
## Checking intensity_src_mins_tall for invalid id values... complete.
## Checking intensity_sum_days_wide for invalid id values... complete.
## Checking intensity_sum_hours_wide for invalid id values... complete.
## Checking intensity_sum_mins_wide for invalid id values... complete.
## Checking mets_src_mins_tall for invalid id values... complete.
## Checking sleep_src_mins_tall for invalid id values... complete.
## Checking sleep_sum_days_wide for invalid id values... complete.
## Checking steps_src_mins_tall for invalid id values... complete.
## Checking steps_sum_days_tall for invalid id values... complete.
## Checking steps_sum_hours_tall for invalid id values... complete.
## Checking steps_sum_mins_wide for invalid id values... complete.
\end{verbatim}

\begin{Shaded}
\begin{Highlighting}[]
\NormalTok{result }\OtherTok{\textless{}{-}} \FunctionTok{map}\NormalTok{(df\_names, }\SpecialCharTok{\textasciitilde{}}\FunctionTok{validate\_string\_length}\NormalTok{(.x, }\AttributeTok{col\_name =} \StringTok{"sleep\_log\_id"}\NormalTok{, }\AttributeTok{valid\_length =} \DecValTok{11}\NormalTok{))}
\end{Highlighting}
\end{Shaded}

\begin{verbatim}
## Checking activity_sum_days_wide for invalid sleep_log_id values... complete.
## Checking bodycomp_src_logs_wide for invalid sleep_log_id values... complete.
## Checking calories_src_mins_tall for invalid sleep_log_id values... complete.
## Checking calories_sum_days_tall for invalid sleep_log_id values... complete.
## Checking calories_sum_hours_tall for invalid sleep_log_id values... complete.
## Checking calories_sum_mins_wide for invalid sleep_log_id values... complete.
## Checking heartrate_src_seconds_tall for invalid sleep_log_id values... complete.
## Checking intensity_src_mins_tall for invalid sleep_log_id values... complete.
## Checking intensity_sum_days_wide for invalid sleep_log_id values... complete.
## Checking intensity_sum_hours_wide for invalid sleep_log_id values... complete.
## Checking intensity_sum_mins_wide for invalid sleep_log_id values... complete.
## Checking mets_src_mins_tall for invalid sleep_log_id values... complete.
## Checking sleep_src_mins_tall for invalid sleep_log_id values... complete.
## Checking sleep_sum_days_wide for invalid sleep_log_id values... complete.
## Checking steps_src_mins_tall for invalid sleep_log_id values... complete.
## Checking steps_sum_days_tall for invalid sleep_log_id values... complete.
## Checking steps_sum_hours_tall for invalid sleep_log_id values... complete.
## Checking steps_sum_mins_wide for invalid sleep_log_id values... complete.
\end{verbatim}

\begin{Shaded}
\begin{Highlighting}[]
\NormalTok{result }\OtherTok{\textless{}{-}} \FunctionTok{map}\NormalTok{(df\_names, }\SpecialCharTok{\textasciitilde{}}\FunctionTok{validate\_string\_length}\NormalTok{(.x, }\AttributeTok{col\_name =} \StringTok{"bodycomp\_log\_id"}\NormalTok{, }\AttributeTok{valid\_length =} \DecValTok{13}\NormalTok{))}
\end{Highlighting}
\end{Shaded}

\begin{verbatim}
## Checking activity_sum_days_wide for invalid bodycomp_log_id values... complete.
## Checking bodycomp_src_logs_wide for invalid bodycomp_log_id values... complete.
## Checking calories_src_mins_tall for invalid bodycomp_log_id values... complete.
## Checking calories_sum_days_tall for invalid bodycomp_log_id values... complete.
## Checking calories_sum_hours_tall for invalid bodycomp_log_id values... complete.
## Checking calories_sum_mins_wide for invalid bodycomp_log_id values... complete.
## Checking heartrate_src_seconds_tall for invalid bodycomp_log_id values... complete.
## Checking intensity_src_mins_tall for invalid bodycomp_log_id values... complete.
## Checking intensity_sum_days_wide for invalid bodycomp_log_id values... complete.
## Checking intensity_sum_hours_wide for invalid bodycomp_log_id values... complete.
## Checking intensity_sum_mins_wide for invalid bodycomp_log_id values... complete.
## Checking mets_src_mins_tall for invalid bodycomp_log_id values... complete.
## Checking sleep_src_mins_tall for invalid bodycomp_log_id values... complete.
## Checking sleep_sum_days_wide for invalid bodycomp_log_id values... complete.
## Checking steps_src_mins_tall for invalid bodycomp_log_id values... complete.
## Checking steps_sum_days_tall for invalid bodycomp_log_id values... complete.
## Checking steps_sum_hours_tall for invalid bodycomp_log_id values... complete.
## Checking steps_sum_mins_wide for invalid bodycomp_log_id values... complete.
\end{verbatim}

\begin{Shaded}
\begin{Highlighting}[]
\FunctionTok{rm}\NormalTok{(result)}
\end{Highlighting}
\end{Shaded}

To further validate the large number of ID data points in the data set,
I wrote a short function to check all ID variables in each data frame
and determine the minimum and maximum values. The intent was to identify
any possible erroneous values within the acceptable length limits,
e.g.~all zeroes or all nines. The function makes use of the direct
conversion of strings to numerals in the min() and max() functions to
carry out the comparison on character-type values.

The validation confirmed all values were within a realistic range.

\begin{Shaded}
\begin{Highlighting}[]
\FunctionTok{cat}\NormalTok{(}\StringTok{"Checking min/max ID value ranges...}\SpecialCharTok{\textbackslash{}n}\StringTok{"}\NormalTok{,}\AttributeTok{sep=}\StringTok{""}\NormalTok{)}
\end{Highlighting}
\end{Shaded}

\begin{verbatim}
## Checking min/max ID value ranges...
\end{verbatim}

\begin{Shaded}
\begin{Highlighting}[]
\NormalTok{id\_col\_names }\OtherTok{\textless{}{-}} \FunctionTok{c}\NormalTok{(}\StringTok{"id"}\NormalTok{, }\StringTok{"sleep\_log\_id"}\NormalTok{, }\StringTok{"bodycomp\_log\_id"}\NormalTok{)}
\ControlFlowTok{for}\NormalTok{ (df\_name }\ControlFlowTok{in}\NormalTok{ df\_names) \{}
\NormalTok{  df }\OtherTok{\textless{}{-}} \FunctionTok{get}\NormalTok{(df\_name)}
  \ControlFlowTok{for}\NormalTok{ (col\_name }\ControlFlowTok{in}\NormalTok{ id\_col\_names) \{}
    \ControlFlowTok{if}\NormalTok{(col\_name }\SpecialCharTok{\%in\%} \FunctionTok{colnames}\NormalTok{(df)) \{}
      \FunctionTok{cat}\NormalTok{(df\_name,}\StringTok{"["}\NormalTok{,col\_name,}\StringTok{"] Min = "}\NormalTok{,}\FunctionTok{min}\NormalTok{(df[[col\_name]]),}\StringTok{" Max = "}\NormalTok{,}\FunctionTok{max}\NormalTok{(df[[col\_name]]),}\StringTok{".}\SpecialCharTok{\textbackslash{}n}\StringTok{"}\NormalTok{,}\AttributeTok{sep=}\StringTok{""}\NormalTok{)}
\NormalTok{    \}}
\NormalTok{  \}}
\NormalTok{\}}
\end{Highlighting}
\end{Shaded}

\begin{verbatim}
## activity_sum_days_wide[id] Min = 1503960366 Max = 8877689391.
## bodycomp_src_logs_wide[id] Min = 1503960366 Max = 8877689391.
## bodycomp_src_logs_wide[bodycomp_log_id] Min = 1460443631000 Max = 1463097599000.
## calories_src_mins_tall[id] Min = 1503960366 Max = 8877689391.
## calories_sum_days_tall[id] Min = 1503960366 Max = 8877689391.
## calories_sum_hours_tall[id] Min = 1503960366 Max = 8877689391.
## calories_sum_mins_wide[id] Min = 1503960366 Max = 8877689391.
## heartrate_src_seconds_tall[id] Min = 2022484408 Max = 8877689391.
## intensity_src_mins_tall[id] Min = 1503960366 Max = 8877689391.
## intensity_sum_days_wide[id] Min = 1503960366 Max = 8877689391.
## intensity_sum_hours_wide[id] Min = 1503960366 Max = 8877689391.
## intensity_sum_mins_wide[id] Min = 1503960366 Max = 8877689391.
## mets_src_mins_tall[id] Min = 1503960366 Max = 8877689391.
## sleep_src_mins_tall[id] Min = 1503960366 Max = 8792009665.
## sleep_src_mins_tall[sleep_log_id] Min = 11372227280 Max = 11616251768.
## sleep_sum_days_wide[id] Min = 1503960366 Max = 8792009665.
## steps_src_mins_tall[id] Min = 1503960366 Max = 8877689391.
## steps_sum_days_tall[id] Min = 1503960366 Max = 8877689391.
## steps_sum_hours_tall[id] Min = 1503960366 Max = 8877689391.
## steps_sum_mins_wide[id] Min = 1503960366 Max = 8877689391.
\end{verbatim}

\begin{Shaded}
\begin{Highlighting}[]
\FunctionTok{rm}\NormalTok{(df, df\_name, col\_name, id\_col\_names)}
\FunctionTok{cat}\NormalTok{(}\StringTok{"Checking min/max ID value ranges complete.}\SpecialCharTok{\textbackslash{}n}\StringTok{"}\NormalTok{,}\AttributeTok{sep=}\StringTok{""}\NormalTok{)}
\end{Highlighting}
\end{Shaded}

\begin{verbatim}
## Checking min/max ID value ranges complete.
\end{verbatim}

\hypertarget{dates-are-all-within-range}{%
\subsubsection{Dates are all within
range}\label{dates-are-all-within-range}}

The data set is described as containing data from users collected
between ``03.12.2016-05.12.2016'': all records in the data set were
validated against this date range.

\begin{Shaded}
\begin{Highlighting}[]
\NormalTok{validate\_dates }\OtherTok{\textless{}{-}} \ControlFlowTok{function}\NormalTok{(df\_name) \{}
\NormalTok{  valid\_dates\_stt }\OtherTok{\textless{}{-}} \FunctionTok{as.Date}\NormalTok{(}\StringTok{"2016{-}03{-}12"}\NormalTok{, }\AttributeTok{format =} \StringTok{"\%Y{-}\%m{-}\%d"}\NormalTok{)}
\NormalTok{  valid\_dates\_end }\OtherTok{\textless{}{-}} \FunctionTok{as.Date}\NormalTok{(}\StringTok{"2016{-}05{-}14"}\NormalTok{, }\AttributeTok{format =} \StringTok{"\%Y{-}\%m{-}\%d"}\NormalTok{)}
  
  \FunctionTok{cat}\NormalTok{(}\StringTok{"Validating dates for "}\NormalTok{,df\_name,}\StringTok{"... "}\NormalTok{,}\AttributeTok{sep=}\StringTok{""}\NormalTok{)}
  
\NormalTok{  date\_columns }\OtherTok{\textless{}{-}} \FunctionTok{get}\NormalTok{(df\_name) }\SpecialCharTok{\%\textgreater{}\%}
    \FunctionTok{select\_if}\NormalTok{(}\ControlFlowTok{function}\NormalTok{(col) }\FunctionTok{is.POSIXct}\NormalTok{(col) }\SpecialCharTok{||} \FunctionTok{is.Date}\NormalTok{(col))}
  
  \ControlFlowTok{if}\NormalTok{ (}\FunctionTok{ncol}\NormalTok{(date\_columns) }\SpecialCharTok{==} \DecValTok{0}\NormalTok{) \{}
    \FunctionTok{cat}\NormalTok{(}\StringTok{"Error: no date column found.}\SpecialCharTok{\textbackslash{}n}\StringTok{"}\NormalTok{,}\AttributeTok{sep=}\StringTok{""}\NormalTok{)}
\NormalTok{  \} }\ControlFlowTok{else} \ControlFlowTok{if}\NormalTok{ (}\FunctionTok{ncol}\NormalTok{(date\_columns) }\SpecialCharTok{\textgreater{}} \DecValTok{1}\NormalTok{) \{}
    \FunctionTok{cat}\NormalTok{(}\StringTok{"Error: more than one date column found for "}\NormalTok{,df\_name,}\StringTok{".}\SpecialCharTok{\textbackslash{}n}\StringTok{"}\NormalTok{,}\AttributeTok{sep=}\StringTok{""}\NormalTok{)}
\NormalTok{  \} }\ControlFlowTok{else}\NormalTok{ \{}
\NormalTok{    outside\_range }\OtherTok{\textless{}{-}}\NormalTok{ date\_columns }\SpecialCharTok{\%\textgreater{}\%}
      \FunctionTok{filter}\NormalTok{(}\FunctionTok{if\_any}\NormalTok{(}\FunctionTok{everything}\NormalTok{(), }\SpecialCharTok{\textasciitilde{}}\NormalTok{ . }\SpecialCharTok{\textless{}}\NormalTok{ valid\_dates\_stt }\SpecialCharTok{|}\NormalTok{ . }\SpecialCharTok{\textgreater{}}\NormalTok{ valid\_dates\_end))}
    \ControlFlowTok{if}\NormalTok{ (}\FunctionTok{nrow}\NormalTok{(outside\_range) }\SpecialCharTok{\textgreater{}} \DecValTok{0}\NormalTok{) \{}
      \FunctionTok{cat}\NormalTok{(}\StringTok{"found "}\NormalTok{,}\FunctionTok{nrow}\NormalTok{(outside\_range),}\StringTok{" invalid values:}\SpecialCharTok{\textbackslash{}n}\StringTok{"}\NormalTok{,}\AttributeTok{sep=}\StringTok{""}\NormalTok{)}
      \FunctionTok{print}\NormalTok{(outside\_range)}
      \FunctionTok{rm}\NormalTok{(outside\_range)}
\NormalTok{    \} }\ControlFlowTok{else}\NormalTok{ \{}
      \FunctionTok{cat}\NormalTok{(}\StringTok{"complete.}\SpecialCharTok{\textbackslash{}n}\StringTok{"}\NormalTok{)}
\NormalTok{    \}}
\NormalTok{  \}}
  \FunctionTok{rm}\NormalTok{(date\_columns)}
\NormalTok{\}}

\NormalTok{result }\OtherTok{\textless{}{-}} \FunctionTok{lapply}\NormalTok{(df\_names, validate\_dates)}
\end{Highlighting}
\end{Shaded}

\begin{verbatim}
## Validating dates for activity_sum_days_wide... complete.
## Validating dates for bodycomp_src_logs_wide... complete.
## Validating dates for calories_src_mins_tall... complete.
## Validating dates for calories_sum_days_tall... complete.
## Validating dates for calories_sum_hours_tall... complete.
## Validating dates for calories_sum_mins_wide... complete.
## Validating dates for heartrate_src_seconds_tall... complete.
## Validating dates for intensity_src_mins_tall... complete.
## Validating dates for intensity_sum_days_wide... complete.
## Validating dates for intensity_sum_hours_wide... complete.
## Validating dates for intensity_sum_mins_wide... complete.
## Validating dates for mets_src_mins_tall... complete.
## Validating dates for sleep_src_mins_tall... complete.
## Validating dates for sleep_sum_days_wide... complete.
## Validating dates for steps_src_mins_tall... complete.
## Validating dates for steps_sum_days_tall... complete.
## Validating dates for steps_sum_hours_tall... complete.
## Validating dates for steps_sum_mins_wide... complete.
\end{verbatim}

\begin{Shaded}
\begin{Highlighting}[]
\FunctionTok{rm}\NormalTok{(result)}
\end{Highlighting}
\end{Shaded}

This check found dates just outside the range, dating up to 8am on
05.13.2016, the day after the data set supposedly ended. I didn't
consider this to be a problem, so the valid end-date was updated to the
14th of May 2016 accordingly, and all dates passed this check.

\hypertarget{non-date-values-are-within-appropriate-ranges-for-their-units}{%
\subsubsection{Non-date values are within appropriate ranges for their
units}\label{non-date-values-are-within-appropriate-ranges-for-their-units}}

Checklist: - All numeric values are non-negative - Percentage values are
less than 100 - Weight values are positive and make sense (e.g.~less
than 200kg) - BMI values are in range - Daily, hourly, and minute
duration sums are no more than one day, hour, or minute, respectively -
Distances make sense - Step counts make sense (check for
\textgreater20000 for a start) - Calories are within normal range -
Heart rates are less than 200

First, a check for negative values was run on all numeric columns in the
data set.

\begin{Shaded}
\begin{Highlighting}[]
\NormalTok{validate\_numerics }\OtherTok{\textless{}{-}} \ControlFlowTok{function}\NormalTok{(df\_name) \{}
  \CommentTok{\# This function performs all checks on numeric values that are required in more than one data{-}frame, e.g. non{-}negativity and summation}
  \FunctionTok{cat}\NormalTok{(}\StringTok{"Validating numerics for "}\NormalTok{,df\_name,}\StringTok{"... "}\NormalTok{,}\AttributeTok{sep=}\StringTok{""}\NormalTok{)}
\NormalTok{  numerics }\OtherTok{\textless{}{-}} \FunctionTok{get}\NormalTok{(df\_name) }\SpecialCharTok{\%\textgreater{}\%} \FunctionTok{select\_if}\NormalTok{(is.numeric)}
  \ControlFlowTok{if}\NormalTok{ (}\FunctionTok{ncol}\NormalTok{(numerics) }\SpecialCharTok{==} \DecValTok{0}\NormalTok{) \{}
    \FunctionTok{cat}\NormalTok{(}\StringTok{"No numeric variables found.}\SpecialCharTok{\textbackslash{}n}\StringTok{"}\NormalTok{,}\AttributeTok{sep=}\StringTok{""}\NormalTok{)}
\NormalTok{  \} }\ControlFlowTok{else}\NormalTok{ \{}
    \CommentTok{\# Check for negative values}
\NormalTok{    negative\_values }\OtherTok{\textless{}{-}}\NormalTok{ numerics }\SpecialCharTok{\%\textgreater{}\%}
      \FunctionTok{filter}\NormalTok{(}\FunctionTok{if\_any}\NormalTok{(}\FunctionTok{everything}\NormalTok{(), }\SpecialCharTok{\textasciitilde{}}\NormalTok{ . }\SpecialCharTok{\textless{}} \DecValTok{0}\NormalTok{))}
    \ControlFlowTok{if}\NormalTok{ (}\FunctionTok{nrow}\NormalTok{(negative\_values) }\SpecialCharTok{\textgreater{}} \DecValTok{0}\NormalTok{) \{}
      \FunctionTok{cat}\NormalTok{(}\StringTok{"found "}\NormalTok{,}\FunctionTok{nrow}\NormalTok{(negative\_values),}\StringTok{" invalid values:}\SpecialCharTok{\textbackslash{}n}\StringTok{"}\NormalTok{,}\AttributeTok{sep=}\StringTok{""}\NormalTok{)}
      \FunctionTok{print}\NormalTok{(negative\_values)}
\NormalTok{    \} }\ControlFlowTok{else}\NormalTok{ \{}
      \FunctionTok{cat}\NormalTok{(}\StringTok{"complete.}\SpecialCharTok{\textbackslash{}n}\StringTok{"}\NormalTok{)}
\NormalTok{    \}}
    \FunctionTok{rm}\NormalTok{(negative\_values)}
    \CommentTok{\# Check for summation}
\NormalTok{  \}}
  \FunctionTok{rm}\NormalTok{(numerics)}
\NormalTok{\}}

\NormalTok{result }\OtherTok{\textless{}{-}} \FunctionTok{lapply}\NormalTok{(df\_names, validate\_numerics)}
\end{Highlighting}
\end{Shaded}

\begin{verbatim}
## Validating numerics for activity_sum_days_wide... complete.
## Validating numerics for bodycomp_src_logs_wide... complete.
## Validating numerics for calories_src_mins_tall... complete.
## Validating numerics for calories_sum_days_tall... complete.
## Validating numerics for calories_sum_hours_tall... complete.
## Validating numerics for calories_sum_mins_wide... complete.
## Validating numerics for heartrate_src_seconds_tall... complete.
## Validating numerics for intensity_src_mins_tall... No numeric variables found.
## Validating numerics for intensity_sum_days_wide... complete.
## Validating numerics for intensity_sum_hours_wide... complete.
## Validating numerics for intensity_sum_mins_wide... complete.
## Validating numerics for mets_src_mins_tall... complete.
## Validating numerics for sleep_src_mins_tall... No numeric variables found.
## Validating numerics for sleep_sum_days_wide... complete.
## Validating numerics for steps_src_mins_tall... complete.
## Validating numerics for steps_sum_days_tall... complete.
## Validating numerics for steps_sum_hours_tall... complete.
## Validating numerics for steps_sum_mins_wide... complete.
\end{verbatim}

\begin{Shaded}
\begin{Highlighting}[]
\FunctionTok{rm}\NormalTok{(result)}
\end{Highlighting}
\end{Shaded}

Results:

\begin{itemize}
\tightlist
\item
  No negative values were found in the data.
\end{itemize}

Second, variable-specific checks were run to confirm the data fell
within realistic ranges given my understanding of the variables being
measured.

Note: The maximum values given are used as thresholds above which values
may not be realistic, not as hard limits for acceptability: a heart-rate
of 200bpm, for example, is entirely possible, but it is high enough that
I would want to check if the data point corresponded to a period of
high-intensity exercise.

\begin{Shaded}
\begin{Highlighting}[]
\CommentTok{\# For a given list of dfs, column names, and a min{-}max range, check all matching columns in all matching dfs against that range}
\NormalTok{validate\_within\_range }\OtherTok{\textless{}{-}} \ControlFlowTok{function}\NormalTok{(df\_name, column\_names, range\_min, range\_max) \{}
\NormalTok{  df }\OtherTok{\textless{}{-}} \FunctionTok{get}\NormalTok{(df\_name)}
  \ControlFlowTok{for}\NormalTok{ (col\_name }\ControlFlowTok{in}\NormalTok{ column\_names) \{}
    \FunctionTok{cat}\NormalTok{(}\StringTok{"Checking ranges for "}\NormalTok{,df\_name,}\StringTok{"["}\NormalTok{,col\_name,}\StringTok{"]... "}\NormalTok{,}\AttributeTok{sep=}\StringTok{""}\NormalTok{)}
    \ControlFlowTok{if}\NormalTok{ (}\SpecialCharTok{!}\NormalTok{(col\_name }\SpecialCharTok{\%in\%} \FunctionTok{colnames}\NormalTok{(df))) \{}
      \FunctionTok{cat}\NormalTok{(}\StringTok{"column not found.}\SpecialCharTok{\textbackslash{}n}\StringTok{"}\NormalTok{)}
\NormalTok{    \} }\ControlFlowTok{else}\NormalTok{ \{}
\NormalTok{      out\_of\_range }\OtherTok{\textless{}{-}}\NormalTok{ df }\SpecialCharTok{\%\textgreater{}\%}
        \FunctionTok{filter}\NormalTok{(}\SpecialCharTok{!!}\FunctionTok{sym}\NormalTok{(col\_name) }\SpecialCharTok{\textless{}}\NormalTok{ range\_min }\SpecialCharTok{|} \SpecialCharTok{!!}\FunctionTok{sym}\NormalTok{(col\_name) }\SpecialCharTok{\textgreater{}}\NormalTok{ range\_max)}
      \ControlFlowTok{if}\NormalTok{ (}\FunctionTok{nrow}\NormalTok{(out\_of\_range) }\SpecialCharTok{\textless{}=} \DecValTok{0}\NormalTok{) \{}
        \FunctionTok{cat}\NormalTok{(}\StringTok{"complete.}\SpecialCharTok{\textbackslash{}n}\StringTok{"}\NormalTok{,}\AttributeTok{sep=}\StringTok{""}\NormalTok{)}
\NormalTok{      \} }\ControlFlowTok{else}\NormalTok{ \{}
        \FunctionTok{cat}\NormalTok{(}\StringTok{"Found "}\NormalTok{,}\FunctionTok{nrow}\NormalTok{(out\_of\_range),}\StringTok{" out{-}of{-}range values:}\SpecialCharTok{\textbackslash{}n}\StringTok{"}\NormalTok{,}\AttributeTok{sep=}\StringTok{""}\NormalTok{)}
        \FunctionTok{glimpse}\NormalTok{(out\_of\_range)}
\NormalTok{      \}}
      \FunctionTok{rm}\NormalTok{(out\_of\_range)}
\NormalTok{    \}}
\NormalTok{  \}}
\FunctionTok{rm}\NormalTok{(df, col\_name)}
\NormalTok{\}}

\CommentTok{\# Minute summation}
\NormalTok{valid\_df\_names }\OtherTok{\textless{}{-}} \FunctionTok{c}\NormalTok{(}
  \StringTok{"activity\_sum\_days\_wide"}\NormalTok{,}
  \StringTok{"intensity\_sum\_days\_wide"}\NormalTok{,}
  \StringTok{"sleep\_sum\_days\_wide"}\NormalTok{)}
\NormalTok{valid\_col\_names }\OtherTok{\textless{}{-}} \FunctionTok{c}\NormalTok{(}
  \StringTok{"minutes\_very\_active"}\NormalTok{,}
  \StringTok{"minutes\_fairly\_active"}\NormalTok{,}
  \StringTok{"minutes\_lightly\_active"}\NormalTok{,}
  \StringTok{"minutes\_sedentary"}\NormalTok{,}
  \StringTok{"minutes\_asleep\_total"}\NormalTok{,}
  \StringTok{"minutes\_in\_bed\_total"}\NormalTok{)}
\NormalTok{range\_max }\OtherTok{\textless{}{-}} \DecValTok{60} \SpecialCharTok{*} \DecValTok{12} \CommentTok{\# Minutes in half a day}
\NormalTok{result }\OtherTok{\textless{}{-}} \FunctionTok{map}\NormalTok{(valid\_df\_names, }\SpecialCharTok{\textasciitilde{}}\FunctionTok{validate\_within\_range}\NormalTok{(.x, valid\_col\_names, }\DecValTok{0}\NormalTok{, range\_max))}
\end{Highlighting}
\end{Shaded}

\begin{verbatim}
## Checking ranges for activity_sum_days_wide[minutes_very_active]... complete.
## Checking ranges for activity_sum_days_wide[minutes_fairly_active]... complete.
## Checking ranges for activity_sum_days_wide[minutes_lightly_active]... complete.
## Checking ranges for activity_sum_days_wide[minutes_sedentary]... Found 723 out-of-range values:
## Rows: 723
## Columns: 15
## $ id                         <chr> "1503960366", "1503960366", "1503960366", "~
## $ activity_day               <date> 2016-04-12, 2016-04-13, 2016-04-14, 2016-0~
## $ steps_total                <dbl> 13162, 10735, 10460, 9762, 12669, 13019, 15~
## $ distance_total             <dbl> 8.50, 6.97, 6.74, 6.28, 8.16, 8.59, 9.88, 6~
## $ distance_tracker           <dbl> 8.50, 6.97, 6.74, 6.28, 8.16, 8.59, 9.88, 6~
## $ distance_logged_activities <dbl> 0, 0, 0, 0, 0, 0, 0, 0, 0, 0, 0, 0, 0, 0, 0~
## $ distance_very_active       <dbl> 1.88, 1.57, 2.44, 2.14, 2.71, 3.25, 3.53, 1~
## $ distance_moderately_active <dbl> 0.55, 0.69, 0.40, 1.26, 0.41, 0.64, 1.32, 0~
## $ distance_lightly_active    <dbl> 6.06, 4.71, 3.91, 2.83, 5.04, 4.71, 5.03, 4~
## $ distance_sedentary         <dbl> 0, 0, 0, 0, 0, 0, 0, 0, 0, 0, 0, 0, 0, 0, 0~
## $ minutes_very_active        <dbl> 25, 21, 30, 29, 36, 42, 50, 28, 19, 66, 41,~
## $ minutes_fairly_active      <dbl> 13, 19, 11, 34, 10, 16, 31, 12, 8, 27, 21, ~
## $ minutes_lightly_active     <dbl> 328, 217, 181, 209, 221, 233, 264, 205, 211~
## $ minutes_sedentary          <dbl> 728, 776, 1218, 726, 773, 1149, 775, 818, 8~
## $ calories                   <dbl> 1985, 1797, 1776, 1745, 1863, 1921, 2035, 1~
## Checking ranges for activity_sum_days_wide[minutes_asleep_total]... column not found.
## Checking ranges for activity_sum_days_wide[minutes_in_bed_total]... column not found.
## Checking ranges for intensity_sum_days_wide[minutes_very_active]... complete.
## Checking ranges for intensity_sum_days_wide[minutes_fairly_active]... complete.
## Checking ranges for intensity_sum_days_wide[minutes_lightly_active]... complete.
## Checking ranges for intensity_sum_days_wide[minutes_sedentary]... Found 723 out-of-range values:
## Rows: 723
## Columns: 10
## $ id                         <chr> "1503960366", "1503960366", "1503960366", "~
## $ activity_day               <date> 2016-04-12, 2016-04-13, 2016-04-14, 2016-0~
## $ minutes_sedentary          <dbl> 728, 776, 1218, 726, 773, 1149, 775, 818, 8~
## $ minutes_lightly_active     <dbl> 328, 217, 181, 209, 221, 233, 264, 205, 211~
## $ minutes_fairly_active      <dbl> 13, 19, 11, 34, 10, 16, 31, 12, 8, 27, 21, ~
## $ minutes_very_active        <dbl> 25, 21, 30, 29, 36, 42, 50, 28, 19, 66, 41,~
## $ distance_sedentary         <dbl> 0, 0, 0, 0, 0, 0, 0, 0, 0, 0, 0, 0, 0, 0, 0~
## $ distance_lightly_active    <dbl> 6.06, 4.71, 3.91, 2.83, 5.04, 4.71, 5.03, 4~
## $ distance_moderately_active <dbl> 0.55, 0.69, 0.40, 1.26, 0.41, 0.64, 1.32, 0~
## $ distance_very_active       <dbl> 1.88, 1.57, 2.44, 2.14, 2.71, 3.25, 3.53, 1~
## Checking ranges for intensity_sum_days_wide[minutes_asleep_total]... column not found.
## Checking ranges for intensity_sum_days_wide[minutes_in_bed_total]... column not found.
## Checking ranges for sleep_sum_days_wide[minutes_very_active]... column not found.
## Checking ranges for sleep_sum_days_wide[minutes_fairly_active]... column not found.
## Checking ranges for sleep_sum_days_wide[minutes_lightly_active]... column not found.
## Checking ranges for sleep_sum_days_wide[minutes_sedentary]... column not found.
## Checking ranges for sleep_sum_days_wide[minutes_asleep_total]... Found 4 out-of-range values:
## Rows: 4
## Columns: 5
## $ id                   <chr> "1644430081", "1844505072", "1927972279", "555395~
## $ sleep_day            <dttm> 2016-05-02, 2016-04-30, 2016-04-12, 2016-04-30
## $ sleep_records_total  <dbl> 1, 1, 3, 2
## $ minutes_asleep_total <dbl> 796, 722, 750, 775
## $ minutes_in_bed_total <dbl> 961, 961, 775, 843
## Checking ranges for sleep_sum_days_wide[minutes_in_bed_total]... Found 8 out-of-range values:
## Rows: 8
## Columns: 5
## $ id                   <chr> "1644430081", "1844505072", "1844505072", "184450~
## $ sleep_day            <dttm> 2016-05-02, 2016-04-15, 2016-04-30, 2016-05-01, ~
## $ sleep_records_total  <dbl> 1, 1, 1, 1, 3, 1, 2, 2
## $ minutes_asleep_total <dbl> 796, 644, 722, 590, 750, 692, 631, 775
## $ minutes_in_bed_total <dbl> 961, 961, 961, 961, 775, 722, 725, 843
\end{verbatim}

\begin{Shaded}
\begin{Highlighting}[]
\CommentTok{\# Weights (kg)}
\NormalTok{valid\_df\_names }\OtherTok{\textless{}{-}} \FunctionTok{c}\NormalTok{(}\StringTok{"bodycomp\_src\_logs\_wide"}\NormalTok{)}
\NormalTok{valid\_col\_names }\OtherTok{\textless{}{-}} \FunctionTok{c}\NormalTok{(}\StringTok{"weight\_kg"}\NormalTok{)}
\NormalTok{range\_max }\OtherTok{\textless{}{-}} \DecValTok{150} \CommentTok{\# Arbitrarily chosen as high enough to be a potentially erroneous value}
\NormalTok{result }\OtherTok{\textless{}{-}} \FunctionTok{map}\NormalTok{(valid\_df\_names, }\SpecialCharTok{\textasciitilde{}}\FunctionTok{validate\_within\_range}\NormalTok{(.x, valid\_col\_names, }\DecValTok{0}\NormalTok{, range\_max))}
\end{Highlighting}
\end{Shaded}

\begin{verbatim}
## Checking ranges for bodycomp_src_logs_wide[weight_kg]... complete.
\end{verbatim}

\begin{Shaded}
\begin{Highlighting}[]
\CommentTok{\# Weights (pounds, same limit as for weight in kilos)}
\NormalTok{valid\_df\_names }\OtherTok{\textless{}{-}} \FunctionTok{c}\NormalTok{(}\StringTok{"bodycomp\_src\_logs\_wide"}\NormalTok{)}
\NormalTok{valid\_col\_names }\OtherTok{\textless{}{-}} \FunctionTok{c}\NormalTok{(}\StringTok{"weight\_pounds"}\NormalTok{)}
\NormalTok{kg2lb }\OtherTok{\textless{}{-}} \FloatTok{2.204623}
\NormalTok{range\_max }\OtherTok{\textless{}{-}} \DecValTok{150} \SpecialCharTok{*}\NormalTok{ kg2lb }\CommentTok{\# Arbitrarily chosen as high enough to be a potentially erroneous value}
\NormalTok{result }\OtherTok{\textless{}{-}} \FunctionTok{map}\NormalTok{(valid\_df\_names, }\SpecialCharTok{\textasciitilde{}}\FunctionTok{validate\_within\_range}\NormalTok{(.x, valid\_col\_names, }\DecValTok{0}\NormalTok{, range\_max))}
\end{Highlighting}
\end{Shaded}

\begin{verbatim}
## Checking ranges for bodycomp_src_logs_wide[weight_pounds]... complete.
\end{verbatim}

\begin{Shaded}
\begin{Highlighting}[]
\FunctionTok{rm}\NormalTok{(kg2lb)}

\CommentTok{\# BMI }
\NormalTok{valid\_df\_names }\OtherTok{\textless{}{-}} \FunctionTok{c}\NormalTok{(}\StringTok{"bodycomp\_src\_logs\_wide"}\NormalTok{)}
\NormalTok{valid\_col\_names }\OtherTok{\textless{}{-}} \FunctionTok{c}\NormalTok{(}\StringTok{"bmi"}\NormalTok{)}
\NormalTok{range\_max }\OtherTok{\textless{}{-}} \FloatTok{40.0} \CommentTok{\# Corresponds with the WHO "Obese (Class III)" weight category}
\NormalTok{result }\OtherTok{\textless{}{-}} \FunctionTok{map}\NormalTok{(valid\_df\_names, }\SpecialCharTok{\textasciitilde{}}\FunctionTok{validate\_within\_range}\NormalTok{(.x, valid\_col\_names, }\DecValTok{0}\NormalTok{, range\_max))}
\end{Highlighting}
\end{Shaded}

\begin{verbatim}
## Checking ranges for bodycomp_src_logs_wide[bmi]... Found 1 out-of-range values:
## Rows: 1
## Columns: 8
## $ id                <chr> "1927972279"
## $ bodycomp_datetime <dttm> 2016-04-13 01:08:52
## $ weight_kg         <dbl> 133.5
## $ weight_pounds     <dbl> 294.3171
## $ fat               <dbl> NA
## $ bmi               <dbl> 47.54
## $ is_manual_report  <lgl> FALSE
## $ bodycomp_log_id   <chr> "1460509732000"
\end{verbatim}

\begin{Shaded}
\begin{Highlighting}[]
\CommentTok{\# Distances}
\NormalTok{valid\_df\_names }\OtherTok{\textless{}{-}} \FunctionTok{c}\NormalTok{(}
  \StringTok{"activity\_sum\_days\_wide"}\NormalTok{,}
  \StringTok{"intensity\_sum\_days\_wide"}\NormalTok{)}
\NormalTok{valid\_col\_names }\OtherTok{\textless{}{-}} \FunctionTok{c}\NormalTok{(}
  \StringTok{"distance\_lightly\_active"}\NormalTok{,}
  \StringTok{"distance\_logged\_activities"}\NormalTok{,}
  \StringTok{"distance\_moderately\_active"}\NormalTok{,}
  \StringTok{"distance\_sedentary"}\NormalTok{,}
  \StringTok{"distance\_total"}\NormalTok{,}
  \StringTok{"distance\_tracker"}\NormalTok{,}
  \StringTok{"distance\_very\_active"}\NormalTok{)}
\NormalTok{range\_max }\OtherTok{\textless{}{-}} \FloatTok{21.08241} \CommentTok{\# Equivalent to one half{-}marathon}
\NormalTok{result }\OtherTok{\textless{}{-}} \FunctionTok{map}\NormalTok{(valid\_df\_names, }\SpecialCharTok{\textasciitilde{}}\FunctionTok{validate\_within\_range}\NormalTok{(.x, valid\_col\_names, }\DecValTok{0}\NormalTok{, range\_max))}
\end{Highlighting}
\end{Shaded}

\begin{verbatim}
## Checking ranges for activity_sum_days_wide[distance_lightly_active]... complete.
## Checking ranges for activity_sum_days_wide[distance_logged_activities]... complete.
## Checking ranges for activity_sum_days_wide[distance_moderately_active]... complete.
## Checking ranges for activity_sum_days_wide[distance_sedentary]... complete.
## Checking ranges for activity_sum_days_wide[distance_total]... Found 3 out-of-range values:
## Rows: 3
## Columns: 15
## $ id                         <chr> "1624580081", "8877689391", "8877689391"
## $ activity_day               <date> 2016-05-01, 2016-04-16, 2016-04-30
## $ steps_total                <dbl> 36019, 29326, 27745
## $ distance_total             <dbl> 28.03, 25.29, 26.72
## $ distance_tracker           <dbl> 28.03, 25.29, 26.72
## $ distance_logged_activities <dbl> 0, 0, 0
## $ distance_very_active       <dbl> 21.92, 13.24, 21.66
## $ distance_moderately_active <dbl> 4.19, 1.21, 0.08
## $ distance_lightly_active    <dbl> 1.91, 10.71, 4.93
## $ distance_sedentary         <dbl> 0.02, 0.00, 0.00
## $ minutes_very_active        <dbl> 186, 94, 124
## $ minutes_fairly_active      <dbl> 63, 29, 4
## $ minutes_lightly_active     <dbl> 171, 429, 223
## $ minutes_sedentary          <dbl> 1020, 888, 1089
## $ calories                   <dbl> 2690, 4547, 4398
## Checking ranges for activity_sum_days_wide[distance_tracker]... Found 3 out-of-range values:
## Rows: 3
## Columns: 15
## $ id                         <chr> "1624580081", "8877689391", "8877689391"
## $ activity_day               <date> 2016-05-01, 2016-04-16, 2016-04-30
## $ steps_total                <dbl> 36019, 29326, 27745
## $ distance_total             <dbl> 28.03, 25.29, 26.72
## $ distance_tracker           <dbl> 28.03, 25.29, 26.72
## $ distance_logged_activities <dbl> 0, 0, 0
## $ distance_very_active       <dbl> 21.92, 13.24, 21.66
## $ distance_moderately_active <dbl> 4.19, 1.21, 0.08
## $ distance_lightly_active    <dbl> 1.91, 10.71, 4.93
## $ distance_sedentary         <dbl> 0.02, 0.00, 0.00
## $ minutes_very_active        <dbl> 186, 94, 124
## $ minutes_fairly_active      <dbl> 63, 29, 4
## $ minutes_lightly_active     <dbl> 171, 429, 223
## $ minutes_sedentary          <dbl> 1020, 888, 1089
## $ calories                   <dbl> 2690, 4547, 4398
## Checking ranges for activity_sum_days_wide[distance_very_active]... Found 2 out-of-range values:
## Rows: 2
## Columns: 15
## $ id                         <chr> "1624580081", "8877689391"
## $ activity_day               <date> 2016-05-01, 2016-04-30
## $ steps_total                <dbl> 36019, 27745
## $ distance_total             <dbl> 28.03, 26.72
## $ distance_tracker           <dbl> 28.03, 26.72
## $ distance_logged_activities <dbl> 0, 0
## $ distance_very_active       <dbl> 21.92, 21.66
## $ distance_moderately_active <dbl> 4.19, 0.08
## $ distance_lightly_active    <dbl> 1.91, 4.93
## $ distance_sedentary         <dbl> 0.02, 0.00
## $ minutes_very_active        <dbl> 186, 124
## $ minutes_fairly_active      <dbl> 63, 4
## $ minutes_lightly_active     <dbl> 171, 223
## $ minutes_sedentary          <dbl> 1020, 1089
## $ calories                   <dbl> 2690, 4398
## Checking ranges for intensity_sum_days_wide[distance_lightly_active]... complete.
## Checking ranges for intensity_sum_days_wide[distance_logged_activities]... column not found.
## Checking ranges for intensity_sum_days_wide[distance_moderately_active]... complete.
## Checking ranges for intensity_sum_days_wide[distance_sedentary]... complete.
## Checking ranges for intensity_sum_days_wide[distance_total]... column not found.
## Checking ranges for intensity_sum_days_wide[distance_tracker]... column not found.
## Checking ranges for intensity_sum_days_wide[distance_very_active]... Found 2 out-of-range values:
## Rows: 2
## Columns: 10
## $ id                         <chr> "1624580081", "8877689391"
## $ activity_day               <date> 2016-05-01, 2016-04-30
## $ minutes_sedentary          <dbl> 1020, 1089
## $ minutes_lightly_active     <dbl> 171, 223
## $ minutes_fairly_active      <dbl> 63, 4
## $ minutes_very_active        <dbl> 186, 124
## $ distance_sedentary         <dbl> 0.02, 0.00
## $ distance_lightly_active    <dbl> 1.91, 4.93
## $ distance_moderately_active <dbl> 4.19, 0.08
## $ distance_very_active       <dbl> 21.92, 21.66
\end{verbatim}

\begin{Shaded}
\begin{Highlighting}[]
\CommentTok{\# Step counts}
\NormalTok{valid\_df\_names }\OtherTok{\textless{}{-}} \FunctionTok{c}\NormalTok{(}
  \StringTok{"activity\_sum\_days\_wide"}\NormalTok{,}
  \StringTok{"steps\_sum\_days\_tall"}\NormalTok{,}
  \StringTok{"steps\_sum\_hours\_tall"}\NormalTok{,}
  \StringTok{"steps\_src\_mins\_tall"}\NormalTok{)}
\NormalTok{valid\_col\_names }\OtherTok{\textless{}{-}} \FunctionTok{c}\NormalTok{(}
  \StringTok{"steps"}\NormalTok{,}
  \StringTok{"steps\_total"}\NormalTok{)}
\NormalTok{range\_max }\OtherTok{\textless{}{-}} \DecValTok{14800} \CommentTok{\# Set to double the average daily step count for Australian adults}
\NormalTok{result }\OtherTok{\textless{}{-}} \FunctionTok{map}\NormalTok{(valid\_df\_names, }\SpecialCharTok{\textasciitilde{}}\FunctionTok{validate\_within\_range}\NormalTok{(.x, valid\_col\_names, }\DecValTok{0}\NormalTok{, range\_max))}
\end{Highlighting}
\end{Shaded}

\begin{verbatim}
## Checking ranges for activity_sum_days_wide[steps]... column not found.
## Checking ranges for activity_sum_days_wide[steps_total]... Found 71 out-of-range values:
## Rows: 71
## Columns: 15
## $ id                         <chr> "1503960366", "1503960366", "1503960366", "~
## $ activity_day               <date> 2016-04-19, 2016-04-25, 2016-04-27, 2016-0~
## $ steps_total                <dbl> 15506, 15355, 18134, 15103, 36019, 15300, 1~
## $ distance_total             <dbl> 9.88, 9.80, 12.21, 9.66, 28.03, 11.12, 13.2~
## $ distance_tracker           <dbl> 9.88, 9.80, 12.21, 9.66, 28.03, 11.12, 13.2~
## $ distance_logged_activities <dbl> 0.00000, 0.00000, 0.00000, 0.00000, 0.00000~
## $ distance_very_active       <dbl> 3.53, 5.29, 6.40, 3.73, 21.92, 4.10, 0.63, ~
## $ distance_moderately_active <dbl> 1.32, 0.57, 0.41, 1.05, 4.19, 1.88, 3.14, 0~
## $ distance_lightly_active    <dbl> 5.03, 3.94, 5.41, 4.88, 1.91, 5.09, 9.46, 6~
## $ distance_sedentary         <dbl> 0.00, 0.00, 0.00, 0.00, 0.02, 0.00, 0.00, 0~
## $ minutes_very_active        <dbl> 50, 73, 78, 50, 186, 51, 9, 28, 48, 13, 66,~
## $ minutes_fairly_active      <dbl> 31, 14, 11, 24, 63, 42, 71, 29, 63, 23, 72,~
## $ minutes_lightly_active     <dbl> 264, 216, 243, 254, 171, 212, 402, 331, 276~
## $ minutes_sedentary          <dbl> 775, 814, 1108, 816, 1020, 1135, 816, 1052,~
## $ calories                   <dbl> 2035, 2013, 2159, 1990, 2690, 3493, 3846, 2~
## Checking ranges for steps_sum_days_tall[steps]... column not found.
## Checking ranges for steps_sum_days_tall[steps_total]... Found 71 out-of-range values:
## Rows: 71
## Columns: 3
## $ id           <chr> "1503960366", "1503960366", "1503960366", "1503960366", "~
## $ activity_day <date> 2016-04-19, 2016-04-25, 2016-04-27, 2016-05-03, 2016-05-~
## $ steps_total  <dbl> 15506, 15355, 18134, 15103, 36019, 15300, 18213, 15112, 1~
## Checking ranges for steps_sum_hours_tall[steps]... column not found.
## Checking ranges for steps_sum_hours_tall[steps_total]... complete.
## Checking ranges for steps_src_mins_tall[steps]... complete.
## Checking ranges for steps_src_mins_tall[steps_total]... column not found.
\end{verbatim}

\begin{Shaded}
\begin{Highlighting}[]
\CommentTok{\# Calories}
\NormalTok{valid\_df\_names }\OtherTok{\textless{}{-}} \FunctionTok{c}\NormalTok{(}
  \StringTok{"activity\_sum\_days\_wide"}\NormalTok{,}
  \StringTok{"calories\_src\_mins\_tall"}\NormalTok{,}
  \StringTok{"calories\_sum\_days\_tall"}\NormalTok{,}
  \StringTok{"calories\_sum\_hours\_tall"}\NormalTok{)}
\NormalTok{valid\_col\_names }\OtherTok{\textless{}{-}} \FunctionTok{c}\NormalTok{(}\StringTok{"calories"}\NormalTok{)}
\NormalTok{range\_max }\OtherTok{\textless{}{-}} \DecValTok{4000} \CommentTok{\# Chosen arbitrarily as double the typically{-}recommended daily caloric intake}
\NormalTok{result }\OtherTok{\textless{}{-}} \FunctionTok{map}\NormalTok{(valid\_df\_names, }\SpecialCharTok{\textasciitilde{}}\FunctionTok{validate\_within\_range}\NormalTok{(.x, valid\_col\_names, }\DecValTok{0}\NormalTok{, range\_max))}
\end{Highlighting}
\end{Shaded}

\begin{verbatim}
## Checking ranges for activity_sum_days_wide[calories]... Found 21 out-of-range values:
## Rows: 21
## Columns: 15
## $ id                         <chr> "4388161847", "5577150313", "5577150313", "~
## $ activity_day               <date> 2016-05-07, 2016-04-14, 2016-04-15, 2016-0~
## $ steps_total                <dbl> 22770, 8596, 12087, 14269, 12231, 10830, 15~
## $ distance_total             <dbl> 17.54, 6.42, 9.08, 10.66, 9.14, 8.09, 11.78~
## $ distance_tracker           <dbl> 17.54, 6.42, 9.08, 10.66, 9.14, 8.09, 11.78~
## $ distance_logged_activities <dbl> 0.000000, 0.000000, 0.000000, 0.000000, 0.0~
## $ distance_very_active       <dbl> 9.45, 3.33, 3.92, 6.64, 5.98, 3.65, 7.65, 5~
## $ distance_moderately_active <dbl> 2.77, 0.31, 1.60, 1.28, 0.83, 1.66, 2.15, 0~
## $ distance_lightly_active    <dbl> 5.33, 2.78, 3.56, 2.73, 2.32, 2.78, 1.98, 2~
## $ distance_sedentary         <dbl> 0, 0, 0, 0, 0, 0, 0, 0, 0, 0, 0, 0, 0, 0, 0~
## $ minutes_very_active        <dbl> 120, 118, 115, 184, 200, 110, 210, 207, 194~
## $ minutes_fairly_active      <dbl> 56, 30, 54, 56, 37, 74, 65, 45, 72, 19, 14,~
## $ minutes_lightly_active     <dbl> 260, 176, 199, 158, 159, 175, 141, 163, 178~
## $ minutes_sedentary          <dbl> 508, 662, 695, 472, 525, 670, 425, 621, 499~
## $ calories                   <dbl> 4022, 4022, 4005, 4274, 4552, 4018, 4392, 4~
## Checking ranges for calories_src_mins_tall[calories]... complete.
## Checking ranges for calories_sum_days_tall[calories]... Found 21 out-of-range values:
## Rows: 21
## Columns: 3
## $ id           <chr> "4388161847", "5577150313", "5577150313", "5577150313", "~
## $ activity_day <date> 2016-05-07, 2016-04-14, 2016-04-15, 2016-04-16, 2016-04-~
## $ calories     <dbl> 4022, 4022, 4005, 4274, 4552, 4018, 4392, 4501, 4546, 490~
## Checking ranges for calories_sum_hours_tall[calories]... complete.
\end{verbatim}

\begin{Shaded}
\begin{Highlighting}[]
\CommentTok{\# }\AlertTok{TODO}\CommentTok{: METs}
\NormalTok{valid\_df\_names }\OtherTok{\textless{}{-}} \FunctionTok{c}\NormalTok{(}\StringTok{"mets\_src\_mins\_tall"}\NormalTok{)}
\NormalTok{valid\_col\_names }\OtherTok{\textless{}{-}} \FunctionTok{c}\NormalTok{(}\StringTok{"mets"}\NormalTok{)}
\NormalTok{range\_max }\OtherTok{\textless{}{-}} \DecValTok{12} \CommentTok{\# Equivalent to vigourous squash playing}
\NormalTok{result }\OtherTok{\textless{}{-}} \FunctionTok{map}\NormalTok{(valid\_df\_names, }\SpecialCharTok{\textasciitilde{}}\FunctionTok{validate\_within\_range}\NormalTok{(.x, valid\_col\_names, }\DecValTok{0}\NormalTok{, range\_max))}
\end{Highlighting}
\end{Shaded}

\begin{verbatim}
## Checking ranges for mets_src_mins_tall[mets]... Found 250847 out-of-range values:
## Rows: 250,847
## Columns: 3
## $ id              <chr> "1503960366", "1503960366", "1503960366", "1503960366"~
## $ activity_minute <dttm> 2016-04-12 00:25:00, 2016-04-12 00:29:00, 2016-04-12 ~
## $ mets            <dbl> 26, 32, 36, 32, 26, 32, 30, 28, 32, 28, 26, 26, 34, 28~
\end{verbatim}

\begin{Shaded}
\begin{Highlighting}[]
\CommentTok{\# Heart Rates}
\NormalTok{valid\_df\_names }\OtherTok{\textless{}{-}} \FunctionTok{c}\NormalTok{(}\StringTok{"heartrate\_src\_seconds\_tall"}\NormalTok{)}
\NormalTok{valid\_col\_names }\OtherTok{\textless{}{-}} \FunctionTok{c}\NormalTok{(}\StringTok{"heart\_rate"}\NormalTok{)}
\NormalTok{range\_max  }\OtherTok{\textless{}{-}} \DecValTok{200} \CommentTok{\# Chosen based on average 100\% heart{-}rate for a 20 y.o. (Source: American Heart Foundation)}
\NormalTok{result }\OtherTok{\textless{}{-}} \FunctionTok{map}\NormalTok{(valid\_df\_names, }\SpecialCharTok{\textasciitilde{}}\FunctionTok{validate\_within\_range}\NormalTok{(.x, valid\_col\_names, }\DecValTok{0}\NormalTok{, range\_max))}
\end{Highlighting}
\end{Shaded}

\begin{verbatim}
## Checking ranges for heartrate_src_seconds_tall[heart_rate]... Found 13 out-of-range values:
## Rows: 13
## Columns: 3
## $ id                <chr> "2022484408", "2022484408", "2022484408", "202248440~
## $ heart_rate_second <dttm> 2016-04-21 16:31:30, 2016-04-21 16:31:40, 2016-04-2~
## $ heart_rate        <dbl> 202, 203, 202, 203, 203, 203, 203, 201, 202, 203, 20~
\end{verbatim}

\begin{Shaded}
\begin{Highlighting}[]
\FunctionTok{rm}\NormalTok{(range\_max, result, valid\_df\_names, valid\_col\_names)}
\end{Highlighting}
\end{Shaded}

Results:

\begin{itemize}
\tightlist
\item
  There were no invalid negative values
\item
  There were no invalid percentage values
\item
  There were no invalid minute summations
\item
  There were 19 step count data points above 20,000: these were manually
  checked and found to be associated with long distances covered and
  high levels of exercise, which made sense, so I considered that data
  valid.
\item
  There were a total of four unique data points with a distance covered
  of more than one half-marathon: all of these also indicatd a very high
  level of exercise for at lest 90 minutes, which made sense, so I
  considered the data valid
\item
  Calories: 21 records showed caloric burns of over 4000 calories. All
  but one of those were associated with very high levels of activity:
  the outlier showed only 30 minutes total as ``Very Active'' or
  ``Lightly Active'', compared to 120+ minutes for all other records. It
  does show 15km distance covered that day, which is nearly a third of a
  marathon, so for now it makes enough sense to keep it in, and I will
  analyse it further after the cleaning stage.
\item
  Heart-rate: 13 records were found with a heart-rate between 200 and
  203, all of which corresponded to a single user over a 30-minute
  period of ``Very Active'' intensity.
\item
  METs: Every value was out of range for the original limit of 12 METs,
  chosen as it was the highest listed MET value for any activity in the
  source data. The smallest repeated value in the data was 10. This
  suggests my interpretation of the data logged for this variable is
  wrong.
\end{itemize}

\hypertarget{validating-the-mets-data}{%
\paragraph{Validating the METs data}\label{validating-the-mets-data}}

\begin{itemize}
\tightlist
\item
  In theory, METs are a rate of energy expenditure: if you expend energy
  at a rate of 3 METs, that value doesn't change whether you maintain it
  for a minute or an hour
\item
  The highest MET value for any given activity in the sources I used was
  12, for sustained ``heavy'' squash playing
\item
  My initial assumption was that the MET values in the data were the
  average MET rate for the sampling period, in this case one minute
\item
  The data points, however, were listing MET values as high as 157
\item
  This suggested the values were not averages but rather some sum or
  projected sum, e.g.~projected MET-hours based on the average over the
  minute.
\item
  Searching through the product manuals again revealed no information on
  METs at all. A
  \href{https://help.fitbit.com/articles/en_US/Help_article/1379.htm}{Help
  article} was the only other official reference to METs that I could
  find: it implies that METs are used to calculate ``Active Minutes'',
  and that Active Minutes count double in higher heart-rate zones, but
  no specific mathematical relationship between the two was described.
\end{itemize}

With this in mind, I attempted to corroborate the METs data by plotting
it against other related data, specifically calories-burned and
heart-rate. Individual charts were plotted for each user ID.

\begin{Shaded}
\begin{Highlighting}[]
\CommentTok{\# Generate minute{-}level summary heart{-}rate data for merging with calorie/}
\NormalTok{heartrate\_sum\_minutes\_tall }\OtherTok{\textless{}{-}}\NormalTok{ heartrate\_src\_seconds\_tall }\SpecialCharTok{\%\textgreater{}\%} 
  \FunctionTok{mutate}\NormalTok{(}\AttributeTok{activity\_minute =} \FunctionTok{floor\_date}\NormalTok{(heart\_rate\_second, }\AttributeTok{unit =} \StringTok{"minute"}\NormalTok{)) }\SpecialCharTok{\%\textgreater{}\%}
  \FunctionTok{group\_by}\NormalTok{(id, activity\_minute) }\SpecialCharTok{\%\textgreater{}\%}
  \FunctionTok{summarize}\NormalTok{(}\AttributeTok{heart\_rate\_mean =} \FunctionTok{mean}\NormalTok{(heart\_rate))}

\NormalTok{mets\_vs\_heart\_rate }\OtherTok{\textless{}{-}}\NormalTok{ mets\_src\_mins\_tall }\SpecialCharTok{\%\textgreater{}\%}
  \FunctionTok{merge}\NormalTok{(., intensity\_src\_mins\_tall, }\AttributeTok{by =} \FunctionTok{c}\NormalTok{(}\StringTok{"id"}\NormalTok{, }\StringTok{"activity\_minute"}\NormalTok{)) }\SpecialCharTok{\%\textgreater{}\%}
  \FunctionTok{merge}\NormalTok{(., heartrate\_sum\_minutes\_tall, }\AttributeTok{by =} \FunctionTok{c}\NormalTok{(}\StringTok{"id"}\NormalTok{, }\StringTok{"activity\_minute"}\NormalTok{))}

\NormalTok{mets\_vs\_calories }\OtherTok{\textless{}{-}}\NormalTok{ mets\_src\_mins\_tall }\SpecialCharTok{\%\textgreater{}\%}
  \FunctionTok{merge}\NormalTok{(., intensity\_src\_mins\_tall, }\AttributeTok{by =} \FunctionTok{c}\NormalTok{(}\StringTok{"id"}\NormalTok{, }\StringTok{"activity\_minute"}\NormalTok{)) }\SpecialCharTok{\%\textgreater{}\%}
  \FunctionTok{merge}\NormalTok{(., calories\_src\_mins\_tall, }\AttributeTok{by =} \FunctionTok{c}\NormalTok{(}\StringTok{"id"}\NormalTok{, }\StringTok{"activity\_minute"}\NormalTok{))}

\NormalTok{plot\_mets\_vs\_heartrate }\OtherTok{\textless{}{-}} \FunctionTok{ggplot}\NormalTok{(}\AttributeTok{data =}\NormalTok{ mets\_vs\_heart\_rate) }\SpecialCharTok{+}
  \FunctionTok{geom\_point}\NormalTok{(}\AttributeTok{mapping=}\FunctionTok{aes}\NormalTok{(}\AttributeTok{x=}\NormalTok{heart\_rate\_mean, }\AttributeTok{y=}\NormalTok{mets, }\AttributeTok{shape=}\NormalTok{intensity, }\AttributeTok{color=}\NormalTok{intensity)) }\SpecialCharTok{+}
  \FunctionTok{geom\_smooth}\NormalTok{(}\AttributeTok{mapping=}\FunctionTok{aes}\NormalTok{(}\AttributeTok{x=}\NormalTok{heart\_rate\_mean, }\AttributeTok{y=}\NormalTok{mets), }\AttributeTok{method =} \StringTok{"lm"}\NormalTok{, }\AttributeTok{se =} \ConstantTok{FALSE}\NormalTok{, }\AttributeTok{color =} \StringTok{"blue"}\NormalTok{) }\SpecialCharTok{+}
  \FunctionTok{stat\_cor}\NormalTok{(}\FunctionTok{aes}\NormalTok{(}\AttributeTok{x=}\NormalTok{heart\_rate\_mean, }\AttributeTok{y=}\NormalTok{mets, }\AttributeTok{label=}\NormalTok{..rr.label..), }\AttributeTok{label.x=}\DecValTok{0}\NormalTok{, }\AttributeTok{label.y=}\DecValTok{50}\NormalTok{) }\SpecialCharTok{+}
  \FunctionTok{facet\_wrap}\NormalTok{(}\FunctionTok{vars}\NormalTok{(id))}
\FunctionTok{print}\NormalTok{(plot\_mets\_vs\_heartrate)}

\NormalTok{plot\_mets\_vs\_calories }\OtherTok{\textless{}{-}} \FunctionTok{ggplot}\NormalTok{(}\AttributeTok{data =}\NormalTok{ mets\_vs\_calories) }\SpecialCharTok{+}
  \FunctionTok{geom\_point}\NormalTok{(}\AttributeTok{mapping=}\FunctionTok{aes}\NormalTok{(}\AttributeTok{x=}\NormalTok{calories,}\AttributeTok{y=}\NormalTok{mets, }\AttributeTok{shape=}\NormalTok{intensity, }\AttributeTok{color=}\NormalTok{intensity)) }\SpecialCharTok{+}
  \CommentTok{\# geom\_smooth(mapping=aes(x=calories,y=mets), method = "lm", se = FALSE, color = "blue") +}
  \FunctionTok{stat\_cor}\NormalTok{(}\FunctionTok{aes}\NormalTok{(}\AttributeTok{x=}\NormalTok{calories, }\AttributeTok{y=}\NormalTok{mets, }\AttributeTok{label=}\NormalTok{..rr.label..), }\AttributeTok{label.x=}\DecValTok{0}\NormalTok{, }\AttributeTok{label.y=}\DecValTok{50}\NormalTok{) }\SpecialCharTok{+}
  \FunctionTok{facet\_wrap}\NormalTok{(}\FunctionTok{vars}\NormalTok{(id))}
\FunctionTok{print}\NormalTok{(plot\_mets\_vs\_calories)}

\NormalTok{plot\_mets\_vs\_intensity }\OtherTok{\textless{}{-}} \FunctionTok{ggplot}\NormalTok{(}\AttributeTok{data =}\NormalTok{ mets\_vs\_calories) }\SpecialCharTok{+}
  \FunctionTok{geom\_point}\NormalTok{(}\AttributeTok{mapping=}\FunctionTok{aes}\NormalTok{(}\AttributeTok{x=}\NormalTok{intensity,}\AttributeTok{y=}\NormalTok{mets)) }\SpecialCharTok{+}
  \CommentTok{\# geom\_smooth(mapping=aes(x=calories,y=mets), method = "lm", se = FALSE, color = "blue") +}
  \CommentTok{\#stat\_cor(aes(x=calories, y=mets, label=..rr.label..), label.x=0, label.y=50) +}
  \FunctionTok{facet\_wrap}\NormalTok{(}\FunctionTok{vars}\NormalTok{(id))}
\FunctionTok{print}\NormalTok{(plot\_mets\_vs\_intensity)}
\end{Highlighting}
\end{Shaded}

Results:

\begin{itemize}
\tightlist
\item
  There was a clear positive correlation between heart-rate and METs,
  with R\^{}2 values ranging from 0.43 to 0.82
\item
  A much tighter positive correlation was present between calories and
  METs, with R\^{}2 values of 1 across all user ID values.
\end{itemize}

Given the very tight relationship between METs and calories, and the
unclear sampling method used to generate the MET data, I decided to keep
the METs data as-is, and consider all values as valid for the purposes
of cleaning..After cleaning the data, further analysis of the MET data
may shed more light on its meaning: in the meantime, the calories,
heart-rate, and intensity variables can be used to analyse energy
expenditure and activity type.

\hypertarget{sources-used-to-determine-realistic-limits-for-numerical-values}{%
\paragraph{Sources used to determine realistic limits for numerical
values}\label{sources-used-to-determine-realistic-limits-for-numerical-values}}

Sources for ranges: * BMI: Based on WHO statistical categories for
classifying BMI values:
\url{https://en.wikipedia.org/wiki/Body_mass_index\#Categories} *
Half-marathon: \url{https://en.wikipedia.org/wiki/Half_marathon} *
Step-count:
\url{https://www.abs.gov.au/statistics/health/health-conditions-and-risks/australian-health-survey-physical-activity/latest-release\#pedometer-steps}
* Caloric intake: Chosen arbitrarily as double my own daily required
intake, based on a calculator accessed at:
\url{https://www.eatforhealth.gov.au/nutrition-calculators/daily-energy-requirements-calculator}
* Max HR:
\url{https://www.heart.org/en/healthy-living/fitness/fitness-basics/target-heart-rates}
* METs:
\url{https://www.hsph.harvard.edu/nutritionsource/staying-active/},
\url{https://onlinelibrary.wiley.com/doi/epdf/10.1002/clc.4960130809}

\hypertarget{todo-step-4-analyse}{%
\section{TODO Step 4: Analyse}\label{todo-step-4-analyse}}

\hypertarget{method}{%
\subsection{Method}\label{method}}

Use the advertised features of the Ivy to guide the analysis

\hypertarget{function-declarations}{%
\subsection{Function Declarations}\label{function-declarations}}

These are generic functions I developed for use during the analysis.
They are declared here to be made available to subsequent code chunks.

\begin{Shaded}
\begin{Highlighting}[]
\NormalTok{round\_data\_to\_bin }\OtherTok{\textless{}{-}} \ControlFlowTok{function}\NormalTok{(data, bin\_width) \{}
\NormalTok{  rounding\_factor }\OtherTok{\textless{}{-}} \DecValTok{1} \SpecialCharTok{/}\NormalTok{ bin\_width}
  \FunctionTok{return}\NormalTok{(}\FunctionTok{round}\NormalTok{(data }\SpecialCharTok{*}\NormalTok{ rounding\_factor) }\SpecialCharTok{/}\NormalTok{ rounding\_factor)}
\NormalTok{\}}

\NormalTok{get\_histogram\_max\_count }\OtherTok{\textless{}{-}} \ControlFlowTok{function}\NormalTok{(data, bin\_width) \{}
\NormalTok{  rounding\_factor }\OtherTok{\textless{}{-}} \DecValTok{1} \SpecialCharTok{/}\NormalTok{ bin\_width}
  \CommentTok{\# Round data to nearest multiple of bin\_width}
\NormalTok{  data\_rounded }\OtherTok{\textless{}{-}} \FunctionTok{round\_data\_to\_bin}\NormalTok{(data, bin\_width)}
\NormalTok{  max\_count }\OtherTok{\textless{}{-}} \FunctionTok{max}\NormalTok{(}\FunctionTok{table}\NormalTok{(data\_rounded))}
  \FunctionTok{return}\NormalTok{(max\_count)}
\NormalTok{\}}

\NormalTok{rescale\_plot }\OtherTok{\textless{}{-}} \ControlFlowTok{function}\NormalTok{(p, }\AttributeTok{x\_min =} \DecValTok{0}\NormalTok{, }\AttributeTok{x\_max =} \DecValTok{10}\NormalTok{, }\AttributeTok{x\_step =} \DecValTok{1}\NormalTok{, }\AttributeTok{y\_min =} \DecValTok{0}\NormalTok{, }\AttributeTok{y\_max =} \DecValTok{10}\NormalTok{, }\AttributeTok{y\_step =} \DecValTok{1}\NormalTok{) \{}
\NormalTok{  p }\OtherTok{\textless{}{-}}\NormalTok{ p }\SpecialCharTok{+}
    \FunctionTok{scale\_x\_continuous}\NormalTok{(}\AttributeTok{breaks =} \FunctionTok{seq}\NormalTok{(x\_min, x\_max, }\AttributeTok{by=}\NormalTok{x\_step),}
                       \AttributeTok{labels =}\NormalTok{ scales}\SpecialCharTok{::}\FunctionTok{comma\_format}\NormalTok{(),}
                       \AttributeTok{limits=}\FunctionTok{c}\NormalTok{(x\_min, x\_max)) }\SpecialCharTok{+}
    \FunctionTok{scale\_y\_continuous}\NormalTok{(}\AttributeTok{breaks =} \FunctionTok{seq}\NormalTok{(y\_min, y\_max, }\AttributeTok{by=}\NormalTok{y\_step),}
                       \AttributeTok{labels =}\NormalTok{ scales}\SpecialCharTok{::}\FunctionTok{comma\_format}\NormalTok{(),}
                       \AttributeTok{limits=}\FunctionTok{c}\NormalTok{(y\_min, y\_max))}
  \FunctionTok{return}\NormalTok{(p)}
\NormalTok{\}}
\end{Highlighting}
\end{Shaded}

\begin{Shaded}
\begin{Highlighting}[]
\NormalTok{plot\_histo\_pareto }\OtherTok{\textless{}{-}} \ControlFlowTok{function}\NormalTok{ (data, bin\_width) \{}
  \CommentTok{\# Cumulative data uses same bins as histogram}
\NormalTok{  data }\OtherTok{\textless{}{-}} \FunctionTok{sort}\NormalTok{(data)}
\NormalTok{  data\_rounded }\OtherTok{\textless{}{-}} \FunctionTok{round\_data\_to\_bin}\NormalTok{(data, bin\_width)}
  
  \CommentTok{\# Calculate highest count so axes can be adjusted}
\NormalTok{  highest\_count }\OtherTok{\textless{}{-}} \FunctionTok{get\_histogram\_max\_count}\NormalTok{(data, bin\_width)}
\NormalTok{  data\_pareto }\OtherTok{\textless{}{-}} \FunctionTok{seq}\NormalTok{(}\DecValTok{1}\NormalTok{, }\FunctionTok{length}\NormalTok{(data), }\AttributeTok{by =} \DecValTok{1}\NormalTok{) }\SpecialCharTok{/} \FunctionTok{length}\NormalTok{(data) }\SpecialCharTok{*}\NormalTok{ highest\_count}

  \CommentTok{\# Calculate stats for overlay on histogram}
\NormalTok{  data\_max }\OtherTok{\textless{}{-}} \FunctionTok{max}\NormalTok{(data)}
\NormalTok{  data\_mean }\OtherTok{\textless{}{-}} \FunctionTok{round}\NormalTok{(}\FunctionTok{mean}\NormalTok{(data), }\AttributeTok{digits =} \DecValTok{2}\NormalTok{)}
\NormalTok{  data\_median }\OtherTok{\textless{}{-}} \FunctionTok{round}\NormalTok{(}\FunctionTok{median}\NormalTok{(data), }\AttributeTok{digits =} \DecValTok{2}\NormalTok{)}
\NormalTok{  sum\_stats }\OtherTok{\textless{}{-}} \FunctionTok{data.frame}\NormalTok{(}\AttributeTok{Statistics =} \FunctionTok{c}\NormalTok{(}\StringTok{"Mean"}\NormalTok{, }\StringTok{"Median"}\NormalTok{),}
                          \AttributeTok{value =} \FunctionTok{c}\NormalTok{(data\_mean, data\_median))}
\NormalTok{  label\_text }\OtherTok{\textless{}{-}} \FunctionTok{paste}\NormalTok{(}\StringTok{"Mean ="}\NormalTok{, data\_mean, }\StringTok{", Median ="}\NormalTok{, data\_median)}

\NormalTok{  p }\OtherTok{\textless{}{-}} \FunctionTok{ggplot}\NormalTok{() }\SpecialCharTok{+}
    \FunctionTok{geom\_histogram}\NormalTok{(}\FunctionTok{aes}\NormalTok{(}\AttributeTok{x =}\NormalTok{ data),}
                   \AttributeTok{color =} \StringTok{"white"}\NormalTok{,}
                   \AttributeTok{binwidth =}\NormalTok{ bin\_width) }\SpecialCharTok{+}
    \FunctionTok{geom\_line}\NormalTok{(}\FunctionTok{aes}\NormalTok{(}\AttributeTok{x =}\NormalTok{ data\_rounded,}
                  \AttributeTok{y =}\NormalTok{ data\_pareto),}
                  \AttributeTok{color=}\StringTok{"forestgreen"}\NormalTok{) }\SpecialCharTok{+}
    \FunctionTok{geom\_vline}\NormalTok{(}\AttributeTok{data =}\NormalTok{ sum\_stats,}
               \FunctionTok{aes}\NormalTok{(}\AttributeTok{xintercept =}\NormalTok{ value,}
                   \AttributeTok{linetype =}\NormalTok{ Statistics,}
                   \AttributeTok{color =}\NormalTok{ Statistics),}
               \AttributeTok{size =} \DecValTok{1}\NormalTok{) }\SpecialCharTok{+}
    \FunctionTok{scale\_x\_continuous}\NormalTok{(}\AttributeTok{breaks =} \FunctionTok{seq}\NormalTok{(}\DecValTok{0}\NormalTok{, data\_max }\SpecialCharTok{+}\NormalTok{ bin\_width, }\AttributeTok{by =}\NormalTok{ bin\_width)) }\SpecialCharTok{+}
    \FunctionTok{scale\_y\_continuous}\NormalTok{(}\AttributeTok{name =} \StringTok{"Users"}\NormalTok{,}
                       \AttributeTok{breaks =} \FunctionTok{seq}\NormalTok{(}\DecValTok{0}\NormalTok{, highest\_count, }\AttributeTok{by =} \DecValTok{1}\NormalTok{),}
                       \AttributeTok{sec.axis =} \FunctionTok{sec\_axis}\NormalTok{(}\SpecialCharTok{\textasciitilde{}}\NormalTok{.}\SpecialCharTok{/}\NormalTok{highest\_count, }\AttributeTok{name =} \StringTok{"Cumulative Percentage of Users"}\NormalTok{)) }\SpecialCharTok{+}
    \CommentTok{\# annotate("text", x = Inf, y = Inf, label = label\_text,}
    \CommentTok{\#        hjust = 1, vjust = 1, size = 4) +}
    \FunctionTok{labs}\NormalTok{(}\AttributeTok{x =} \StringTok{"Value"}\NormalTok{,}
         \AttributeTok{caption =}\NormalTok{ label\_text)}
  \FunctionTok{return}\NormalTok{(p)}
\NormalTok{\}}
\end{Highlighting}
\end{Shaded}

\begin{Shaded}
\begin{Highlighting}[]
\NormalTok{get\_time\_coefficients }\OtherTok{\textless{}{-}} \ControlFlowTok{function}\NormalTok{(ids, timestamps, values) \{}
\NormalTok{  tbl }\OtherTok{\textless{}{-}} \FunctionTok{tibble}\NormalTok{(}
    \AttributeTok{id =}\NormalTok{ ids,}
    \AttributeTok{timestamp =}\NormalTok{ timestamps,}
    \AttributeTok{value =}\NormalTok{ values}
\NormalTok{  )}

\NormalTok{  coeffs }\OtherTok{\textless{}{-}}\NormalTok{ tbl }\SpecialCharTok{\%\textgreater{}\%}
    \FunctionTok{mutate}\NormalTok{(}\AttributeTok{day\_of\_year =} \FunctionTok{yday}\NormalTok{(timestamp)) }\SpecialCharTok{\%\textgreater{}\%}
    \FunctionTok{group\_by}\NormalTok{(id) }\SpecialCharTok{\%\textgreater{}\%}
    \FunctionTok{summarize}\NormalTok{(}\AttributeTok{correlation =} \FunctionTok{cor}\NormalTok{(day\_of\_year, value))}

  \FunctionTok{return}\NormalTok{(coeffs)}
\NormalTok{\}}

\NormalTok{plot\_time\_coefficients }\OtherTok{\textless{}{-}} \ControlFlowTok{function}\NormalTok{(coeffs) \{}
\NormalTok{  bin\_width }\OtherTok{\textless{}{-}} \FloatTok{0.1}
\NormalTok{  max\_count }\OtherTok{\textless{}{-}} \FunctionTok{get\_histogram\_max\_count}\NormalTok{(coeffs}\SpecialCharTok{$}\NormalTok{correlation, bin\_width)}
  
\NormalTok{  coeffs\_mean }\OtherTok{\textless{}{-}} \FunctionTok{round}\NormalTok{(}\FunctionTok{mean}\NormalTok{(coeffs}\SpecialCharTok{$}\NormalTok{correlation, }\AttributeTok{na.rm =} \ConstantTok{TRUE}\NormalTok{), }\AttributeTok{digits =} \DecValTok{2}\NormalTok{)}
\NormalTok{  coeffs\_median }\OtherTok{\textless{}{-}} \FunctionTok{round}\NormalTok{(}\FunctionTok{median}\NormalTok{(coeffs}\SpecialCharTok{$}\NormalTok{correlation, }\AttributeTok{na.rm =} \ConstantTok{TRUE}\NormalTok{), }\AttributeTok{digits =} \DecValTok{2}\NormalTok{)}
\NormalTok{  sum\_stats }\OtherTok{\textless{}{-}} \FunctionTok{data.frame}\NormalTok{(}\AttributeTok{Statistics =} \FunctionTok{c}\NormalTok{(}\StringTok{"Mean"}\NormalTok{, }\StringTok{"Median"}\NormalTok{),}
                          \AttributeTok{value =} \FunctionTok{c}\NormalTok{(coeffs\_mean, coeffs\_median))}
\NormalTok{  label\_text }\OtherTok{\textless{}{-}} \FunctionTok{paste}\NormalTok{(}\StringTok{"Mean ="}\NormalTok{, coeffs\_mean, }\StringTok{"}\SpecialCharTok{\textbackslash{}n}\StringTok{Median ="}\NormalTok{, coeffs\_median)}
  
  \FunctionTok{ggplot}\NormalTok{(coeffs, }\FunctionTok{aes}\NormalTok{(}\AttributeTok{x =}\NormalTok{ correlation)) }\SpecialCharTok{+}
    \FunctionTok{geom\_histogram}\NormalTok{(}\AttributeTok{binwidth =}\NormalTok{ bin\_width, }\AttributeTok{color =} \StringTok{"white"}\NormalTok{) }\SpecialCharTok{+}
    \FunctionTok{geom\_vline}\NormalTok{(}\AttributeTok{data =}\NormalTok{ sum\_stats,}
               \FunctionTok{aes}\NormalTok{(}\AttributeTok{xintercept =}\NormalTok{ value,}
                   \AttributeTok{linetype =}\NormalTok{ Statistics,}
                   \AttributeTok{color =}\NormalTok{ Statistics),}
               \AttributeTok{size =} \DecValTok{1}\NormalTok{) }\SpecialCharTok{+}
    \FunctionTok{scale\_x\_continuous}\NormalTok{(}\AttributeTok{breaks =} \FunctionTok{seq}\NormalTok{(}\SpecialCharTok{{-}}\FloatTok{1.0}\NormalTok{, }\FloatTok{1.0}\NormalTok{, }\AttributeTok{by=}\NormalTok{bin\_width)) }\SpecialCharTok{+}
    \FunctionTok{scale\_y\_continuous}\NormalTok{(}\AttributeTok{breaks =} \FunctionTok{seq}\NormalTok{(}\DecValTok{0}\NormalTok{, max\_count, }\AttributeTok{by=}\DecValTok{1}\NormalTok{)) }\SpecialCharTok{+}
    \FunctionTok{annotate}\NormalTok{(}\StringTok{"text"}\NormalTok{, }\AttributeTok{x =} \ConstantTok{Inf}\NormalTok{, }\AttributeTok{y =} \ConstantTok{Inf}\NormalTok{, }\AttributeTok{label =}\NormalTok{ label\_text,}
           \AttributeTok{hjust =} \DecValTok{1}\NormalTok{, }\AttributeTok{vjust =} \DecValTok{1}\NormalTok{, }\AttributeTok{size =} \DecValTok{4}\NormalTok{) }\SpecialCharTok{+}
    \FunctionTok{labs}\NormalTok{(}\AttributeTok{title =} \StringTok{"Correlations with Time"}\NormalTok{,}
         \AttributeTok{caption =} \StringTok{"Positive values imply variable of interest generally increased over time."}\NormalTok{,}
         \AttributeTok{x =} \StringTok{"Coefficient of Correlation"}\NormalTok{,}
         \AttributeTok{y =} \StringTok{"Count"}\NormalTok{)}
\NormalTok{\}}

\NormalTok{get\_time\_coefficients\_plot }\OtherTok{\textless{}{-}} \ControlFlowTok{function}\NormalTok{(ids, timestamps, values) \{}
\NormalTok{  coeffs }\OtherTok{\textless{}{-}} \FunctionTok{get\_time\_coefficients}\NormalTok{(ids, timestamps, values)}
\NormalTok{  p }\OtherTok{\textless{}{-}} \FunctionTok{plot\_time\_coefficients}\NormalTok{(coeffs)}
  \FunctionTok{return}\NormalTok{(p)}
\NormalTok{\}}
\end{Highlighting}
\end{Shaded}

\begin{Shaded}
\begin{Highlighting}[]
\NormalTok{wide\_to\_stacked\_bar\_plot }\OtherTok{\textless{}{-}} \ControlFlowTok{function}\NormalTok{(data\_wide, key, value, key\_order) \{}
  \ControlFlowTok{if}\NormalTok{(}\SpecialCharTok{!}\NormalTok{(}\StringTok{"id"} \SpecialCharTok{\%in\%} \FunctionTok{colnames}\NormalTok{(data\_wide))) \{}
    \FunctionTok{print}\NormalTok{(}\StringTok{"ERROR: data does not include }\SpecialCharTok{\textbackslash{}"}\StringTok{id}\SpecialCharTok{\textbackslash{}"}\StringTok{ column: cannot convert."}\NormalTok{)}
\NormalTok{  \} }\ControlFlowTok{else}\NormalTok{ \{}
    \CommentTok{\# Convert the data from wide to long. Set the factor levels to control the stacking order of the bars}
\NormalTok{    data\_long }\OtherTok{\textless{}{-}}\NormalTok{ data\_wide }\SpecialCharTok{\%\textgreater{}\%}
\NormalTok{      tidyr}\SpecialCharTok{::}\FunctionTok{gather}\NormalTok{(}\AttributeTok{key =} \SpecialCharTok{!!}\FunctionTok{sym}\NormalTok{(key), }\AttributeTok{value =} \SpecialCharTok{!!}\FunctionTok{sym}\NormalTok{(value), }\SpecialCharTok{{-}}\NormalTok{id) }\SpecialCharTok{\%\textgreater{}\%}
      \FunctionTok{mutate}\NormalTok{(}\SpecialCharTok{!!}\FunctionTok{sym}\NormalTok{(key) }\SpecialCharTok{:=} \FunctionTok{factor}\NormalTok{(}\SpecialCharTok{!!}\FunctionTok{sym}\NormalTok{(key), }\AttributeTok{levels =}\NormalTok{ key\_order))}

    \CommentTok{\# Order IDs in wide data based on value of first key, then rearrange long data.}
    \CommentTok{\# This ensures the resultant plot sorts the IDs in ascending order of the first key}
\NormalTok{    first\_key }\OtherTok{\textless{}{-}}\NormalTok{ key\_order[}\DecValTok{1}\NormalTok{]}
\NormalTok{    data\_wide }\OtherTok{\textless{}{-}}\NormalTok{ data\_wide }\SpecialCharTok{\%\textgreater{}\%} \FunctionTok{arrange}\NormalTok{(}\SpecialCharTok{!!}\FunctionTok{sym}\NormalTok{(first\_key))}
\NormalTok{    data\_long}\SpecialCharTok{$}\NormalTok{id }\OtherTok{\textless{}{-}} \FunctionTok{factor}\NormalTok{(}
\NormalTok{      data\_long}\SpecialCharTok{$}\NormalTok{id,}
      \AttributeTok{levels =}\NormalTok{ data\_wide}\SpecialCharTok{$}\NormalTok{id)}
    
    \CommentTok{\# Plot graph with angled ID labels}
\NormalTok{    p }\OtherTok{\textless{}{-}} \FunctionTok{ggplot}\NormalTok{(data\_long, }\FunctionTok{aes}\NormalTok{(}\AttributeTok{x=}\NormalTok{ id, }\AttributeTok{y=} \SpecialCharTok{!!}\FunctionTok{sym}\NormalTok{(value), }\AttributeTok{fill=} \SpecialCharTok{!!}\FunctionTok{sym}\NormalTok{(key))) }\SpecialCharTok{+}
      \FunctionTok{geom\_bar}\NormalTok{(}\AttributeTok{stat =} \StringTok{"identity"}\NormalTok{) }\SpecialCharTok{+}
      \FunctionTok{theme}\NormalTok{(}\AttributeTok{axis.text.x =} \FunctionTok{element\_text}\NormalTok{(}\AttributeTok{angle =} \DecValTok{45}\NormalTok{, }\AttributeTok{hjust =} \DecValTok{1}\NormalTok{))}
    \CommentTok{\# scale\_y\_continuous(breaks = seq(0, 24, by = 1))}
\NormalTok{  \}}
\NormalTok{\}}
\end{Highlighting}
\end{Shaded}

\begin{Shaded}
\begin{Highlighting}[]
\NormalTok{scatter\_with\_LOBF }\OtherTok{\textless{}{-}} \ControlFlowTok{function}\NormalTok{(data\_x, data\_y, }\AttributeTok{labels =} \ConstantTok{NULL}\NormalTok{) \{}
\NormalTok{  data }\OtherTok{\textless{}{-}} \FunctionTok{tibble}\NormalTok{(}
    \AttributeTok{x =}\NormalTok{ data\_x,}
    \AttributeTok{y =}\NormalTok{ data\_y}
\NormalTok{  )}
  
\NormalTok{  p }\OtherTok{\textless{}{-}} \FunctionTok{ggplot}\NormalTok{(data,}
              \FunctionTok{aes}\NormalTok{(}\AttributeTok{x =}\NormalTok{ x, }\AttributeTok{y =}\NormalTok{ y)) }\SpecialCharTok{+}
    \FunctionTok{geom\_point}\NormalTok{() }\SpecialCharTok{+}
    \FunctionTok{geom\_smooth}\NormalTok{(}\AttributeTok{method =} \StringTok{"lm"}\NormalTok{,}
                \AttributeTok{se =} \ConstantTok{FALSE}\NormalTok{,}
                \AttributeTok{color =} \StringTok{"blue"}\NormalTok{) }\SpecialCharTok{+}
    \FunctionTok{stat\_cor}\NormalTok{(}\AttributeTok{mapping=}\FunctionTok{aes}\NormalTok{(}\AttributeTok{label=}\NormalTok{..rr.label..),}
             \AttributeTok{method=}\StringTok{"pearson"}\NormalTok{,}
             \AttributeTok{label.x=}\SpecialCharTok{{-}}\ConstantTok{Inf}\NormalTok{,}
             \AttributeTok{label.y=}\ConstantTok{Inf}\NormalTok{,}
             \AttributeTok{hjust =} \SpecialCharTok{{-}}\FloatTok{0.1}\NormalTok{,}
             \AttributeTok{vjust =} \FloatTok{1.1}\NormalTok{) }\SpecialCharTok{+}
    \FunctionTok{geom\_text}\NormalTok{(}\FunctionTok{aes}\NormalTok{(}\AttributeTok{label =} \FunctionTok{sprintf}\NormalTok{(}\StringTok{"Gradient: \%.2f"}\NormalTok{, }\FunctionTok{coef}\NormalTok{(}\FunctionTok{lm}\NormalTok{(y }\SpecialCharTok{\textasciitilde{}}\NormalTok{ x, }\AttributeTok{data =}\NormalTok{ data))[}\DecValTok{2}\NormalTok{])),}
              \AttributeTok{x =} \SpecialCharTok{{-}}\ConstantTok{Inf}\NormalTok{,}
              \AttributeTok{y=} \ConstantTok{Inf}\NormalTok{,}
              \AttributeTok{hjust =} \SpecialCharTok{{-}}\FloatTok{0.05}\NormalTok{,}
              \AttributeTok{vjust =} \FloatTok{3.5}\NormalTok{)}
  
  \ControlFlowTok{if}\NormalTok{(}\SpecialCharTok{!}\FunctionTok{is.null}\NormalTok{(labels) }\SpecialCharTok{\&\&} \FunctionTok{length}\NormalTok{(labels }\SpecialCharTok{\textgreater{}} \DecValTok{1}\NormalTok{)) \{}
\NormalTok{    p }\OtherTok{\textless{}{-}}\NormalTok{ p }\SpecialCharTok{+} \FunctionTok{labs}\NormalTok{(}\AttributeTok{title =} \FunctionTok{paste}\NormalTok{(labels[}\DecValTok{1}\NormalTok{], }\StringTok{"vs."}\NormalTok{, labels[}\DecValTok{2}\NormalTok{]),}
                  \AttributeTok{x =}\NormalTok{ labels[}\DecValTok{1}\NormalTok{],}
                  \AttributeTok{y =}\NormalTok{ labels[}\DecValTok{2}\NormalTok{])}
\NormalTok{  \}}
  
  \FunctionTok{return}\NormalTok{(p)}
\NormalTok{\}}
\end{Highlighting}
\end{Shaded}

\hypertarget{heart-rate-tracking-activity-tracking-and-step-counts}{%
\subsection{Heart Rate Tracking, Activity Tracking, and Step
Counts}\label{heart-rate-tracking-activity-tracking-and-step-counts}}

\hypertarget{feature-overview}{%
\subsubsection{Feature Overview}\label{feature-overview}}

\begin{itemize}
\item
  Heart-rate Tracking: ``Use it to track your workout progress and
  optimize personal training routines.''
\item
  Exercise Tracking: ``Ivy will recognize your activity during the day,
  help you track up to 80 types of activity, count your steps, and
  discover how all that affects your body.''
\item
  Step Counts
\item
  Do people count their steps? Yeah, all but three of them did
\item
  Do people engage in different types of activity? Yeah, big spread of
  Fairly/Very Active intensity levels, safe to say people don't all work
  out the same, so the more activity tracking the merrier
\end{itemize}

These three features are closely related, since one of the primary
functions of the HR tracking is to detect exercise intensity; steps and
distance tracking can also be used to further analyse users' activity.

In order to better understand how people are using their FitBits, I
analysed the amounts and intensity of different users exercise.

\hypertarget{analysis}{%
\subsubsection{Analysis}\label{analysis}}

\hypertarget{how-much-time-do-people-spend-exercising}{%
\paragraph{How much time do people spend
exercising?}\label{how-much-time-do-people-spend-exercising}}

I start by getting the top-level breakdown of users time vs intensity:

\begin{Shaded}
\begin{Highlighting}[]
\CommentTok{\# For each ID, generate averages for the three non{-}sedentary intensity levels}
\NormalTok{mean\_daily\_intensities\_wide }\OtherTok{\textless{}{-}}\NormalTok{ intensity\_sum\_days\_wide }\SpecialCharTok{\%\textgreater{}\%}
  \FunctionTok{group\_by}\NormalTok{(id) }\SpecialCharTok{\%\textgreater{}\%}
  \FunctionTok{summarize}\NormalTok{(}\StringTok{"sedentary"} \OtherTok{=} \FunctionTok{mean}\NormalTok{(minutes\_sedentary) }\SpecialCharTok{/} \DecValTok{60}\NormalTok{,}
            \StringTok{"lightly\_active"} \OtherTok{=} \FunctionTok{mean}\NormalTok{(minutes\_lightly\_active) }\SpecialCharTok{/} \DecValTok{60}\NormalTok{,}
            \StringTok{"fairly\_active"} \OtherTok{=} \FunctionTok{mean}\NormalTok{(minutes\_fairly\_active) }\SpecialCharTok{/} \DecValTok{60}\NormalTok{,}
            \StringTok{"very\_active"} \OtherTok{=} \FunctionTok{mean}\NormalTok{(minutes\_very\_active) }\SpecialCharTok{/} \DecValTok{60}\NormalTok{)}

\CommentTok{\#Convert data to long{-}format for plotting as a histogram}
\NormalTok{intensity\_order }\OtherTok{=} \FunctionTok{c}\NormalTok{(}\StringTok{"sedentary"}\NormalTok{, }\StringTok{"lightly\_active"}\NormalTok{, }\StringTok{"fairly\_active"}\NormalTok{, }\StringTok{"very\_active"}\NormalTok{)}
\NormalTok{mean\_daily\_intensities\_long }\OtherTok{\textless{}{-}}\NormalTok{ mean\_daily\_intensities\_wide }\SpecialCharTok{\%\textgreater{}\%}
\NormalTok{  tidyr}\SpecialCharTok{::}\FunctionTok{gather}\NormalTok{(}\AttributeTok{key =} \StringTok{"intensity"}\NormalTok{, }\AttributeTok{value =} \StringTok{"mean\_hours"}\NormalTok{, }\SpecialCharTok{{-}}\NormalTok{id) }\SpecialCharTok{\%\textgreater{}\%}
  \FunctionTok{mutate}\NormalTok{(}\AttributeTok{intensity =} \FunctionTok{factor}\NormalTok{(intensity, }\AttributeTok{levels =}\NormalTok{ intensity\_order))}

\CommentTok{\# Reorder IDs in order of "Very Active" time}
\NormalTok{mean\_daily\_intensities\_wide }\OtherTok{\textless{}{-}}\NormalTok{ mean\_daily\_intensities\_wide }\SpecialCharTok{\%\textgreater{}\%}
  \FunctionTok{mutate}\NormalTok{(}\StringTok{"total\_active"} \OtherTok{=}\NormalTok{ fairly\_active }\SpecialCharTok{+}\NormalTok{ very\_active) }\SpecialCharTok{\%\textgreater{}\%}
  \FunctionTok{arrange}\NormalTok{(total\_active)}

\CommentTok{\# Apply order of IDs to long{-}format data to force plot order}
\NormalTok{mean\_daily\_intensities\_long}\SpecialCharTok{$}\NormalTok{id }\OtherTok{\textless{}{-}} \FunctionTok{factor}\NormalTok{(}
\NormalTok{  mean\_daily\_intensities\_long}\SpecialCharTok{$}\NormalTok{id,}
  \AttributeTok{levels =}\NormalTok{ mean\_daily\_intensities\_wide}\SpecialCharTok{$}\NormalTok{id)}
\end{Highlighting}
\end{Shaded}

\begin{Shaded}
\begin{Highlighting}[]
\NormalTok{viz\_avg\_time\_by\_intensity }\OtherTok{\textless{}{-}} \FunctionTok{ggplot}\NormalTok{(mean\_daily\_intensities\_long, }\FunctionTok{aes}\NormalTok{(}\AttributeTok{x=}\NormalTok{ id, }\AttributeTok{y=}\NormalTok{ mean\_hours, }\AttributeTok{fill=}\NormalTok{ intensity)) }\SpecialCharTok{+}
  \FunctionTok{geom\_bar}\NormalTok{(}\AttributeTok{stat =} \StringTok{"identity"}\NormalTok{) }\SpecialCharTok{+}
  \FunctionTok{theme}\NormalTok{(}\AttributeTok{axis.text.x =} \FunctionTok{element\_text}\NormalTok{(}\AttributeTok{angle =} \DecValTok{45}\NormalTok{, }\AttributeTok{hjust =} \DecValTok{1}\NormalTok{)) }\SpecialCharTok{+}
  \FunctionTok{scale\_y\_continuous}\NormalTok{(}\AttributeTok{breaks =} \FunctionTok{seq}\NormalTok{(}\DecValTok{0}\NormalTok{, }\DecValTok{24}\NormalTok{, }\AttributeTok{by =} \DecValTok{1}\NormalTok{)) }\SpecialCharTok{+}
  \FunctionTok{labs}\NormalTok{(}\AttributeTok{title=}\StringTok{"Average Time by Intensity Zone"}\NormalTok{,}
       \AttributeTok{x =} \StringTok{"User ID"}\NormalTok{,}
       \AttributeTok{y =} \StringTok{"Total Average Daily Time (hours)"}\NormalTok{)}
\FunctionTok{print}\NormalTok{(viz\_avg\_time\_by\_intensity)}
\end{Highlighting}
\end{Shaded}

\includegraphics{BellabeatCaseStudy_files/figure-latex/viz_mean_daily_intensities_by_id-1.pdf}

Preliminary Findings:

\begin{itemize}
\tightlist
\item
  The large majority of people's time is clearly being spent in a
  sedentary state.
\item
  This makes sense, since it includes sleep tracking as well as time
  spent awake but immobile
\item
  There is a wide spread of times spent non-sedentary, with no clear
  outliers at this point
\item
  The large majority of non-sedentary time is Lightly Active
\end{itemize}

Given how large the majority of non-Sedentary time is ``Lightly
Active'', and based on our understanding of the Intensity zones from
previous sections, I will assume ``Lightly Active'' time includes any
activity more intense than sitting down, up to an including activities
like walking for leisure. Anything more intense would therefore fall
into the ``Fairly Active'' or ``Very Active'' categories. With this in
mind, I'll proceed with the assumption that the ``Fairly Active'' and
``Very Active'' zones represent intentional exercise.

Next, I analyse users time spent intentionally exercising:

\begin{Shaded}
\begin{Highlighting}[]
\CommentTok{\# Update total\_active to reflect new definition}
\NormalTok{mean\_daily\_intensities\_wide }\OtherTok{\textless{}{-}}\NormalTok{ mean\_daily\_intensities\_wide }\SpecialCharTok{\%\textgreater{}\%}
  \FunctionTok{mutate}\NormalTok{(}\AttributeTok{total\_active =}\NormalTok{ fairly\_active }\SpecialCharTok{+}\NormalTok{ very\_active) }\SpecialCharTok{\%\textgreater{}\%}
  \FunctionTok{arrange}\NormalTok{(total\_active)}

\NormalTok{mean\_active\_hours }\OtherTok{\textless{}{-}}\NormalTok{ mean\_daily\_intensities\_wide }\SpecialCharTok{\%\textgreater{}\%}
  \FunctionTok{select}\NormalTok{(id, total\_active)}

\CommentTok{\# Add cumulative user count for Pareto analysis}
\NormalTok{total\_users }\OtherTok{\textless{}{-}} \FunctionTok{nrow}\NormalTok{(mean\_active\_hours)}
\NormalTok{bin\_width }\OtherTok{\textless{}{-}} \FloatTok{0.25}
\NormalTok{rounding\_factor }\OtherTok{\textless{}{-}} \DecValTok{1} \SpecialCharTok{/}\NormalTok{ bin\_width}
\NormalTok{mean\_active\_hours }\OtherTok{\textless{}{-}}\NormalTok{ mean\_active\_hours }\SpecialCharTok{\%\textgreater{}\%}
  \FunctionTok{mutate}\NormalTok{(}\AttributeTok{total\_active\_rounded =} \FunctionTok{round}\NormalTok{(total\_active }\SpecialCharTok{*}\NormalTok{ rounding\_factor) }\SpecialCharTok{/}\NormalTok{ rounding\_factor) }\SpecialCharTok{\%\textgreater{}\%}
  \FunctionTok{mutate}\NormalTok{(}\AttributeTok{cumulative\_users =} \FunctionTok{row\_number}\NormalTok{() }\SpecialCharTok{/}\NormalTok{ total\_users)}

\NormalTok{mean\_hours }\OtherTok{\textless{}{-}} \FunctionTok{mean}\NormalTok{(mean\_active\_hours}\SpecialCharTok{$}\NormalTok{total\_active)}
\NormalTok{median\_hours }\OtherTok{\textless{}{-}} \FunctionTok{median}\NormalTok{(mean\_active\_hours}\SpecialCharTok{$}\NormalTok{total\_active)}

\NormalTok{ymax }\OtherTok{\textless{}{-}} \DecValTok{12}
\NormalTok{xmax }\OtherTok{\textless{}{-}} \FloatTok{13.5}
\end{Highlighting}
\end{Shaded}

\begin{Shaded}
\begin{Highlighting}[]
\NormalTok{viz\_active\_hrs\_per\_week }\OtherTok{\textless{}{-}} \FunctionTok{ggplot}\NormalTok{(mean\_active\_hours, }\FunctionTok{aes}\NormalTok{(}\AttributeTok{x=}\NormalTok{total\_active)) }\SpecialCharTok{+}
  \FunctionTok{geom\_histogram}\NormalTok{(}\AttributeTok{binwidth =}\NormalTok{ bin\_width, }\AttributeTok{col=}\StringTok{"white"}\NormalTok{) }\SpecialCharTok{+}
  \FunctionTok{scale\_x\_continuous}\NormalTok{(}\AttributeTok{breaks =} \FunctionTok{seq}\NormalTok{(}\DecValTok{0}\NormalTok{, xmax, }\AttributeTok{by =}\NormalTok{ bin\_width)) }\SpecialCharTok{+}
  \FunctionTok{geom\_line}\NormalTok{(}\FunctionTok{aes}\NormalTok{(}\AttributeTok{x =}\NormalTok{ total\_active\_rounded,}
                \AttributeTok{y =}\NormalTok{ cumulative\_users }\SpecialCharTok{*}\NormalTok{ ymax),}
            \AttributeTok{col=}\StringTok{"red"}\NormalTok{) }\SpecialCharTok{+}
  \FunctionTok{geom\_vline}\NormalTok{(}\FunctionTok{aes}\NormalTok{(}\AttributeTok{xintercept =}\NormalTok{ mean\_hours, }\AttributeTok{col =} \StringTok{\textquotesingle{}red\textquotesingle{}}\NormalTok{)) }\SpecialCharTok{+}
  \FunctionTok{geom\_vline}\NormalTok{(}\FunctionTok{aes}\NormalTok{(}\AttributeTok{xintercept =}\NormalTok{ median\_hours), }\AttributeTok{col =} \StringTok{\textquotesingle{}blue\textquotesingle{}}\NormalTok{) }\SpecialCharTok{+}
  \FunctionTok{scale\_y\_continuous}\NormalTok{(}\AttributeTok{name =} \StringTok{"Number of Users"}\NormalTok{,}
                     \AttributeTok{breaks =} \FunctionTok{seq}\NormalTok{(}\DecValTok{0}\NormalTok{, ymax, }\AttributeTok{by =} \DecValTok{1}\NormalTok{),}
                     \AttributeTok{sec.axis =} \FunctionTok{sec\_axis}\NormalTok{(}\SpecialCharTok{\textasciitilde{}}\NormalTok{.}\SpecialCharTok{/}\NormalTok{ymax,}\AttributeTok{name =} \StringTok{"Cumulative Percentage of Users"}\NormalTok{)) }\SpecialCharTok{+}
  \FunctionTok{labs}\NormalTok{(}\AttributeTok{title =} \StringTok{"Active Hours per Week"}\NormalTok{,}
       \AttributeTok{x =} \StringTok{"Hours"}\NormalTok{)}
\NormalTok{viz\_active\_hrs\_per\_week }\OtherTok{\textless{}{-}} \FunctionTok{plot\_histo\_pareto}\NormalTok{(mean\_active\_hours}\SpecialCharTok{$}\NormalTok{total\_active, }\AttributeTok{bin\_width =} \FloatTok{0.25}\NormalTok{) }\SpecialCharTok{+}
  \FunctionTok{labs}\NormalTok{(}\AttributeTok{title =} \StringTok{"Active Hours per Week"}\NormalTok{,}
       \AttributeTok{x =} \StringTok{"Hours"}\NormalTok{)}
\FunctionTok{print}\NormalTok{(viz\_active\_hrs\_per\_week)}
\end{Highlighting}
\end{Shaded}

\includegraphics{BellabeatCaseStudy_files/figure-latex/viz_active_hours_by_count-1.pdf}

Preliminary Findings:

\begin{itemize}
\tightlist
\item
  72.7\% of the cohort (24 users) get less than 45 minutes of Active
  Time per week on average.
\item
  18\% of the cohort (6 users) do more than 1 hour a week on average.
\item
  Most people in the dataset are not doing much more than very light
  exercise.
\end{itemize}

\hypertarget{how-intensely-do-people-exercise}{%
\paragraph{How intensely do people
exercise?}\label{how-intensely-do-people-exercise}}

Here I analyse the data to see if there is any variation in the level of
intensity of exercise between users. The variation was analysed by
examining the ratio of ``Fairly Active'' to ``Very Active'' time for
each user:

\begin{Shaded}
\begin{Highlighting}[]
\CommentTok{\# For each ID, generate proportion Very Active time}
\NormalTok{mean\_daily\_intensities\_wide }\OtherTok{\textless{}{-}}\NormalTok{ mean\_daily\_intensities\_wide }\SpecialCharTok{\%\textgreater{}\%}
  \FunctionTok{mutate}\NormalTok{(}\AttributeTok{proportion\_very\_active =}\NormalTok{ very\_active }\SpecialCharTok{/}\NormalTok{ total\_active)}

\CommentTok{\# Convert data to long{-}format for plotting as a histogram}
\NormalTok{intensity\_order }\OtherTok{=} \FunctionTok{c}\NormalTok{(}\StringTok{"fairly\_active"}\NormalTok{, }\StringTok{"very\_active"}\NormalTok{)}
\NormalTok{mean\_daily\_intensities\_long }\OtherTok{\textless{}{-}}\NormalTok{ mean\_daily\_intensities\_wide }\SpecialCharTok{\%\textgreater{}\%}
\NormalTok{  tidyr}\SpecialCharTok{::}\FunctionTok{gather}\NormalTok{(}\AttributeTok{key =} \StringTok{"intensity"}\NormalTok{, }\AttributeTok{value =} \StringTok{"mean\_hours"}\NormalTok{, }\SpecialCharTok{{-}}\NormalTok{id) }\SpecialCharTok{\%\textgreater{}\%}
  \FunctionTok{mutate}\NormalTok{(}\AttributeTok{intensity =} \FunctionTok{factor}\NormalTok{(intensity, }\AttributeTok{levels =}\NormalTok{ intensity\_order)) }\SpecialCharTok{\%\textgreater{}\%}
  \FunctionTok{filter}\NormalTok{(intensity }\SpecialCharTok{==} \StringTok{"fairly\_active"} \SpecialCharTok{|}\NormalTok{ intensity }\SpecialCharTok{==} \StringTok{"very\_active"}\NormalTok{)}

\CommentTok{\# Reorder IDs in order of "Very Active" time}
\NormalTok{mean\_daily\_intensities\_wide }\OtherTok{\textless{}{-}}\NormalTok{ mean\_daily\_intensities\_wide }\SpecialCharTok{\%\textgreater{}\%}
  \FunctionTok{arrange}\NormalTok{(proportion\_very\_active)}

\CommentTok{\# Apply order of IDs to long{-}format data to force plot order}
\NormalTok{mean\_daily\_intensities\_long}\SpecialCharTok{$}\NormalTok{id }\OtherTok{\textless{}{-}} \FunctionTok{factor}\NormalTok{(}
\NormalTok{  mean\_daily\_intensities\_long}\SpecialCharTok{$}\NormalTok{id,}
  \AttributeTok{levels =}\NormalTok{ mean\_daily\_intensities\_wide}\SpecialCharTok{$}\NormalTok{id)}
\end{Highlighting}
\end{Shaded}

\begin{Shaded}
\begin{Highlighting}[]
\NormalTok{viz\_very\_active\_proportion }\OtherTok{\textless{}{-}} \FunctionTok{plot\_histo\_pareto}\NormalTok{(mean\_daily\_intensities\_wide}\SpecialCharTok{$}\NormalTok{proportion\_very\_active,}
                                                \AttributeTok{bin\_width =} \FloatTok{0.1}\NormalTok{) }\SpecialCharTok{+}
  \FunctionTok{labs}\NormalTok{(}\AttributeTok{title =} \StringTok{"Percentage of Active Time spent Very Active"}\NormalTok{)}
\FunctionTok{print}\NormalTok{(viz\_very\_active\_proportion)}
\end{Highlighting}
\end{Shaded}

\includegraphics{BellabeatCaseStudy_files/figure-latex/viz_proportion_intensity-1.pdf}

Preliminary Findings:

\begin{itemize}
\tightlist
\item
  It's quite evenly spread out from around 10\% Very Active to around
  90\% Very Active across all users, with no clear outliers.
\item
  It would appear that the user group engages in a range of exercise
  activities of varying intensity.
\end{itemize}

\hypertarget{do-people-who-track-more-do-more-intense-exercise}{%
\paragraph{Do people who track more do more-intense
exercise?}\label{do-people-who-track-more-do-more-intense-exercise}}

The proportion of Active Time spent Very Active was plotted against the
total Active Time to see if there was a correlation. From the earlier
analysis of overall Active Hours per week, we can see that Active Hours
are skewed right and concentrated towards the lower end: given this
distribution, I also re-ran the analysis twice, once with the upper
quintile of users removed, and once with the lower quintile removed. The
results are plotted below.

\begin{Shaded}
\begin{Highlighting}[]
\NormalTok{plot\_all\_data }\OtherTok{\textless{}{-}} \FunctionTok{ggplot}\NormalTok{(mean\_daily\_intensities\_wide, }\FunctionTok{aes}\NormalTok{(}\AttributeTok{x=}\NormalTok{total\_active, }\AttributeTok{y=}\NormalTok{proportion\_very\_active)) }\SpecialCharTok{+}
  \FunctionTok{geom\_point}\NormalTok{() }\SpecialCharTok{+}
  \FunctionTok{geom\_smooth}\NormalTok{(}\AttributeTok{method =} \StringTok{"lm"}\NormalTok{,}
              \AttributeTok{se =} \ConstantTok{FALSE}\NormalTok{,}
              \AttributeTok{color =} \StringTok{"blue"}\NormalTok{) }\SpecialCharTok{+}
  \FunctionTok{stat\_cor}\NormalTok{(}\FunctionTok{aes}\NormalTok{(}\AttributeTok{label=}\NormalTok{..rr.label..),}
           \AttributeTok{label.x=}\DecValTok{0}\NormalTok{,}
           \AttributeTok{label.y=}\DecValTok{0}\NormalTok{) }\SpecialCharTok{+}
  \FunctionTok{labs}\NormalTok{(}\AttributeTok{title=}\StringTok{"Proportion Very Active vs. Total Active Time"}\NormalTok{,}
       \AttributeTok{x =} \StringTok{"Average Weekly Active Time (hours)"}\NormalTok{,}
       \AttributeTok{y =} \StringTok{"Proportion of Very Active Time (\%)"}\NormalTok{)}
\end{Highlighting}
\end{Shaded}

\begin{Shaded}
\begin{Highlighting}[]
\NormalTok{data }\OtherTok{\textless{}{-}}\NormalTok{ mean\_daily\_intensities\_wide }\SpecialCharTok{\%\textgreater{}\%}
  \FunctionTok{filter}\NormalTok{(total\_active }\SpecialCharTok{\textless{}=} \FloatTok{1.25}\NormalTok{) }\CommentTok{\# id != "3977333714")}

\NormalTok{plot\_no\_upper\_quint }\OtherTok{\textless{}{-}} \FunctionTok{ggplot}\NormalTok{(data, }\FunctionTok{aes}\NormalTok{(}\AttributeTok{x=}\NormalTok{total\_active, }\AttributeTok{y=}\NormalTok{proportion\_very\_active)) }\SpecialCharTok{+}
  \FunctionTok{geom\_point}\NormalTok{() }\SpecialCharTok{+}
  \FunctionTok{geom\_smooth}\NormalTok{(}\AttributeTok{method =} \StringTok{"lm"}\NormalTok{,}
              \AttributeTok{se =} \ConstantTok{FALSE}\NormalTok{,}
              \AttributeTok{color =} \StringTok{"blue"}\NormalTok{) }\SpecialCharTok{+}
  \FunctionTok{stat\_cor}\NormalTok{(}\FunctionTok{aes}\NormalTok{(}\AttributeTok{label=}\NormalTok{..rr.label..),}
           \AttributeTok{label.x=}\DecValTok{0}\NormalTok{,}
           \AttributeTok{label.y=}\DecValTok{0}\NormalTok{) }\SpecialCharTok{+}
  \FunctionTok{labs}\NormalTok{(}\AttributeTok{title=}\StringTok{"Proportion Very Active vs. Total Active Time (Highest Quintlie Removed)"}\NormalTok{,}
       \AttributeTok{x =} \StringTok{"Average Weekly Active Time (hours)"}\NormalTok{,}
       \AttributeTok{y =} \StringTok{"Proportion of Very Active Time (\%)"}\NormalTok{)}

\NormalTok{data }\OtherTok{\textless{}{-}}\NormalTok{ mean\_daily\_intensities\_wide }\SpecialCharTok{\%\textgreater{}\%}
  \FunctionTok{filter}\NormalTok{(total\_active }\SpecialCharTok{\textgreater{}} \FloatTok{0.25}\NormalTok{) }\CommentTok{\# id != "3977333714")}

\NormalTok{plot\_no\_lower\_quint }\OtherTok{\textless{}{-}} \FunctionTok{ggplot}\NormalTok{(data, }\FunctionTok{aes}\NormalTok{(}\AttributeTok{x=}\NormalTok{total\_active, }\AttributeTok{y=}\NormalTok{proportion\_very\_active)) }\SpecialCharTok{+}
  \FunctionTok{geom\_point}\NormalTok{() }\SpecialCharTok{+}
  \FunctionTok{geom\_smooth}\NormalTok{(}\AttributeTok{method =} \StringTok{"lm"}\NormalTok{,}
              \AttributeTok{se =} \ConstantTok{FALSE}\NormalTok{,}
              \AttributeTok{color =} \StringTok{"blue"}\NormalTok{) }\SpecialCharTok{+}
  \FunctionTok{stat\_cor}\NormalTok{(}\FunctionTok{aes}\NormalTok{(}\AttributeTok{label=}\NormalTok{..rr.label..),}
           \AttributeTok{label.x=}\DecValTok{0}\NormalTok{,}
           \AttributeTok{label.y=}\DecValTok{0}\NormalTok{) }\SpecialCharTok{+}
  \FunctionTok{labs}\NormalTok{(}\AttributeTok{title=}\StringTok{"Proportion Very Active vs. Total Active Time (Lowest Quintile Removed)"}\NormalTok{,}
       \AttributeTok{x =} \StringTok{"Average Weekly Active Time (hours)"}\NormalTok{,}
       \AttributeTok{y =} \StringTok{"Proportion of Very Active Time (\%)"}\NormalTok{)}

\FunctionTok{rm}\NormalTok{(data)}

\NormalTok{viz\_very\_active\_vs\_total\_active }\OtherTok{\textless{}{-}}\NormalTok{ plot\_all\_data }\SpecialCharTok{+} 
  \FunctionTok{scale\_x\_continuous}\NormalTok{(}\AttributeTok{limits=}\FunctionTok{c}\NormalTok{(}\DecValTok{0}\NormalTok{,}\DecValTok{2}\NormalTok{)) }\SpecialCharTok{+} 
  \FunctionTok{scale\_y\_continuous}\NormalTok{(}\AttributeTok{limits=}\FunctionTok{c}\NormalTok{(}\DecValTok{0}\NormalTok{,}\DecValTok{1}\NormalTok{))}
\FunctionTok{print}\NormalTok{(viz\_very\_active\_vs\_total\_active)}
\end{Highlighting}
\end{Shaded}

\includegraphics{BellabeatCaseStudy_files/figure-latex/viz_logged_minutes_vs_intensity-1.pdf}

\begin{Shaded}
\begin{Highlighting}[]
\CommentTok{\# plot\_no\_upper\_quint \textless{}{-} plot\_no\_upper\_quint + }
\CommentTok{\#   scale\_x\_continuous(limits=c(0,2)) + }
\CommentTok{\#   scale\_y\_continuous(limits=c(0,1))}
\CommentTok{\# plot\_no\_lower\_quint \textless{}{-} plot\_no\_lower\_quint + }
\CommentTok{\#   scale\_x\_continuous(limits=c(0,2)) + }
\CommentTok{\#   scale\_y\_continuous(limits=c(0,1))}
\CommentTok{\# }
\CommentTok{\# grid.arrange(viz\_very\_active\_vs\_total\_active, plot\_no\_upper\_quint, plot\_no\_lower\_quint, ncol = 3)}
\end{Highlighting}
\end{Shaded}

Preliminary Findings: * Very Active Proportion was positively correlated
with Active time, with an R\^{}2 value of 0.27. * Removing the upper
quintile decreased the correlation to 0.23, indicating that those doing
less exercise also do a wider range of intensities * Removing the lower
quintile increased the correlation to 0.32, indicating that those doing
the most exercise are doing a narrower range of higher-intensity
activities. * Overall, the results indicate that the majority of the
cohort is engaged in a wide range of activities. There is a large subset
of the group doing a low amount of varied exercise, as well as a small
group of outliers doing a higher amount of more-intense exercise.

\hypertarget{do-people-prefer-to-walk-or-run}{%
\paragraph{Do people prefer to walk or
run?}\label{do-people-prefer-to-walk-or-run}}

Why do I care: If people like to run, we can market it at those people

How do we find out if people run? Steps/second Distance/second How will
we know if they're running? Speed above a certain level

I can use the src logs directly as they are logged every minute, so each
data point is also an average speed in steps per minute. From this, I
can set up some ranges for different walking types based on speed, count
the number of steps logs in each range for each person/day, then finally
average the sums over the days to get each users' average time spent at
each speed

\begin{Shaded}
\begin{Highlighting}[]
\NormalTok{brisk\_walk\_thld }\OtherTok{\textless{}{-}} \DecValTok{80}
\NormalTok{running\_thld }\OtherTok{\textless{}{-}} \DecValTok{110}

\NormalTok{mean\_movement\_times\_daily }\OtherTok{\textless{}{-}}\NormalTok{ steps\_src\_mins\_tall }\SpecialCharTok{\%\textgreater{}\%}
  \FunctionTok{mutate}\NormalTok{(}\AttributeTok{activity\_day =} \FunctionTok{as.POSIXct}\NormalTok{(}\FunctionTok{format}\NormalTok{(}\FunctionTok{as.Date}\NormalTok{(activity\_minute)))) }\SpecialCharTok{\%\textgreater{}\%}
  \FunctionTok{group\_by}\NormalTok{(id, activity\_day) }\SpecialCharTok{\%\textgreater{}\%}
  \FunctionTok{summarize}\NormalTok{(}\AttributeTok{not\_walking =} \FunctionTok{sum}\NormalTok{(steps }\SpecialCharTok{==} \DecValTok{0}\NormalTok{) }\SpecialCharTok{/} \DecValTok{60}\NormalTok{,}
            \AttributeTok{moderate\_walking =} \FunctionTok{sum}\NormalTok{(steps }\SpecialCharTok{\textgreater{}} \DecValTok{0} \SpecialCharTok{\&}\NormalTok{ steps }\SpecialCharTok{\textless{}=}\NormalTok{ brisk\_walk\_thld) }\SpecialCharTok{/} \DecValTok{60}\NormalTok{,}
            \AttributeTok{brisk\_walking =} \FunctionTok{sum}\NormalTok{(steps }\SpecialCharTok{\textgreater{}}\NormalTok{ brisk\_walk\_thld }\SpecialCharTok{\&}\NormalTok{ steps }\SpecialCharTok{\textless{}=}\NormalTok{ running\_thld) }\SpecialCharTok{/} \DecValTok{60}\NormalTok{,}
            \AttributeTok{running =} \FunctionTok{sum}\NormalTok{(steps }\SpecialCharTok{\textgreater{}}\NormalTok{ running\_thld) }\SpecialCharTok{/} \DecValTok{60}\NormalTok{)}

\NormalTok{mean\_movement\_times }\OtherTok{\textless{}{-}}\NormalTok{ mean\_movement\_times\_daily }\SpecialCharTok{\%\textgreater{}\%}
  \FunctionTok{group\_by}\NormalTok{(id) }\SpecialCharTok{\%\textgreater{}\%}
  \FunctionTok{summarize}\NormalTok{(}\AttributeTok{not\_walking =} \FunctionTok{mean}\NormalTok{(not\_walking),}
            \AttributeTok{moderate\_walking =} \FunctionTok{mean}\NormalTok{(moderate\_walking),}
            \AttributeTok{brisk\_walking =} \FunctionTok{mean}\NormalTok{(brisk\_walking),}
            \AttributeTok{running =} \FunctionTok{mean}\NormalTok{(running)) }\SpecialCharTok{\%\textgreater{}\%}
  \FunctionTok{arrange}\NormalTok{(running)}

\NormalTok{speed\_order }\OtherTok{=} \FunctionTok{c}\NormalTok{(}\StringTok{"not\_walking"}\NormalTok{, }\StringTok{"moderate\_walking"}\NormalTok{, }\StringTok{"brisk\_walking"}\NormalTok{, }\StringTok{"running"}\NormalTok{)}
\NormalTok{mean\_movement\_times\_long }\OtherTok{\textless{}{-}}\NormalTok{ mean\_movement\_times }\SpecialCharTok{\%\textgreater{}\%}
\NormalTok{  tidyr}\SpecialCharTok{::}\FunctionTok{gather}\NormalTok{(}\AttributeTok{key =} \StringTok{"speed"}\NormalTok{, }\AttributeTok{value =} \StringTok{"mean\_hours"}\NormalTok{, }\SpecialCharTok{{-}}\NormalTok{id) }\SpecialCharTok{\%\textgreater{}\%}
  \FunctionTok{mutate}\NormalTok{(}\AttributeTok{speed =} \FunctionTok{factor}\NormalTok{(speed, }\AttributeTok{levels =}\NormalTok{ speed\_order))}

\NormalTok{mean\_movement\_times\_long}\SpecialCharTok{$}\NormalTok{id }\OtherTok{\textless{}{-}} \FunctionTok{factor}\NormalTok{(}
\NormalTok{  mean\_movement\_times\_long}\SpecialCharTok{$}\NormalTok{id,}
  \AttributeTok{levels =}\NormalTok{ mean\_movement\_times}\SpecialCharTok{$}\NormalTok{id}
\NormalTok{)}
\end{Highlighting}
\end{Shaded}

\begin{Shaded}
\begin{Highlighting}[]
\FunctionTok{ggplot}\NormalTok{(mean\_movement\_times\_long }\SpecialCharTok{\%\textgreater{}\%} \FunctionTok{filter}\NormalTok{(speed }\SpecialCharTok{!=} \StringTok{"not\_walking"}\NormalTok{),}
       \FunctionTok{aes}\NormalTok{(}\AttributeTok{x =}\NormalTok{ id, }\AttributeTok{y =}\NormalTok{ mean\_hours, }\AttributeTok{fill =}\NormalTok{ speed)) }\SpecialCharTok{+}
  \FunctionTok{geom\_bar}\NormalTok{(}\AttributeTok{stat =} \StringTok{"identity"}\NormalTok{) }\SpecialCharTok{+}
  \FunctionTok{theme}\NormalTok{(}\AttributeTok{axis.text.x =} \FunctionTok{element\_text}\NormalTok{(}\AttributeTok{angle =} \DecValTok{45}\NormalTok{, }\AttributeTok{hjust =} \DecValTok{1}\NormalTok{)) }\SpecialCharTok{+}
  \FunctionTok{scale\_y\_continuous}\NormalTok{(}\AttributeTok{breaks =} \FunctionTok{seq}\NormalTok{(}\DecValTok{0}\NormalTok{,}\DecValTok{33}\NormalTok{,}\AttributeTok{by=}\DecValTok{1}\NormalTok{)) }\SpecialCharTok{+}
  \FunctionTok{labs}\NormalTok{(}\AttributeTok{title =} \StringTok{"Average Daily Running Speeds by User ID"}\NormalTok{,}
       \AttributeTok{x =} \StringTok{"User ID"}\NormalTok{,}
       \AttributeTok{y =} \StringTok{"Average Daily Hours"}\NormalTok{)}
\end{Highlighting}
\end{Shaded}

\includegraphics{BellabeatCaseStudy_files/figure-latex/viz_running_habits-1.pdf}

\begin{Shaded}
\begin{Highlighting}[]
\NormalTok{viz\_mean\_mod\_walking }\OtherTok{\textless{}{-}} \FunctionTok{plot\_histo\_pareto}\NormalTok{(mean\_movement\_times}\SpecialCharTok{$}\NormalTok{moderate\_walking,}
                                      \AttributeTok{bin\_width =} \FloatTok{0.25}\NormalTok{) }\SpecialCharTok{+}
  \FunctionTok{labs}\NormalTok{(}\AttributeTok{title =} \StringTok{"Average Moderate Walking Hours per Week"}\NormalTok{,}
       \AttributeTok{x =} \StringTok{"Average Hours per Week"}\NormalTok{)}
\FunctionTok{print}\NormalTok{(viz\_mean\_mod\_walking)}
\end{Highlighting}
\end{Shaded}

\includegraphics{BellabeatCaseStudy_files/figure-latex/viz_running_habits-2.pdf}

\begin{Shaded}
\begin{Highlighting}[]
\NormalTok{viz\_mean\_brisk\_walking }\OtherTok{\textless{}{-}} \FunctionTok{plot\_histo\_pareto}\NormalTok{(mean\_movement\_times}\SpecialCharTok{$}\NormalTok{brisk\_walking,}
                                        \AttributeTok{bin\_width =} \FloatTok{0.25}\NormalTok{) }\SpecialCharTok{+}
  \FunctionTok{labs}\NormalTok{(}\AttributeTok{title =} \StringTok{"Average Brisk Walking Hours per Week"}\NormalTok{,}
       \AttributeTok{x =} \StringTok{"Average Hours per Week"}\NormalTok{)}
\FunctionTok{print}\NormalTok{(viz\_mean\_brisk\_walking)}
\end{Highlighting}
\end{Shaded}

\includegraphics{BellabeatCaseStudy_files/figure-latex/viz_running_habits-3.pdf}

\begin{Shaded}
\begin{Highlighting}[]
\NormalTok{viz\_mean\_running }\OtherTok{\textless{}{-}} \FunctionTok{plot\_histo\_pareto}\NormalTok{(mean\_movement\_times}\SpecialCharTok{$}\NormalTok{running,}
                                  \AttributeTok{bin\_width =} \FloatTok{0.25}\NormalTok{) }\SpecialCharTok{+}
  \FunctionTok{labs}\NormalTok{(}\AttributeTok{title =} \StringTok{"Average Running Hours per Week"}\NormalTok{,}
       \AttributeTok{x =} \StringTok{"Average Hours per Week"}\NormalTok{)}
\FunctionTok{print}\NormalTok{(viz\_mean\_running)}
\end{Highlighting}
\end{Shaded}

\includegraphics{BellabeatCaseStudy_files/figure-latex/viz_running_habits-4.pdf}

\begin{Shaded}
\begin{Highlighting}[]
\FunctionTok{grid.arrange}\NormalTok{(viz\_mean\_mod\_walking,}
\NormalTok{             viz\_mean\_brisk\_walking,}
\NormalTok{             viz\_mean\_running,}
             \AttributeTok{ncol =} \DecValTok{1}\NormalTok{)}
\end{Highlighting}
\end{Shaded}

\includegraphics{BellabeatCaseStudy_files/figure-latex/viz_running_habits-5.pdf}

Preliminary Findings:

\begin{itemize}
\tightlist
\item
  Most people spend most of their time walking slowly
\item
  Brisk walking is somewhat more common, with 24 of 33 users logging
  between 0.25 and 0.5 hours per week
\item
  Runners are rare, with 30 of 33 users logging 0.25 hours or less per
  week, and the remaining three outliers logging between 0.5 and 1.25
  hours per week
\end{itemize}

\hypertarget{do-people-increase-their-active-walking-hours-over-time}{%
\paragraph{Do people increase their active walking hours over
time?}\label{do-people-increase-their-active-walking-hours-over-time}}

\begin{Shaded}
\begin{Highlighting}[]
\NormalTok{viz\_mean\_mod\_walking\_vs\_time }\OtherTok{\textless{}{-}} \FunctionTok{get\_time\_coefficients\_plot}\NormalTok{(mean\_movement\_times\_daily}\SpecialCharTok{$}\NormalTok{id,}
\NormalTok{                                                       mean\_movement\_times\_daily}\SpecialCharTok{$}\NormalTok{activity\_day,}
\NormalTok{                                                       mean\_movement\_times\_daily}\SpecialCharTok{$}\NormalTok{moderate\_walking) }\SpecialCharTok{+}
  \FunctionTok{labs}\NormalTok{(}\AttributeTok{title =} \StringTok{"Moderate Walking Over Time"}\NormalTok{)}

\NormalTok{viz\_mean\_brisk\_walking\_vs\_time }\OtherTok{\textless{}{-}} \FunctionTok{get\_time\_coefficients\_plot}\NormalTok{(mean\_movement\_times\_daily}\SpecialCharTok{$}\NormalTok{id,}
\NormalTok{                                                    mean\_movement\_times\_daily}\SpecialCharTok{$}\NormalTok{activity\_day,}
\NormalTok{                                                    mean\_movement\_times\_daily}\SpecialCharTok{$}\NormalTok{brisk\_walking) }\SpecialCharTok{+}
  \FunctionTok{labs}\NormalTok{(}\AttributeTok{title =} \StringTok{"Brisk Walking Over Time"}\NormalTok{)}

\NormalTok{viz\_mean\_running\_vs\_time }\OtherTok{\textless{}{-}} \FunctionTok{get\_time\_coefficients\_plot}\NormalTok{(mean\_movement\_times\_daily}\SpecialCharTok{$}\NormalTok{id,}
\NormalTok{                                                    mean\_movement\_times\_daily}\SpecialCharTok{$}\NormalTok{activity\_day,}
\NormalTok{                                                    mean\_movement\_times\_daily}\SpecialCharTok{$}\NormalTok{running) }\SpecialCharTok{+}
  \FunctionTok{labs}\NormalTok{(}\AttributeTok{title =} \StringTok{"Running Over Time"}\NormalTok{)}

\FunctionTok{grid.arrange}\NormalTok{(viz\_mean\_mod\_walking\_vs\_time,}
\NormalTok{             viz\_mean\_brisk\_walking\_vs\_time,}
\NormalTok{             viz\_mean\_running\_vs\_time,}
             \AttributeTok{nrow =} \DecValTok{3}\NormalTok{)}
\end{Highlighting}
\end{Shaded}

\includegraphics{BellabeatCaseStudy_files/figure-latex/viz_walking_over_time-1.pdf}

Preliminary Findings:

\begin{itemize}
\tightlist
\item
  Most users actually decreased or did not change their walking time
  over the data period
\item
  This finding was consistent across all walking types
\item
  There is no evidence to suggest FitBit users increased their walknig
  times as a result of their FitBit device usage
\end{itemize}

\hypertarget{are-people-getting-their-steps-in}{%
\paragraph{Are people getting their steps
in?}\label{are-people-getting-their-steps-in}}

\begin{Shaded}
\begin{Highlighting}[]
\NormalTok{mean\_daily\_steps }\OtherTok{\textless{}{-}}\NormalTok{ steps\_sum\_days\_tall }\SpecialCharTok{\%\textgreater{}\%}
  \FunctionTok{group\_by}\NormalTok{(id) }\SpecialCharTok{\%\textgreater{}\%}
  \FunctionTok{summarize}\NormalTok{(}\AttributeTok{mean\_steps =} \FunctionTok{mean}\NormalTok{(steps\_total))}
\end{Highlighting}
\end{Shaded}

\begin{Shaded}
\begin{Highlighting}[]
\NormalTok{viz\_mean\_daily\_steps }\OtherTok{\textless{}{-}} \FunctionTok{plot\_histo\_pareto}\NormalTok{(mean\_daily\_steps}\SpecialCharTok{$}\NormalTok{mean\_steps, }\DecValTok{2000}\NormalTok{) }\SpecialCharTok{+}
  \FunctionTok{labs}\NormalTok{(}\AttributeTok{title =} \StringTok{"Average Daily Steps by User"}\NormalTok{,}
       \AttributeTok{x =} \StringTok{"Average Daily Steps"}\NormalTok{)}
\FunctionTok{print}\NormalTok{(viz\_mean\_daily\_steps)}
\end{Highlighting}
\end{Shaded}

\includegraphics{BellabeatCaseStudy_files/figure-latex/viz_mean_daily_step_counts-1.pdf}

Answer: Typically 7500.

\hypertarget{do-people-increase-their-step-counts-over-time}{%
\paragraph{Do people increase their step counts over
time?}\label{do-people-increase-their-step-counts-over-time}}

\begin{Shaded}
\begin{Highlighting}[]
\NormalTok{viz\_mean\_daily\_steps\_vs\_time }\OtherTok{\textless{}{-}} \FunctionTok{get\_time\_coefficients\_plot}\NormalTok{(steps\_sum\_days\_tall}\SpecialCharTok{$}\NormalTok{id,}
\NormalTok{                                                steps\_sum\_days\_tall}\SpecialCharTok{$}\NormalTok{activity\_day,}
\NormalTok{                                                steps\_sum\_days\_tall}\SpecialCharTok{$}\NormalTok{steps\_total) }\SpecialCharTok{+}
  \FunctionTok{labs}\NormalTok{(}\AttributeTok{title =} \StringTok{"Correlation between Step Counts and Time"}\NormalTok{)}
\FunctionTok{print}\NormalTok{(viz\_mean\_daily\_steps\_vs\_time)}
\end{Highlighting}
\end{Shaded}

\includegraphics{BellabeatCaseStudy_files/figure-latex/viz_mean_daily_steps_vs_time-1.pdf}

Preliminary Findings:

\begin{itemize}
\tightlist
\item
  As with walking types, the majority of users actually did fewer steps
  over time
\end{itemize}

\hypertarget{do-people-engage-in-stationary-or-mobile-exercise}{%
\paragraph{Do people engage in stationary or mobile
exercise?}\label{do-people-engage-in-stationary-or-mobile-exercise}}

Examples of stationary exercise include gym-based activities like
treadmill running and weight-lifting. Examples of mobile exercise
include outdoor activities like jogging, but also indoor activities like
squash or, depending on the accuracy of the FitBit distance tracking,
gym workouts based on rotation through various equipment and exercises
spaced around the venue. Determining if users have a preference for one
activity type over another can help guide the marketing for the Ivy.

I analysed stationary versus mobile activity preferences by plotting the
data for Active Minutes against Active Distance. The theory was that the
more stationary exercise users do, the weaker the correlation between
active time and active distance should be, because users are not moving
while exercising (i.e.~machine-based cardio and resistance training).

\begin{Shaded}
\begin{Highlighting}[]
\NormalTok{distance\_vs\_active }\OtherTok{\textless{}{-}}\NormalTok{ activity\_sum\_days\_wide }\SpecialCharTok{\%\textgreater{}\%}
  \FunctionTok{select}\NormalTok{(id,}
\NormalTok{         activity\_day,}
\NormalTok{         distance\_moderately\_active,}
\NormalTok{         distance\_very\_active,}
\NormalTok{         minutes\_fairly\_active,}
\NormalTok{         minutes\_very\_active) }\SpecialCharTok{\%\textgreater{}\%}
  \FunctionTok{mutate}\NormalTok{(}\AttributeTok{distance\_active =}\NormalTok{ distance\_moderately\_active }\SpecialCharTok{+}\NormalTok{ distance\_very\_active,}
         \AttributeTok{minutes\_active =}\NormalTok{ minutes\_fairly\_active }\SpecialCharTok{+}\NormalTok{ minutes\_very\_active) }\SpecialCharTok{\%\textgreater{}\%}
  \FunctionTok{group\_by}\NormalTok{(id) }\SpecialCharTok{\%\textgreater{}\%}
  \FunctionTok{summarize}\NormalTok{(}\AttributeTok{mean\_distance\_active =} \FunctionTok{mean}\NormalTok{(distance\_active),}
            \AttributeTok{mean\_minutes\_active =} \FunctionTok{mean}\NormalTok{(minutes\_active))}
\end{Highlighting}
\end{Shaded}

\begin{Shaded}
\begin{Highlighting}[]
\NormalTok{viz\_distance\_vs\_activity }\OtherTok{\textless{}{-}} \FunctionTok{scatter\_with\_LOBF}\NormalTok{(distance\_vs\_active}\SpecialCharTok{$}\NormalTok{mean\_minutes\_active,}
\NormalTok{                       distance\_vs\_active}\SpecialCharTok{$}\NormalTok{mean\_distance\_active,}
                       \FunctionTok{c}\NormalTok{(}\StringTok{"Mean Daily Active Minutes"}\NormalTok{, }\StringTok{"Mean Daily Active Distance"}\NormalTok{)) }\SpecialCharTok{+}
  \FunctionTok{labs}\NormalTok{(}\AttributeTok{caption =} \StringTok{"}\SpecialCharTok{\textbackslash{}"}\StringTok{Active}\SpecialCharTok{\textbackslash{}"}\StringTok{ refers to time spent either Fairly or Very Active"}\NormalTok{) }\SpecialCharTok{+}
  \FunctionTok{scale\_y\_continuous}\NormalTok{(}\AttributeTok{breaks =} \FunctionTok{seq}\NormalTok{(}\DecValTok{0}\NormalTok{,}\DecValTok{10}\NormalTok{,}\AttributeTok{by=}\DecValTok{1}\NormalTok{))}
\FunctionTok{print}\NormalTok{(viz\_distance\_vs\_activity)}
\end{Highlighting}
\end{Shaded}

\includegraphics{BellabeatCaseStudy_files/figure-latex/viz_distance_vs_intensity-1.pdf}

Preliminary Findings:

\begin{itemize}
\tightlist
\item
  Active Minutes correlate strongly with Active distance, with R\^{}2 =
  0.77.
\item
  Clearly absent from the plot is a significant number of users with
  high Active Minutes but low Active Distance
\item
  Users typically log an additional 3.6 km of distance for every hour of
  Active time: to achieve this in a gym setting a user would have to
  follow a typical pattern resembling, for example, two one-minute
  stationary resistance training sets, followed by one minute moving
  around the gym at approximately 10.8 kilometers per hour, which seems
  unrealistic
\end{itemize}

Overall, based on the strong correlation between Active Time and Active
distance, and the scale of the distances covered by active users, it
would appear that the cohort engages in a variety of mobile exercise
activities. The exact breakdown of mobile vs.~stationary activities has
not been determined, but I believe it is safe to say that the marketing
for the Ivy should represent mobile and outdoor activities more strongly
over indoor, gym-based, or other stationary activities.

\hypertarget{recommendations}{%
\subsubsection{Recommendations}\label{recommendations}}

\begin{itemize}
\tightlist
\item
  To appeal to the FitBit market, the marketing for the Ivy's exercise
  tracking should highlight a broad range of activities ranging from
  gentle to more intense. The marketing should not target
  highly-athletic groups, as these represent only a small part of the
  user base.
\item
  The marketing for the Ivy should avoid any claims that users will
  increase their activity levels or their step counts as a result of
  their purchasing the device, as the data does not support this claim.
\item
  The cohort does appear to engage in outdoor activities that cover long
  distances such as running, as opposed to stationary activities such as
  gym workouts, so the marketing can highlight activities like light
  jogging, walking dogs, and cycling
\end{itemize}

\hypertarget{readiness-score}{%
\subsection{Readiness Score}\label{readiness-score}}

\hypertarget{feature-overview-1}{%
\subsubsection{Feature Overview}\label{feature-overview-1}}

Subtitle: ``Includes Resting Heart Rate, Respiratory Rate, and Cardiac
Coherence functions''

Description:

\begin{itemize}
\tightlist
\item
  ``While you sleep at night and your body is in its calmest state, Ivy
  measures your resting heart rate, respiratory rate, and cardiac
  coherence.''
\item
  ``Resting heart rate (RHR) is the number of heartbeats per minute,
  measured when the body is fully calm during the night.''
\item
  ``The cardiac coherence parameter shows how your heart rate
  variability and breathing rate are synchronized.''
\end{itemize}

The Readiness Score is one of several features of the Ivy that can be
marketed as a value-add over the FitBit devices in the data set. The
feature makes use of several functions of the device, some of which the
FitBits in the data set do not have, and builds on other features
already available, like Respiratory Rate and Resting Heart Rate

The FitBit devices in the dataset don't log any direct equivalent to the
Readiness Score, nor does the documentation for the devices describe any
comparable feature.

While the Blaze does have a breathing-detection function, it does not
appear to be tracked over time in the data, and the Blaze manual
describes it used only for ``Guided Breathing'' activities, which track
respiratory rate only during the activity.

The Surge manual also mentions that (``Wearing your Surge while you
sleep has the added benefit of making the Resting Heart Rate
measurements on your dashboard more
accurate''){[}\url{https://myhelp.fitbit.com/resource/manual_surge_en_US}{]},
so it appears to have that feature as well, but it is not built into an
overall ``Readiness Score''

This makes a potentially valuable function of the device, but also makes
it difficult to investigate the existing data for evidence of how users
might make use of this feature.

One aspect we can analyse is whether or not users currently wear their
devices to bed.

\hypertarget{analysis-1}{%
\subsubsection{Analysis}\label{analysis-1}}

\hypertarget{do-people-sleep-with-their-fitbits-on}{%
\paragraph{Do people sleep with their FitBits
on?}\label{do-people-sleep-with-their-fitbits-on}}

Nighttime usage of FitBits is particularly relevant to the Readiness
Score, as that function does not work unless the user wears their device
to bed. If people are already frequently wearing their devices to bed,
then the marketing can leverage this tendency to push the feature as an
improvement on existing features. On the other hand, if users are not
wearing their devices overnight, this may indicate that users are not
interested in existing overnight-monitoring functions, or that there are
other factors that make overnight wear unappealing, which marketing
teams will need to take into account.

\hypertarget{which-models-can-tell-when-the-user-is-sleeping}{%
\subparagraph{Which models can tell when the user is
sleeping?}\label{which-models-can-tell-when-the-user-is-sleeping}}

As of April 2016, the following wristband FitBits were available:

\begin{longtable}[]{@{}
  >{\raggedright\arraybackslash}p{(\columnwidth - 6\tabcolsep) * \real{0.2787}}
  >{\centering\arraybackslash}p{(\columnwidth - 6\tabcolsep) * \real{0.2459}}
  >{\centering\arraybackslash}p{(\columnwidth - 6\tabcolsep) * \real{0.3443}}
  >{\raggedright\arraybackslash}p{(\columnwidth - 6\tabcolsep) * \real{0.1311}}@{}}
\toprule\noalign{}
\begin{minipage}[b]{\linewidth}\raggedright
Model
\end{minipage} & \begin{minipage}[b]{\linewidth}\centering
Sleep Logging
\end{minipage} & \begin{minipage}[b]{\linewidth}\centering
Heart-rate Tracking
\end{minipage} & \begin{minipage}[b]{\linewidth}\raggedright
Source
\end{minipage} \\
\midrule\noalign{}
\endhead
\bottomrule\noalign{}
\endlastfoot
Alta & Yes & No &
\href{https://www.fitbit.com/content/assets/help/manuals/manual_alta_en_US.pdf}{User
Manual} \\
Blaze & Yes & Yes &
\href{https://staticcs.fitbit.com/content/assets/help/manuals/manual_blaze_en_US.pdf}{User
Manual} \\
Charge & Yes & No &
\href{https://staticcs.fitbit.com/content/assets/help/manuals/manual_charge_en_US.pdf}{User
Manual} \\
Charge HR & Yes & Yes &
\href{https://staticcs.fitbit.com/content/assets/help/manuals/manual_charge_hr_en_US.pdf}{User
Manual} \\
Flex 1 & Yes & No &
\href{https://help.fitbit.com/manuals/manual_flex_en_US.pdf}{User
Manual} \\
Flex 2 & Yes & No &
\href{https://help.fitbit.com/manuals/manual_flex_2_en_US.pdf}{User
Manual} \\
Surge & Yes & Yes &
\href{https://myhelp.fitbit.com/resource/manual_surge_en_US}{User
Manual} \\
\end{longtable}

Preliminary Findings:

\begin{itemize}
\tightlist
\item
  Sleep logging is available on all devices, so sleep log data should
  indicate how the overall cohort uses this feature
\item
  Heart-rate logging is limited to later devices, so heartrate data will
  only indicate how that subset of the cohort uses that feature
\end{itemize}

\hypertarget{sleep-logs-analysis}{%
\subparagraph{Sleep Logs Analysis}\label{sleep-logs-analysis}}

For each user, I calculated the number of nights where the total sleep
time logged was greater than 6 hours: this value was picked as being
close enough to the recommended 8 hours of sleep to indicate the user
did actually go to bed with their device on, but not so high that it
would not pick up data from users who don't get a full night's sleep.

\begin{Shaded}
\begin{Highlighting}[]
\NormalTok{hours\_asleep }\OtherTok{\textless{}{-}} \DecValTok{6}
\NormalTok{total\_nights }\OtherTok{\textless{}{-}} \DecValTok{33}
\NormalTok{nights\_logged\_by\_sleep\_log }\OtherTok{\textless{}{-}}\NormalTok{ sleep\_sum\_days\_wide }\SpecialCharTok{\%\textgreater{}\%}
  \FunctionTok{group\_by}\NormalTok{(id) }\SpecialCharTok{\%\textgreater{}\%}
  \FunctionTok{summarize}\NormalTok{(}\AttributeTok{count\_logged\_asleep =} \FunctionTok{sum}\NormalTok{(minutes\_asleep\_total }\SpecialCharTok{/} \DecValTok{60} \SpecialCharTok{\textgreater{}}\NormalTok{ hours\_asleep),}
            \AttributeTok{count\_logged\_in\_bed =} \FunctionTok{sum}\NormalTok{(minutes\_in\_bed\_total }\SpecialCharTok{/} \DecValTok{60} \SpecialCharTok{\textgreater{}}\NormalTok{ hours\_asleep),}
            \AttributeTok{pct\_logged\_asleep =}\NormalTok{ count\_logged\_asleep }\SpecialCharTok{/}\NormalTok{ total\_nights,}
            \AttributeTok{pct\_logged\_in\_bed =}\NormalTok{ count\_logged\_in\_bed }\SpecialCharTok{/}\NormalTok{ total\_nights)}

\FunctionTok{plot\_histo\_pareto}\NormalTok{(nights\_logged\_by\_sleep\_log}\SpecialCharTok{$}\NormalTok{pct\_logged\_in\_bed, }\AttributeTok{bin\_width =} \FloatTok{0.05}\NormalTok{) }\SpecialCharTok{+}
  \FunctionTok{labs}\NormalTok{(}\AttributeTok{title =} \StringTok{"Percentage of Nights Logged In Bed {-} Logged Users Only"}\NormalTok{,}
       \AttributeTok{subtitle =} \StringTok{"Source: Sleep Log Data"}\NormalTok{)}
\end{Highlighting}
\end{Shaded}

\includegraphics{BellabeatCaseStudy_files/figure-latex/analyse_night_usage_sleep_logs-1.pdf}

\begin{Shaded}
\begin{Highlighting}[]
\CommentTok{\# Add missing ID values to the dataset to analyse the full cohort}
\NormalTok{missing\_ids }\OtherTok{\textless{}{-}} \FunctionTok{setdiff}\NormalTok{(}\FunctionTok{unique}\NormalTok{(activity\_sum\_days\_wide}\SpecialCharTok{$}\NormalTok{id), nights\_logged\_by\_sleep\_log}\SpecialCharTok{$}\NormalTok{id)}
\ControlFlowTok{for}\NormalTok{ (id }\ControlFlowTok{in}\NormalTok{ missing\_ids) \{}
\NormalTok{  nights\_logged\_by\_sleep\_log }\OtherTok{\textless{}{-}}\NormalTok{ nights\_logged\_by\_sleep\_log }\SpecialCharTok{\%\textgreater{}\%}
    \FunctionTok{rbind}\NormalTok{(.,}\FunctionTok{data.frame}\NormalTok{(}
          \AttributeTok{id =}\NormalTok{ id,}
          \AttributeTok{count\_logged\_asleep =} \DecValTok{0}\NormalTok{,}
          \AttributeTok{count\_logged\_in\_bed =} \DecValTok{0}\NormalTok{,}
          \AttributeTok{pct\_logged\_asleep =} \FloatTok{0.0}\NormalTok{,}
          \AttributeTok{pct\_logged\_in\_bed =} \FloatTok{0.0}\NormalTok{))}
\NormalTok{\}}
\end{Highlighting}
\end{Shaded}

\begin{Shaded}
\begin{Highlighting}[]
\FunctionTok{plot\_histo\_pareto}\NormalTok{(nights\_logged\_by\_sleep\_log}\SpecialCharTok{$}\NormalTok{pct\_logged\_in\_bed, }\AttributeTok{bin\_width =} \FloatTok{0.05}\NormalTok{) }\SpecialCharTok{+}
  \FunctionTok{labs}\NormalTok{(}\AttributeTok{title =} \StringTok{"Percentage of Nights Logged In Bed {-} All Users"}\NormalTok{,}
       \AttributeTok{subtitle =} \StringTok{"Source: Sleep Log Data"}\NormalTok{)}
\end{Highlighting}
\end{Shaded}

\includegraphics{BellabeatCaseStudy_files/figure-latex/viz_night_usage_sleep_logs-1.pdf}

Preliminary Findings:

\begin{itemize}
\tightlist
\item
  Out of all users in the data set, very few users are logging, with
  36\% (12 users) logging between 0\% and 5\% of nights, and a majority
  of 51\% (17 users) logging 10\% of nights or fewer
\item
  Out of users with sleep log data, the results are more evenly spread,
  with 37\% (9 users) logging 0\% to 15\% of nights, and another 42\%
  (10 users) logging 70\% to 90\% of nights.
\end{itemize}

Given that the FitBit devices available at the time the dataset was
recorded all have sleep-tracking functionality, these findings indicate
that the majority of users are not making use of this function. One
possible reason for this is indicated by the manual for the FitBit
Surge, which informs the user that
\href{https://myhelp.fitbit.com/resource/manual_surge_en_US}{``wearing
your tracker 24/7 does not allow your skin to breathe''}

\hypertarget{heart-rate-analysis}{%
\subparagraph{Heart-rate Analysis}\label{heart-rate-analysis}}

The sleep logs data gives a good indication of whether users are
interested in sleep logging data, but it doesn't directly confirm
whether people are wearing their devices to bed. To investigate this, I
also analysed the heart-rate data for each user. This is by far the
largest and least-processed data set, but it has the advantage of only
generating data if the device is physically contacting the user.

Taking the brute-force approach to plotting the data for each user makes
it much easier to visualise the patterns in user behaviour:

TODO: Widen out this plot

\begin{Shaded}
\begin{Highlighting}[]
\NormalTok{data }\OtherTok{\textless{}{-}}\NormalTok{ heartrate\_src\_seconds\_tall }\SpecialCharTok{\%\textgreater{}\%}
    \FunctionTok{mutate}\NormalTok{(}\AttributeTok{is\_nighttime =} \FunctionTok{hour}\NormalTok{(heart\_rate\_second) }\SpecialCharTok{\textgreater{}=} \DecValTok{22} \SpecialCharTok{|} \FunctionTok{hour}\NormalTok{(heart\_rate\_second) }\SpecialCharTok{\textless{}} \DecValTok{6}\NormalTok{)}
\NormalTok{p }\OtherTok{\textless{}{-}} \FunctionTok{ggplot}\NormalTok{(data) }\SpecialCharTok{+}
  \FunctionTok{geom\_point}\NormalTok{(}\FunctionTok{aes}\NormalTok{(}\AttributeTok{x =}\NormalTok{ heart\_rate\_second,}
                 \AttributeTok{y =}\NormalTok{ heart\_rate,}
                 \AttributeTok{color =}\NormalTok{ is\_nighttime),}
             \AttributeTok{size =} \FloatTok{0.1}\NormalTok{,}
             \AttributeTok{shape =} \DecValTok{3}\NormalTok{) }\SpecialCharTok{+}
  \FunctionTok{facet\_wrap}\NormalTok{(}\FunctionTok{vars}\NormalTok{(id))}
\FunctionTok{print}\NormalTok{(p)}


\FunctionTok{ggsave}\NormalTok{(}\StringTok{"viz\_all\_heartrate\_data.png"}\NormalTok{,}
       \AttributeTok{plot =}\NormalTok{ p,}
       \AttributeTok{device =} \StringTok{"png"}\NormalTok{,}
       \AttributeTok{width =} \DecValTok{3840}\NormalTok{,}
       \AttributeTok{height =} \DecValTok{2160}\NormalTok{,}
       \AttributeTok{units =} \StringTok{"px"}\NormalTok{)}

\FunctionTok{rm}\NormalTok{(data)}
\end{Highlighting}
\end{Shaded}

Once R is done chewing through the data, we can see that all users have
clear gaps in their heart-rate logs.

The heart rate data takes the form of individual polls recording the
user's momentary heart rate. This presents a challenge compared to using
the summary data, in that these individual logs must be processed to
determine when the user was wearing their device continuously. Counting
the number of logs over a given time period will not work, as the time
between logs varies even when the device is worn. Polling times less
than 60 seconds, for example, are typically between 1 and 20 seconds, as
shown below:

\begin{Shaded}
\begin{Highlighting}[]
\NormalTok{heartrate\_polling\_variance }\OtherTok{\textless{}{-}}\NormalTok{ heartrate\_src\_seconds\_tall }\SpecialCharTok{\%\textgreater{}\%}
  \FunctionTok{group\_by}\NormalTok{(id) }\SpecialCharTok{\%\textgreater{}\%}
  \FunctionTok{mutate}\NormalTok{(}\AttributeTok{polling\_diff =} \FunctionTok{as.double}\NormalTok{(}\FunctionTok{difftime}\NormalTok{(heart\_rate\_second, }\FunctionTok{lag}\NormalTok{(heart\_rate\_second), }\AttributeTok{units =} \StringTok{"secs"}\NormalTok{))) }\SpecialCharTok{\%\textgreater{}\%}
  \FunctionTok{filter}\NormalTok{(polling\_diff }\SpecialCharTok{\textless{}} \DecValTok{60}\NormalTok{)}
\end{Highlighting}
\end{Shaded}

\begin{Shaded}
\begin{Highlighting}[]
\FunctionTok{ggplot}\NormalTok{(heartrate\_polling\_variance) }\SpecialCharTok{+}
  \FunctionTok{geom\_boxplot}\NormalTok{(}\FunctionTok{aes}\NormalTok{(}\AttributeTok{y =}\NormalTok{ polling\_diff)) }\SpecialCharTok{+}
  \FunctionTok{scale\_y\_continuous}\NormalTok{(}\AttributeTok{breaks =} \FunctionTok{seq}\NormalTok{(}\DecValTok{0}\NormalTok{,}\DecValTok{60}\NormalTok{,}\AttributeTok{by=}\DecValTok{5}\NormalTok{))}
\end{Highlighting}
\end{Shaded}

\includegraphics{BellabeatCaseStudy_files/figure-latex/viz_heartrate_polling_variance-1.pdf}

This means that even choosing the median value of 5 seconds could result
in an inferred ``time logged'' value that's out by a factor of 5.

To build out the data, I decided on an approach where the time between
logs was used to infer the time logged, but only within an acceptable
time difference. If one log follows its predecessor by less than this
difference, the time between the two is considered ``Logged'' time: if
the delay is greater, we assume the device was taken off for that time.

From the polling variance analysis boxplot, we saw that the median,
75th-percentile, and Maximum values were 5 seconds, 10 seconds, and 17
seconds, respectively. Given this range, I selected 20 seconds as the
maximum acceptable time between polls.

\begin{Shaded}
\begin{Highlighting}[]
\CommentTok{\# Find nights where enough heart{-}rate data was logged to imply the user slept with their FitBit on}

\NormalTok{sec2hour }\OtherTok{\textless{}{-}} \DecValTok{1} \SpecialCharTok{/} \DecValTok{3600}
\CommentTok{\# Sleep range set to between 10pm and 6am}
\NormalTok{sleep\_range\_stt\_hour }\OtherTok{\textless{}{-}} \DecValTok{22}
\NormalTok{sleep\_range\_end\_hour }\OtherTok{\textless{}{-}} \DecValTok{6}
\CommentTok{\# Anything more than this time between logs is considered a removal (typical poll rate is 1{-}20 seconds)}
\NormalTok{max\_poll\_gap\_secs }\OtherTok{\textless{}{-}} \DecValTok{20}
\CommentTok{\# Minimum six hours must be logged to be counted}
\NormalTok{min\_hours\_logged }\OtherTok{\textless{}{-}} \DecValTok{6}
\NormalTok{nights\_total }\OtherTok{\textless{}{-}} \DecValTok{33}

\NormalTok{nights\_logged\_by\_heart\_rate }\OtherTok{\textless{}{-}}\NormalTok{ heartrate\_src\_seconds\_tall }\SpecialCharTok{\%\textgreater{}\%}
  \CommentTok{\# For logs in the AM, the night{-}of is set to the day before}
  \FunctionTok{mutate}\NormalTok{(}\AttributeTok{night\_of\_yday =} \FunctionTok{ifelse}\NormalTok{(}\FunctionTok{hour}\NormalTok{(heart\_rate\_second) }\SpecialCharTok{\textgreater{}=}\NormalTok{ sleep\_range\_stt\_hour, }\FunctionTok{yday}\NormalTok{(heart\_rate\_second),}
                                \FunctionTok{ifelse}\NormalTok{(}\FunctionTok{hour}\NormalTok{(heart\_rate\_second) }\SpecialCharTok{\textless{}}\NormalTok{ sleep\_range\_end\_hour, }\FunctionTok{yday}\NormalTok{(heart\_rate\_second) }\SpecialCharTok{{-}} \DecValTok{1}\NormalTok{,}
                                       \ConstantTok{NA}\NormalTok{))) }\SpecialCharTok{\%\textgreater{}\%}
  \CommentTok{\# Only consider logs from the "sleep" range}
  \FunctionTok{filter}\NormalTok{(}\SpecialCharTok{!}\FunctionTok{is.na}\NormalTok{(night\_of\_yday)) }\SpecialCharTok{\%\textgreater{}\%}
  \CommentTok{\# Group by id so that difftime only compares timestamps from the same user}
  \FunctionTok{group\_by}\NormalTok{(id) }\SpecialCharTok{\%\textgreater{}\%}
  \FunctionTok{mutate}\NormalTok{(}\AttributeTok{time\_since\_last\_poll =} \FunctionTok{as.double}\NormalTok{(}\FunctionTok{difftime}\NormalTok{(heart\_rate\_second, }\FunctionTok{lag}\NormalTok{(heart\_rate\_second), }\AttributeTok{units =} \StringTok{"secs"}\NormalTok{))) }\SpecialCharTok{\%\textgreater{}\%}
  \CommentTok{\# Remove any logs that are followed by too long a gap}
  \FunctionTok{filter}\NormalTok{(time\_since\_last\_poll }\SpecialCharTok{\textless{}=}\NormalTok{ max\_poll\_gap\_secs) }\SpecialCharTok{\%\textgreater{}\%}
  \FunctionTok{group\_by}\NormalTok{(id, night\_of\_yday) }\SpecialCharTok{\%\textgreater{}\%}
  \CommentTok{\# Find the total time logged as the sum of time between valid logs}
  \FunctionTok{summarize}\NormalTok{(}\AttributeTok{logged\_time\_hours =} \FunctionTok{sum}\NormalTok{(time\_since\_last\_poll) }\SpecialCharTok{*}\NormalTok{ sec2hour) }\SpecialCharTok{\%\textgreater{}\%}
  \CommentTok{\# Count up the nights where each user logged sufficient time to be considered asleep with their fitbit}
  \FunctionTok{group\_by}\NormalTok{(id) }\SpecialCharTok{\%\textgreater{}\%}
  \FunctionTok{summarize}\NormalTok{(}\AttributeTok{nights\_logged =} \FunctionTok{sum}\NormalTok{(logged\_time\_hours }\SpecialCharTok{\textgreater{}}\NormalTok{ min\_hours\_logged),}
            \AttributeTok{nights\_logged\_pct =}\NormalTok{ nights\_logged }\SpecialCharTok{/}\NormalTok{ nights\_total)}
\end{Highlighting}
\end{Shaded}

\begin{Shaded}
\begin{Highlighting}[]
\CommentTok{\# Plot histogram showing who logged what fraction of their nights}
\FunctionTok{plot\_histo\_pareto}\NormalTok{(nights\_logged\_by\_heart\_rate}\SpecialCharTok{$}\NormalTok{nights\_logged\_pct, }\AttributeTok{bin\_width =} \FloatTok{0.05}\NormalTok{) }\SpecialCharTok{+}
  \FunctionTok{labs}\NormalTok{(}\AttributeTok{title =} \StringTok{"Percentage of Nights Logged In Bed {-} HR Users Only"}\NormalTok{,}
       \AttributeTok{subtitle =} \StringTok{"Source: Heart Rate Data"}\NormalTok{)}
\end{Highlighting}
\end{Shaded}

\includegraphics{BellabeatCaseStudy_files/figure-latex/viz_night_usage_heartrate-1.pdf}

\begin{Shaded}
\begin{Highlighting}[]
\CommentTok{\# }
\CommentTok{\# \# Add missing ID values to the dataset to analyse the full cohort}
\CommentTok{\# missing\_ids \textless{}{-} setdiff(unique(activity\_sum\_days\_wide$id), nights\_logged\_by\_heart\_rate$id)}
\CommentTok{\# for (id in missing\_ids) \{}
\CommentTok{\#   nights\_logged\_by\_heart\_rate \textless{}{-} nights\_logged\_by\_heart\_rate \%\textgreater{}\%}
\CommentTok{\#     rbind(.,data.frame(}
\CommentTok{\#           id = id,}
\CommentTok{\#           nights\_logged = 0,}
\CommentTok{\#           nights\_logged\_pct = 0.0))}
\CommentTok{\# \}}
\CommentTok{\# }
\CommentTok{\# \# Plot histogram showing who logged what fraction of their nights}
\CommentTok{\# plot\_histo\_pareto(nights\_logged\_by\_heart\_rate$nights\_logged\_pct, bin\_width = 0.05) +}
\CommentTok{\#   labs(title = "Percentage of Nights Logged In Bed {-} All Users",}
\CommentTok{\#        subtitle = "Source: Heart Rate Data")}
\end{Highlighting}
\end{Shaded}

Given the small number of heart-rate users, I also compared the the
heart-rate results directly to the sleep log results to see if there was
any difference in the nights logged for individual IDs:

\begin{Shaded}
\begin{Highlighting}[]
\NormalTok{sleep\_data\_merged }\OtherTok{\textless{}{-}} \FunctionTok{merge}\NormalTok{(nights\_logged\_by\_sleep\_log, nights\_logged\_by\_heart\_rate, }\AttributeTok{by =} \StringTok{"id"}\NormalTok{, }\AttributeTok{all =} \ConstantTok{TRUE}\NormalTok{) }\SpecialCharTok{\%\textgreater{}\%}
  \FunctionTok{select}\NormalTok{(id, pct\_logged\_asleep, nights\_logged\_pct) }\SpecialCharTok{\%\textgreater{}\%}
  \FunctionTok{rename}\NormalTok{(}\AttributeTok{sleep\_logs =}\NormalTok{ pct\_logged\_asleep,}
         \AttributeTok{heartrate\_logs =}\NormalTok{ nights\_logged\_pct)}
\NormalTok{sleep\_data\_merged }\OtherTok{\textless{}{-}}\NormalTok{ tidyr}\SpecialCharTok{::}\FunctionTok{gather}\NormalTok{(sleep\_data\_merged, }\AttributeTok{key =} \StringTok{"Variable"}\NormalTok{, }\AttributeTok{value =} \StringTok{"Value"}\NormalTok{, }\SpecialCharTok{{-}}\NormalTok{id)}

\FunctionTok{ggplot}\NormalTok{(sleep\_data\_merged, }\FunctionTok{aes}\NormalTok{(}\AttributeTok{x =}\NormalTok{ id, }\AttributeTok{y =}\NormalTok{ Value, }\AttributeTok{fill =}\NormalTok{ Variable)) }\SpecialCharTok{+}
  \FunctionTok{geom\_bar}\NormalTok{(}\AttributeTok{stat =} \StringTok{"identity"}\NormalTok{, }\AttributeTok{position =} \StringTok{"dodge"}\NormalTok{, }\AttributeTok{width =} \FloatTok{0.7}\NormalTok{, }\AttributeTok{color =} \StringTok{"black"}\NormalTok{) }\SpecialCharTok{+}
  \FunctionTok{theme\_minimal}\NormalTok{() }\SpecialCharTok{+}
  \FunctionTok{theme}\NormalTok{(}\AttributeTok{axis.text.x =} \FunctionTok{element\_text}\NormalTok{(}\AttributeTok{angle =} \DecValTok{45}\NormalTok{, }\AttributeTok{hjust =} \DecValTok{1}\NormalTok{)) }\SpecialCharTok{+}
  \FunctionTok{scale\_y\_continuous}\NormalTok{(}\AttributeTok{breaks =} \FunctionTok{seq}\NormalTok{(}\DecValTok{0}\NormalTok{, }\DecValTok{1}\NormalTok{, }\AttributeTok{by =} \FloatTok{0.1}\NormalTok{)) }\SpecialCharTok{+}
  \FunctionTok{labs}\NormalTok{(}\AttributeTok{title =} \StringTok{"Nights Logged: Sleep Logs vs. Heartrate Logs"}\NormalTok{,}
       \AttributeTok{x =} \StringTok{"ID"}\NormalTok{,}
       \AttributeTok{y =} \StringTok{"\% of Nights Logged"}\NormalTok{,}
       \AttributeTok{fill =} \StringTok{"Source"}\NormalTok{)}
\end{Highlighting}
\end{Shaded}

\includegraphics{BellabeatCaseStudy_files/figure-latex/viz_sleep_logs_vs_heartrate_logs-1.pdf}

For those IDs present in both data sets, the results were almost
identical, with only three users showing variation equating to 2-3
nights, indicating the method is valid.

Preliminary Findings:

\begin{itemize}
\tightlist
\item
  Out of all users, 73\% (24 users) did not log any heartrate-based
  sleep logs.
\item
  Out of all users, 42\% (14 users) logged heart-rate data
\item
  Out of those users, the results were split, with 50\% (7 users)
  logging sleep on 35\% to 90\% of nights, and the other 50\% logging
  sleep on 0\% to 15\% of nights
\item
  The heart-rate-based analysis produces the same results for individual
  users as the sleep-log-based analysis
\item
  Overall, the results reinforce the findings of the sleep-log analysis,
  but only for the smaller sub-set of users that have devices with
  heart-rate tracking.
\end{itemize}

\hypertarget{recommendations-1}{%
\subsubsection{Recommendations}\label{recommendations-1}}

\begin{itemize}
\tightlist
\item
  Overnight usage of FitBits appears to be split between a large
  majority who rarely wear their devices overnight, and a smaller
  minority who wear more frequently, up to 90\% of the time
\item
  The Readiness Score can therefore be marketed as a value-add feature
  that produces more useful information when worn to bed than competing
  products
\item
  The non-allergenic materials used in the Ivy should be used to
  highlight the comfort aspects of the device when marketing features
  that require overnight wearing, as discomfort from overnight wear may
  be one of the factors contributing to the low utilisation of sleep
  logging in existing products.
\end{itemize}

\hypertarget{sleep-tracking}{%
\subsection{Sleep Tracking}\label{sleep-tracking}}

\hypertarget{feature-overview-2}{%
\subsubsection{Feature Overview}\label{feature-overview-2}}

\hypertarget{analysis-2}{%
\subsubsection{Analysis}\label{analysis-2}}

\hypertarget{how-many-people-used-the-sleep-tracking-feature}{%
\paragraph{How many people used the sleep tracking
feature?}\label{how-many-people-used-the-sleep-tracking-feature}}

From the previous analysis for the Readiness Score feature, we know that
sleep tracking is not commonly used, with a majority of 51\% (17 users)
logging 10\% of nights or fewer, and the remainder spread out over a
range up to 90\% of nights.

\hypertarget{what-quantity-of-sleep-are-users-getting}{%
\paragraph{What quantity of sleep are users
getting?}\label{what-quantity-of-sleep-are-users-getting}}

\begin{itemize}
\tightlist
\item
  Sleep is ranked as Awake, Restless, or Asleep
\item
  Quantity and quantity can be tracked as time spent Asleep as a
  percentage of total logged
\item
  Quality can also be tracked as
\end{itemize}

\begin{Shaded}
\begin{Highlighting}[]
\NormalTok{sleep\_sum\_quant\_qual }\OtherTok{\textless{}{-}}\NormalTok{ sleep\_src\_mins\_tall }\SpecialCharTok{\%\textgreater{}\%}
  \CommentTok{\# Set the date for sleep logs in the AM to the day before, i.e. the Night Of [date]}
  \FunctionTok{mutate}\NormalTok{(}\AttributeTok{sleep\_date =} \FunctionTok{as.Date}\NormalTok{(}\FunctionTok{ifelse}\NormalTok{(}\FunctionTok{hour}\NormalTok{(sleep\_minute) }\SpecialCharTok{\textgreater{}} \DecValTok{12}\NormalTok{,}
                                     \FunctionTok{as.Date}\NormalTok{(sleep\_minute),}
                                     \FunctionTok{as.Date}\NormalTok{(sleep\_minute) }\SpecialCharTok{{-}} \DecValTok{1}\NormalTok{))) }\SpecialCharTok{\%\textgreater{}\%}
  \CommentTok{\# mutate(sleep\_date = as.Date(sleep\_date)) \%\textgreater{}\%}
  \FunctionTok{group\_by}\NormalTok{(id, sleep\_date) }\SpecialCharTok{\%\textgreater{}\%}
  \FunctionTok{summarize}\NormalTok{(}\AttributeTok{minutes\_awake    =} \FunctionTok{sum}\NormalTok{(sleep\_rank }\SpecialCharTok{==} \StringTok{"Awake"}\NormalTok{),}
            \AttributeTok{minutes\_restless =} \FunctionTok{sum}\NormalTok{(sleep\_rank }\SpecialCharTok{==} \StringTok{"Restless"}\NormalTok{),}
            \AttributeTok{minutes\_asleep   =} \FunctionTok{sum}\NormalTok{(sleep\_rank }\SpecialCharTok{==} \StringTok{"Asleep"}\NormalTok{),}
            \AttributeTok{minutes\_total    =}\NormalTok{ minutes\_awake }\SpecialCharTok{+}\NormalTok{ minutes\_restless }\SpecialCharTok{+}\NormalTok{ minutes\_asleep,}
            \AttributeTok{pct\_awake        =}\NormalTok{ minutes\_awake }\SpecialCharTok{/}\NormalTok{ minutes\_total,}
            \AttributeTok{pct\_restless     =}\NormalTok{ minutes\_restless }\SpecialCharTok{/}\NormalTok{ minutes\_total,}
            \AttributeTok{pct\_asleep       =}\NormalTok{ minutes\_asleep }\SpecialCharTok{/}\NormalTok{ minutes\_total)}

\NormalTok{sleep\_sum\_quant\_qual\_mean }\OtherTok{\textless{}{-}}\NormalTok{ sleep\_sum\_quant\_qual }\SpecialCharTok{\%\textgreater{}\%}
  \FunctionTok{group\_by}\NormalTok{(id) }\SpecialCharTok{\%\textgreater{}\%}
  \FunctionTok{summarize}\NormalTok{(}\FunctionTok{across}\NormalTok{(}\AttributeTok{.cols =} \SpecialCharTok{{-}}\NormalTok{sleep\_date, \textbackslash{}(x) }\FunctionTok{mean}\NormalTok{(x, }\AttributeTok{na.rm =} \ConstantTok{TRUE}\NormalTok{), }\AttributeTok{.names =} \StringTok{"\{.col\}\_mean"}\NormalTok{)) }\SpecialCharTok{\%\textgreater{}\%}
  \FunctionTok{mutate}\NormalTok{(}\AttributeTok{hours\_total\_mean =}\NormalTok{ minutes\_total\_mean }\SpecialCharTok{/} \DecValTok{60}\NormalTok{)}
\end{Highlighting}
\end{Shaded}

\begin{Shaded}
\begin{Highlighting}[]
\FunctionTok{plot\_histo\_pareto}\NormalTok{(sleep\_sum\_quant\_qual\_mean}\SpecialCharTok{$}\NormalTok{hours\_total\_mean, }\AttributeTok{bin\_width =} \DecValTok{1}\NormalTok{) }\SpecialCharTok{+}
  \FunctionTok{labs}\NormalTok{(}\AttributeTok{title =} \StringTok{"Average Hours Asleep"}\NormalTok{,}
       \AttributeTok{x =} \StringTok{"Hours Asleep"}\NormalTok{)}
\end{Highlighting}
\end{Shaded}

\includegraphics{BellabeatCaseStudy_files/figure-latex/viz_sleep_quantity-1.pdf}

Preliminary Findings:

\begin{itemize}
\tightlist
\item
  The results appear realistic: the mean and median are 6.9 and 7.4
  hours, respectively, with 79\% of users (19 users) getting between 6
  and 9 hours of sleep on average
\end{itemize}

\hypertarget{what-quality-of-sleep-are-users-getting}{%
\paragraph{What quality of sleep are users
getting?}\label{what-quality-of-sleep-are-users-getting}}

\begin{Shaded}
\begin{Highlighting}[]
\FunctionTok{plot\_histo\_pareto}\NormalTok{(sleep\_sum\_quant\_qual\_mean}\SpecialCharTok{$}\NormalTok{pct\_awake\_mean, }\AttributeTok{bin\_width =} \FloatTok{0.05}\NormalTok{) }\SpecialCharTok{+}
  \FunctionTok{labs}\NormalTok{(}\AttributeTok{title =} \StringTok{"Percentage of Sleep Logs Spent Awake"}\NormalTok{,}
       \AttributeTok{x=} \StringTok{"\% of Time"}\NormalTok{)}
\end{Highlighting}
\end{Shaded}

\includegraphics{BellabeatCaseStudy_files/figure-latex/viz_sleep_quality-1.pdf}

\begin{Shaded}
\begin{Highlighting}[]
\FunctionTok{plot\_histo\_pareto}\NormalTok{(sleep\_sum\_quant\_qual\_mean}\SpecialCharTok{$}\NormalTok{pct\_restless\_mean, }\AttributeTok{bin\_width =} \FloatTok{0.05}\NormalTok{) }\SpecialCharTok{+}
  \FunctionTok{labs}\NormalTok{(}\AttributeTok{title =} \StringTok{"Percentage of Sleep Logs Spent Restless"}\NormalTok{,}
       \AttributeTok{x=} \StringTok{"\% of Time"}\NormalTok{)}
\end{Highlighting}
\end{Shaded}

\includegraphics{BellabeatCaseStudy_files/figure-latex/viz_sleep_quality-2.pdf}

\begin{Shaded}
\begin{Highlighting}[]
\FunctionTok{plot\_histo\_pareto}\NormalTok{(sleep\_sum\_quant\_qual\_mean}\SpecialCharTok{$}\NormalTok{pct\_asleep\_mean, }\AttributeTok{bin\_width =} \FloatTok{0.05}\NormalTok{) }\SpecialCharTok{+}
  \FunctionTok{labs}\NormalTok{(}\AttributeTok{title =} \StringTok{"Percentage of Sleep Logs Spent Asleep"}\NormalTok{,}
       \AttributeTok{x=} \StringTok{"\% of Time"}\NormalTok{)}
\end{Highlighting}
\end{Shaded}

\includegraphics{BellabeatCaseStudy_files/figure-latex/viz_sleep_quality-3.pdf}

Preliminary Findings:

\begin{itemize}
\tightlist
\item
  The majority of users appear to be getting high-quality sleep, with
  88\% of users logging between 90\% and 95\% of time as Asleep
\item
  Time spent Awake is almost nil, with 88\% of users logging less than
  5\% of time asleep. This is likely due to the programming of the
  FitBit devices, which feature an automatic sleep logging function that
  kicks in when it detects the user may be asleep, based on movement and
  heart rate data
\end{itemize}

\hypertarget{did-sleep-quantity-or-quality-improve-over-time}{%
\paragraph{Did sleep quantity or quality improve over
time?}\label{did-sleep-quantity-or-quality-improve-over-time}}

\begin{Shaded}
\begin{Highlighting}[]
\CommentTok{\# get\_time\_coefficients\_plot \textless{}{-} function(ids, timestamps, values)}
\FunctionTok{ggplot}\NormalTok{(sleep\_sum\_quant\_qual,}
       \FunctionTok{aes}\NormalTok{(}\AttributeTok{x =}\NormalTok{ sleep\_date,}
           \AttributeTok{y =}\NormalTok{ pct\_asleep)) }\SpecialCharTok{+}
  \FunctionTok{geom\_point}\NormalTok{() }\SpecialCharTok{+}
  \FunctionTok{geom\_smooth}\NormalTok{(}\AttributeTok{method =} \StringTok{"lm"}\NormalTok{,}
                \AttributeTok{se =} \ConstantTok{FALSE}\NormalTok{,}
                \AttributeTok{color =} \StringTok{"blue"}\NormalTok{) }\SpecialCharTok{+}
    \FunctionTok{stat\_cor}\NormalTok{(}\AttributeTok{mapping=}\FunctionTok{aes}\NormalTok{(}\AttributeTok{label=}\NormalTok{..rr.label..),}
             \AttributeTok{method=}\StringTok{"pearson"}\NormalTok{,}
             \AttributeTok{label.x=}\SpecialCharTok{{-}}\ConstantTok{Inf}\NormalTok{,}
             \AttributeTok{label.y=}\ConstantTok{Inf}\NormalTok{,}
             \AttributeTok{hjust =} \SpecialCharTok{{-}}\FloatTok{0.1}\NormalTok{,}
             \AttributeTok{vjust =} \FloatTok{1.1}\NormalTok{) }\SpecialCharTok{+}
  \FunctionTok{facet\_wrap}\NormalTok{(}\FunctionTok{vars}\NormalTok{(id))}
\CommentTok{\# Brute{-}force Preliminary Findings: Most lines are flat, with some (5{-}6) that are slightly down}
\end{Highlighting}
\end{Shaded}

I analysed sleep quantity and quality over time by calculating the
coefficients with time of ``total hours asleep'' and ``percentage of
time awake'', respectively:

\begin{Shaded}
\begin{Highlighting}[]
\FunctionTok{get\_time\_coefficients\_plot}\NormalTok{(sleep\_sum\_quant\_qual}\SpecialCharTok{$}\NormalTok{id,}
\NormalTok{                         sleep\_sum\_quant\_qual}\SpecialCharTok{$}\NormalTok{sleep\_date,}
\NormalTok{                         sleep\_sum\_quant\_qual}\SpecialCharTok{$}\NormalTok{minutes\_asleep) }\SpecialCharTok{+}
  \FunctionTok{labs}\NormalTok{(}\AttributeTok{title =} \StringTok{"Coefficient of Time: Total Time Asleep"}\NormalTok{)}
\end{Highlighting}
\end{Shaded}

\includegraphics{BellabeatCaseStudy_files/figure-latex/viz_sleep_quantity_over_time-1.pdf}

\begin{Shaded}
\begin{Highlighting}[]
\FunctionTok{get\_time\_coefficients\_plot}\NormalTok{(sleep\_sum\_quant\_qual}\SpecialCharTok{$}\NormalTok{id,}
\NormalTok{                           sleep\_sum\_quant\_qual}\SpecialCharTok{$}\NormalTok{sleep\_date,}
\NormalTok{                           sleep\_sum\_quant\_qual}\SpecialCharTok{$}\NormalTok{pct\_asleep) }\SpecialCharTok{+}
  \FunctionTok{labs}\NormalTok{(}\AttributeTok{title =} \StringTok{"Coefficient of Time: \% of Time Asleep"}\NormalTok{)}
\end{Highlighting}
\end{Shaded}

\includegraphics{BellabeatCaseStudy_files/figure-latex/viz_sleep_quality_over_time-1.pdf}

Preliminary Findings:

\begin{itemize}
\tightlist
\item
  The correlations for sleep quantity and quality formed
  approximately-normal distributions around zero, indicating no
  consistent increase for either variable
\item
  Sleep quantity appeared to decrease slightly, with mean and median
  correlations of -0.09 and -0.13, respectively
\item
  Sleep quantity was approximately static, with mean and median
  correlations of -0.04 and -0.02, respectively
\item
  Overall, there is nothing to indicate that users increased the
  quantity or quality of their sleep over time
\item
  The quantity and quality of sleep for this cohort of users was already
  relatively good, with the majority of users getting between 6-9 hours
  sleep, and spending upwards of 90\% of that time in a stable Asleep
  state.
\end{itemize}

\hypertarget{recommendations-2}{%
\subsubsection{Recommendations}\label{recommendations-2}}

\begin{itemize}
\item
  Most users are not logging sleep
\item
  Those that are appear to already be getting high-quality sleep
\item
  There is nothing to indicate the use of the FitBits improved the
  quantity or quality of their sleep above this high baseline
\item
  Don't market this feature as something that will improve your sleep
\item
  Instead, given the consistency of users in getting their sleep, as
  evidenced by the small correlation with time for sleep quantity and
  quality, the Sleep Tracking could be marketed more as a way of
  tracking and reinforcing the user's existing good sleep habits
\item
  As with the Readiness Score, marketing teams could frame the feature
  more as a way of getting a heads-up on days where you haven't slept
  well, since most of the time the feature would just tell users (from
  this cohort, at least) that they just had a reasonably good night's
  sleep. This would work well with the ``track existing habits''
  marketing angle: think ``You're great at giving your body the sleep it
  needs, but for those nights where things just don't work out, Ivy can
  give you guidance on whether you're up for the day, or better off
  taking it down a notch to recover.''
\item
  If it exists, highlight a manual-activation mode for the Sleep
  Tracking mode. The low amount of non-Asleep data in this data set may
  be a result of people mostly using the automatic mode, which is
  convenient, but doesn't track time spent \emph{trying} to get to
  sleep: a manual-activation feature could therefore be pushed harder as
  a sort of second feature that gets better data on how you go getting
  to sleep.
\end{itemize}

\hypertarget{resting-heart-rate-and-cardiac-coherence}{%
\subsection{Resting Heart Rate and Cardiac
Coherence}\label{resting-heart-rate-and-cardiac-coherence}}

\hypertarget{feature-overview-3}{%
\subsubsection{Feature Overview}\label{feature-overview-3}}

Although the FitBits in the data set do have heart rate and breathing
rate tracking, there is no data for resting heart rate or cardiac
coherence specifically. What I can analyse is the data on heart-rate
tracking generally, specifically:

\begin{itemize}
\tightlist
\item
  How often do people track their heart-rates, either during the day or
  during the night?
\item
  When people do track their heart-rates, what level of activity are
  they engaged in? I.e. are they only putting their trackers on to
  exercise, or do they wear them during lower-activity or sedentary
  periods?
\end{itemize}

This will help me understand whether or not the RHR and CC functions
will appeal to the user base: for instance, users who rarely wear their
existing heart-rate monitors, or only wear them during intense exercise,
may be less likely to care about their longer-term cardiac performance.

\hypertarget{analysis-3}{%
\subsubsection{Analysis}\label{analysis-3}}

\hypertarget{how-often-do-people-track-their-heart-rate}{%
\paragraph{How often do people track their
heart-rate?}\label{how-often-do-people-track-their-heart-rate}}

\hypertarget{nighttime-usage}{%
\subparagraph{Nighttime Usage}\label{nighttime-usage}}

From my previous analysis on heart-rate data, we know that overnight
heart-rate tracking is split, with 50\% (7 users) logging sleep on 35\%
to 90\% of nights, and the other 50\% logging sleep on 0\% to 15\% of
nights.

\hypertarget{daytime-usage}{%
\subparagraph{Daytime Usage}\label{daytime-usage}}

I can adapt the code from the nighttime analysis to investigate daytime
usage:

\begin{Shaded}
\begin{Highlighting}[]
\CommentTok{\# Find nights where enough heart{-}rate data was logged to imply the user slept with their FitBit on}

\NormalTok{sec2hour }\OtherTok{\textless{}{-}} \DecValTok{1} \SpecialCharTok{/} \DecValTok{3600}
\CommentTok{\# Sleep range set to between 10pm and 6am}
\NormalTok{period\_stt\_hour }\OtherTok{\textless{}{-}} \DecValTok{6}
\NormalTok{period\_end\_hour }\OtherTok{\textless{}{-}} \DecValTok{22}
\CommentTok{\# Typical poll rate is 1{-}20 seconds: increased to 60 to allow for device moving out of position during daytime movement}
\NormalTok{max\_poll\_gap\_secs }\OtherTok{\textless{}{-}} \DecValTok{60}
\CommentTok{\# Minimum six hours must be logged to be counted}
\NormalTok{min\_hours\_logged }\OtherTok{\textless{}{-}} \DecValTok{6}
\NormalTok{periods\_total }\OtherTok{\textless{}{-}} \DecValTok{33}
\NormalTok{logging\_days }\OtherTok{\textless{}{-}} \ConstantTok{TRUE}

\ControlFlowTok{if}\NormalTok{ (logging\_days) \{}
\NormalTok{  periods\_logged }\OtherTok{\textless{}{-}}\NormalTok{ heartrate\_src\_seconds\_tall }\SpecialCharTok{\%\textgreater{}\%}
    \FunctionTok{mutate}\NormalTok{(}\AttributeTok{yday =} \FunctionTok{ifelse}\NormalTok{((}\FunctionTok{hour}\NormalTok{(heart\_rate\_second) }\SpecialCharTok{\textgreater{}=}\NormalTok{ period\_stt\_hour }\SpecialCharTok{\&} \FunctionTok{hour}\NormalTok{(heart\_rate\_second) }\SpecialCharTok{\textless{}}\NormalTok{ period\_end\_hour),}
                         \FunctionTok{yday}\NormalTok{(heart\_rate\_second),}
                         \ConstantTok{NA}\NormalTok{))}
\NormalTok{\} }\ControlFlowTok{else}\NormalTok{ \{}
  \CommentTok{\# For night logging, set yday to the day before, i.e. "Night of [yday]"}
\NormalTok{  periods\_logged }\OtherTok{\textless{}{-}}\NormalTok{ heartrate\_src\_seconds\_tall }\SpecialCharTok{\%\textgreater{}\%}
    \FunctionTok{mutate}\NormalTok{(}\AttributeTok{yday =} \FunctionTok{ifelse}\NormalTok{(}\FunctionTok{hour}\NormalTok{(heart\_rate\_second) }\SpecialCharTok{\textgreater{}=}\NormalTok{ period\_stt\_hour,}
                         \FunctionTok{yday}\NormalTok{(heart\_rate\_second),}
                         \FunctionTok{ifelse}\NormalTok{(}\FunctionTok{hour}\NormalTok{(heart\_rate\_second) }\SpecialCharTok{\textless{}}\NormalTok{ period\_end\_hour,}
                                \FunctionTok{yday}\NormalTok{(heart\_rate\_second) }\SpecialCharTok{{-}} \DecValTok{1}\NormalTok{,}
                                \ConstantTok{NA}\NormalTok{)))}
\NormalTok{\}}

\NormalTok{periods\_logged }\OtherTok{\textless{}{-}}\NormalTok{ periods\_logged }\SpecialCharTok{\%\textgreater{}\%}
  \CommentTok{\# Only consider logs within the period of interest}
  \FunctionTok{filter}\NormalTok{(}\SpecialCharTok{!}\FunctionTok{is.na}\NormalTok{(yday)) }\SpecialCharTok{\%\textgreater{}\%}
  \CommentTok{\# Group by id so that difftime only compares timestamps from the same user}
  \FunctionTok{group\_by}\NormalTok{(id) }\SpecialCharTok{\%\textgreater{}\%}
  \FunctionTok{mutate}\NormalTok{(}\AttributeTok{time\_since\_last\_poll =} \FunctionTok{as.double}\NormalTok{(}\FunctionTok{difftime}\NormalTok{(heart\_rate\_second, }\FunctionTok{lag}\NormalTok{(heart\_rate\_second), }\AttributeTok{units =} \StringTok{"secs"}\NormalTok{))) }\SpecialCharTok{\%\textgreater{}\%}
  \CommentTok{\# Remove any logs that are followed by too long a gap}
  \FunctionTok{filter}\NormalTok{(time\_since\_last\_poll }\SpecialCharTok{\textless{}=}\NormalTok{ max\_poll\_gap\_secs) }\SpecialCharTok{\%\textgreater{}\%}
  \FunctionTok{group\_by}\NormalTok{(id, yday) }\SpecialCharTok{\%\textgreater{}\%}
  \CommentTok{\# Find the total time logged as the sum of time between valid logs}
  \FunctionTok{summarize}\NormalTok{(}\AttributeTok{logged\_time\_hours =} \FunctionTok{sum}\NormalTok{(time\_since\_last\_poll) }\SpecialCharTok{*}\NormalTok{ sec2hour) }\SpecialCharTok{\%\textgreater{}\%}
  \CommentTok{\# Count up the nights where each user logged sufficient time to be considered asleep with their fitbit}
  \FunctionTok{group\_by}\NormalTok{(id) }\SpecialCharTok{\%\textgreater{}\%}
  \FunctionTok{summarize}\NormalTok{(}\AttributeTok{periods =} \FunctionTok{sum}\NormalTok{(logged\_time\_hours }\SpecialCharTok{\textgreater{}}\NormalTok{ min\_hours\_logged),}
            \AttributeTok{periods\_pct =}\NormalTok{ periods }\SpecialCharTok{/}\NormalTok{ periods\_total)}
\end{Highlighting}
\end{Shaded}

\begin{Shaded}
\begin{Highlighting}[]
\CommentTok{\# Plot histogram showing who logged what fraction of their nights}
\FunctionTok{plot\_histo\_pareto}\NormalTok{(periods\_logged}\SpecialCharTok{$}\NormalTok{periods\_pct, }\AttributeTok{bin\_width =} \FloatTok{0.05}\NormalTok{) }\SpecialCharTok{+}
  \FunctionTok{labs}\NormalTok{(}\AttributeTok{title =} \StringTok{"Percentage of Periods Logged"}\NormalTok{,}
       \AttributeTok{subtitle =} \StringTok{"Source: Heart Rate Data"}\NormalTok{)}
\end{Highlighting}
\end{Shaded}

\includegraphics{BellabeatCaseStudy_files/figure-latex/viz_daytime_heartrate_usage-1.pdf}

Preliminary Findings:

\begin{itemize}
\tightlist
\item
  Daytime logging is much more popular than nighttime logging
\item
  50\% of users (7 users) logged 6 hours on between 80\% to 95\% of days
\item
  The other 50\% of users logged on between 0\% and 75\% of days
\end{itemize}

\hypertarget{what-levels-of-activity-do-people-typically-track}{%
\paragraph{What levels of activity do people typically
track?}\label{what-levels-of-activity-do-people-typically-track}}

First the naive approach: What percentage of various users' time is
spent in what heart-rate zone?

\begin{Shaded}
\begin{Highlighting}[]
\CommentTok{\# From the Charge HR manual:}
\CommentTok{\#   Sedentary = 0{-}50\% Max HR}
\CommentTok{\#   Lightly Active = 50{-}70\% Max HR}
\CommentTok{\#   Fairly Active = 70{-}85\% Max HR}
\CommentTok{\#   Very Active = 85{-}100\% Max HR}
\CommentTok{\# Max HR = 220 {-} age}
\CommentTok{\# Arbitrarily set median age at 30 for initial analysis}
\NormalTok{assumed\_median\_age }\OtherTok{\textless{}{-}} \DecValTok{30}
\NormalTok{max\_heart\_rate }\OtherTok{\textless{}{-}} \DecValTok{220} \SpecialCharTok{{-}}\NormalTok{ assumed\_median\_age}
\NormalTok{hr\_zone\_1 }\OtherTok{\textless{}{-}} \FloatTok{0.5} \SpecialCharTok{*}\NormalTok{ max\_heart\_rate}
\NormalTok{hr\_zone\_2 }\OtherTok{\textless{}{-}} \FloatTok{0.7} \SpecialCharTok{*}\NormalTok{ max\_heart\_rate}
\NormalTok{hr\_zone\_3 }\OtherTok{\textless{}{-}} \FloatTok{0.85} \SpecialCharTok{*}\NormalTok{ max\_heart\_rate}

\NormalTok{hr\_zone\_data }\OtherTok{\textless{}{-}}\NormalTok{ heartrate\_src\_seconds\_tall }\SpecialCharTok{\%\textgreater{}\%}
  \FunctionTok{mutate}\NormalTok{(}\AttributeTok{hr\_zone =} \FunctionTok{ifelse}\NormalTok{(heart\_rate }\SpecialCharTok{\textgreater{}=} \DecValTok{0} \SpecialCharTok{\&}\NormalTok{ heart\_rate }\SpecialCharTok{\textless{}}\NormalTok{ hr\_zone\_1, }\StringTok{"sedentary"}\NormalTok{,}
                   \FunctionTok{ifelse}\NormalTok{(heart\_rate }\SpecialCharTok{\textgreater{}=}\NormalTok{ hr\_zone\_1 }\SpecialCharTok{\&}\NormalTok{ heart\_rate }\SpecialCharTok{\textless{}}\NormalTok{ hr\_zone\_2, }\StringTok{"lightly\_active"}\NormalTok{,}
                   \FunctionTok{ifelse}\NormalTok{(heart\_rate }\SpecialCharTok{\textgreater{}=}\NormalTok{ hr\_zone\_2 }\SpecialCharTok{\&}\NormalTok{ heart\_rate }\SpecialCharTok{\textless{}}\NormalTok{ hr\_zone\_3, }\StringTok{"fairly\_active"}\NormalTok{,}
                   \FunctionTok{ifelse}\NormalTok{(heart\_rate }\SpecialCharTok{\textgreater{}=}\NormalTok{ hr\_zone\_3, }\StringTok{"very\_active"}\NormalTok{, }\ConstantTok{NA}\NormalTok{))))) }\SpecialCharTok{\%\textgreater{}\%}
  \FunctionTok{group\_by}\NormalTok{(id) }\SpecialCharTok{\%\textgreater{}\%}
  \FunctionTok{summarize}\NormalTok{(}\AttributeTok{sedentary =} \FunctionTok{sum}\NormalTok{(hr\_zone }\SpecialCharTok{==} \StringTok{"sedentary"}\NormalTok{),}
            \AttributeTok{lightly\_active =} \FunctionTok{sum}\NormalTok{(hr\_zone }\SpecialCharTok{==} \StringTok{"lightly\_active"}\NormalTok{),}
            \AttributeTok{fairly\_active =} \FunctionTok{sum}\NormalTok{(hr\_zone }\SpecialCharTok{==} \StringTok{"fairly\_active"}\NormalTok{),}
            \AttributeTok{very\_active =} \FunctionTok{sum}\NormalTok{(hr\_zone }\SpecialCharTok{==} \StringTok{"very\_active"}\NormalTok{))}

\CommentTok{\#Convert data to long{-}format for plotting as a histogram}
\NormalTok{hr\_zone\_order }\OtherTok{=} \FunctionTok{c}\NormalTok{(}\StringTok{"sedentary"}\NormalTok{, }\StringTok{"lightly\_active"}\NormalTok{, }\StringTok{"fairly\_active"}\NormalTok{, }\StringTok{"very\_active"}\NormalTok{)}
\NormalTok{hr\_zone\_data\_long }\OtherTok{\textless{}{-}}\NormalTok{ hr\_zone\_data }\SpecialCharTok{\%\textgreater{}\%}
\NormalTok{  tidyr}\SpecialCharTok{::}\FunctionTok{gather}\NormalTok{(}\AttributeTok{key =} \StringTok{"hr\_zone"}\NormalTok{, }\AttributeTok{value =} \StringTok{"count"}\NormalTok{, }\SpecialCharTok{{-}}\NormalTok{id) }\SpecialCharTok{\%\textgreater{}\%}
  \FunctionTok{mutate}\NormalTok{(}\AttributeTok{hr\_zone =} \FunctionTok{factor}\NormalTok{(hr\_zone, }\AttributeTok{levels =}\NormalTok{ hr\_zone\_order))}
\CommentTok{\# }
\CommentTok{\# \# Generate data to order IDs by average "Very Active" time}
\CommentTok{\# mean\_very\_active\_time\_by\_id \textless{}{-} mean\_daily\_intensities \%\textgreater{}\%}
\CommentTok{\#   filter(intensity == "very\_active") \%\textgreater{}\%}
\CommentTok{\#   arrange(mean\_minutes)}
\CommentTok{\# }
\CommentTok{\# \#Convert ID to factor ordered by average "Very Active" time}
\CommentTok{\# mean\_daily\_intensities$id \textless{}{-} factor(}
\CommentTok{\#   mean\_daily\_intensities$id,}
\CommentTok{\#   levels = mean\_very\_active\_time\_by\_id$id)}
\CommentTok{\# }
\CommentTok{\# \# Comment out this line to include Sedentary time in the graph}
\CommentTok{\# \#mean\_daily\_intensities \textless{}{-} mean\_daily\_intensities \%\textgreater{}\% filter(intensity != "sedentary")}
\CommentTok{\# }
\end{Highlighting}
\end{Shaded}

\begin{Shaded}
\begin{Highlighting}[]
\FunctionTok{ggplot}\NormalTok{(hr\_zone\_data\_long, }\FunctionTok{aes}\NormalTok{(}\AttributeTok{x=}\NormalTok{ id, }\AttributeTok{y=}\NormalTok{ count, }\AttributeTok{fill=}\NormalTok{ hr\_zone)) }\SpecialCharTok{+}
  \FunctionTok{geom\_bar}\NormalTok{(}\AttributeTok{stat =} \StringTok{"identity"}\NormalTok{,}
           \AttributeTok{position =} \StringTok{"fill"}\NormalTok{) }\SpecialCharTok{+}
  \FunctionTok{theme}\NormalTok{(}\AttributeTok{axis.text.x =} \FunctionTok{element\_text}\NormalTok{(}\AttributeTok{angle =} \DecValTok{45}\NormalTok{, }\AttributeTok{hjust =} \DecValTok{1}\NormalTok{)) }\SpecialCharTok{+}
  \FunctionTok{labs}\NormalTok{(}\AttributeTok{title=}\StringTok{"Count of HR Logs by HR Zone"}\NormalTok{,}
       \AttributeTok{x =} \StringTok{"User ID"}\NormalTok{,}
       \AttributeTok{y =} \StringTok{"Count"}\NormalTok{)}
\end{Highlighting}
\end{Shaded}

\includegraphics{BellabeatCaseStudy_files/figure-latex/viz_intensity_levels-1.pdf}

Preliminary Findings:

\begin{itemize}
\tightlist
\item
  The majority of logs for all but one user are in the ``Sedentary''
  category, which equates to less than 95 bpm
\item
  It is fair to assume therefore that users are not only using their
  heart-rate trackers for exercise
\item
  There should be ample data to determine a user's resting heart rate
  from the daytime data, even if there is little nighttime data
\end{itemize}

\hypertarget{recommendations-3}{%
\subsubsection{Recommendations}\label{recommendations-3}}

\begin{itemize}
\tightlist
\item
  In order to match current user behaviours, the marketing for the
  Resting Heart Rate should emphasise daytime usage: for instance,
  advertise that the feature will help understand your resting heart
  rate during your work day,
\item
  The data indicates that users typically do not log overnight, which
  may impact the performance of this feature: part of the usefulness of
  the feature comes from calibrating the data based on overnight
  heart-rate, so if users are not wearing overnight, the resting heart
  rate value may be skewed upward s as it is based primarily on daytime
  usage
\item
  The marketing should therefore encourage users to wear overnight,
  highlighting the synergy between this and the Readiness Score feature
  to maximise the likelihood that users wear overnight, and therefore
  the accuracy of both features
\item
  If users do not wear their devices overnight, they are unlikely to see
  the benefits and may consider other, less-featured but cheaper devices
  for their next purchase
\end{itemize}

\hypertarget{features-without-data}{%
\subsection{Features without Data}\label{features-without-data}}

\hypertarget{feature-overview-4}{%
\subsubsection{Feature Overview}\label{feature-overview-4}}

Several functions and features of the Ivy are not present on the FitBits
in the data set. Given the lack of data, it's not possible to directly
analyse existing user behaviours with respect to these features, which
include:

\begin{itemize}
\tightlist
\item
  Respiratory Rate
\item
  Hydration tracking
\item
  Mindfulness Minute Tracking
\item
  Menstrual Cycle
\item
  Wellness Score
\end{itemize}

\hypertarget{recommendations-4}{%
\subsubsection{Recommendations}\label{recommendations-4}}

\begin{itemize}
\tightlist
\item
  These features can all be marketed as improvements and additions over
  the functionality of existing FitBits.
\item
  As discussed with regards to the Readiness score, marketing for any
  feature that requires overnight wear, like the Cardiac Coherence
  tracking, should take into account the low utilisation of existing
  overnight functions like sleep tracking seen in the existing data.
\end{itemize}

\hypertarget{todo-step-5-share}{%
\section{TODO Step 5: Share}\label{todo-step-5-share}}

In which I present all of my key findings and their supporting
visualisations

\hypertarget{there-is-a-wide-range-of-exercise-habits-in-the-cohort}{%
\subsection{There is a wide range of exercise habits in the
cohort}\label{there-is-a-wide-range-of-exercise-habits-in-the-cohort}}

\begin{itemize}
\tightlist
\item
\end{itemize}

\hypertarget{findings}{%
\subsection{Findings}\label{findings}}

\hypertarget{todo-step-6-act}{%
\section{TODO Step 6: Act}\label{todo-step-6-act}}

In which I summarise my recommendations

\begin{itemize}
\tightlist
\item
  Push hard on any features not used much as a value-add to the customer
\end{itemize}

\end{document}
